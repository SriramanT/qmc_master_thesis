%%%%%%%%%%%%%%%%%%%%%%%%%%%%%%%%%%%%%%%%%%%%%%%%%%%%%%%%%%%%%%%%%%%%%%
%     File: ExtendedAbstract_concl.tex                               %
%     Tex Master: ExtendedAbstract.tex                               %
%                                                                    %
%     Author: Andre Calado Marta                                     %
%     Last modified : 27 Dez 2011                                    %
%%%%%%%%%%%%%%%%%%%%%%%%%%%%%%%%%%%%%%%%%%%%%%%%%%%%%%%%%%%%%%%%%%%%%%
% The main conclusions of the study presented in short form.
%%%%%%%%%%%%%%%%%%%%%%%%%%%%%%%%%%%%%%%%%%%%%%%%%%%%%%%%%%%%%%%%%%%%%%

\section{Conclusions}
\label{sec:concl}

In this work, we investigated edge-magnetism in TMD nanoribbons.
We set up a mean field theory, arriving at the mean field phase diagram of Fig.(\ref{fig:mfPhaseDiag0}).
In the zero temperature case, we found two phase transitions.
We explain the mechanism that is behind them by looking at how the band structure changes in mean field as a function of the on-site (intra-orbital) interaction $U$, and inverse temperature $\beta$.
Then, we used our own implementation of the determinant QMC algorithm to analyze the same type of systems.
Based on similar studies for graphene, we looked for long range magnetic order by analyzing the $S^z$ spin-spin correlation functions.
We found that, while the sign problem limits our simulations, it does not impede us to extract conclusions from them.
Because of the presence of 3 orbitals, we must run the code for much longer to simulate systems of size comparable to the ones usually done in simulations involving graphene nanoribbons, and for which long range order can be accurately investigated.
Our preliminary results for the orbital-resolved spin-spin correlation functions along the rows of the ribbon are promising. For example, correlations along the edge rows tend to larger than those along the bulk rows.
We shall continue this work in an effort to determine whether long range order appears.
