%%%%%%%%%%%%%%%%%%%%%%%%%%%%%%%%%%%%%%%%%%%%%%%%%%%%%%%%%%%%%%%%%%%%%%
%     File: ExtendedAbstract_abstr.tex                               %
%     Tex Master: ExtendedAbstract.tex                               %
%                                                                    %
%     Author: Andre Calado Marta                                     %
%     Last modified : 2 Dez 2011                                     %
%%%%%%%%%%%%%%%%%%%%%%%%%%%%%%%%%%%%%%%%%%%%%%%%%%%%%%%%%%%%%%%%%%%%%%
% The abstract of should have less than 500 words.
% The keywords should be typed here (three to five keywords).
%%%%%%%%%%%%%%%%%%%%%%%%%%%%%%%%%%%%%%%%%%%%%%%%%%%%%%%%%%%%%%%%%%%%%%

%%
%% Abstract
%%
\begin{abstract}
The discovery of two-dimensional materials, has renewed interest in the many-electron problem since electron-electron interactions  play an important role in the description of many of the properties of these systems.
Moreover, the rapid increase in computational power, and algorithmic sophistication in the last few decades has made it possible to attack many problems in the field that were previously intractable.
In this work, we focus on emerging edge-state magnetism in Transition Metal Dichalcogenides nanoribbons.
We consider a recently introduced symmetry-based tight-binding model, found to capture edge state-related properties of the system, and we generalize it to interacting case by considering intra-orbital Hubbard-type interactions.
Our approach to this problem is two-fold.
We start by performing original numerical mean field calculations and build an approximate physical picture of the system at hand.
Then, we use our own implementation of the unbiased, state-of-the-art Determinant Quantum Monte Carlo algorithm to simulate this interacting, quantum many-fermion system.
\\
%%
%% Keywords (max 5)
%%

\noindent{{\bf Keywords:}} Two-dimensional Materials, Hubbard Model, Strongly Correlated Electrons, Transition Metal Dichalcogenide Nanoribbons, Mean Field Theory, Determinant or Auxiliary Field Quantum Monte Carlo \\

\end{abstract}

