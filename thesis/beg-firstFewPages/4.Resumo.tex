\begin{resumo}

O comportamento coletivo tem estado na linha da frente da investigação feita em física da matéria condensada durante grande parte do último século.
Esta era dos sistemas complexos trouxe-nos um conceito denominado emergência que tem, atualmente, uma natureza transversal.
Em particular, em sistemas de eletrões fortemente correlacionados, fenómenos emergentes conduzem a uma enorme variedade de propriedades exóticas que desafiam a intuição.

A descoberta de materiais bidimensionais renovou o interesse no problema a $N$ eletrões dado que as interações eletrão-eletrão têm um papel determinante na descrição de muitas das propriedades destes sistemas.
Além disso, o rápido aumento de poder computacional e de sofisticação algorítmica tornou possível atacar problemas que eram até então inacessíveis.
Neste trabalho, focamo-nos numa classe particular de materiais bidimensionais, mostrando uma grande riqueza de propriedades óticas e eletrónicas: os dicalcogenetos de metais de transição.

Neste contexto, investigamos a emergência de magnetismo associado a estados de fronteira, num tipo de nanoestrutura chamado de nanofita, dada a sua semelhança a uma fita, por ser muito mais longa numa direção que na outra.
Para estudar este fenómeno, consideramos um modelo de tight-binding baseado em considerações de simetria que captura a maioria das propriedades relacionadas com a \say{física de fronteiras} do problema.
Depois, generalizamos este modelo para o caso interatuante, considerando interações do tipo Hubbard intra-orbitais.

A nossa abordagem ao problema divide-se em duas partes.
Começamos por levar a cabo cálculos numéricos originais na aproximação de campo médio para construir uma imagem física aproximada do sistema.
Depois, usamos a nossa própria implementação de um algoritmo de ponta, livre de enviesamento estatístico - o método do determinante de Monte Carlo Quântico - para simular o sistema interatuante a $N$ fermiões em estudo.
\end{resumo}
\vspace{-2.5cm}