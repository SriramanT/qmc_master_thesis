\begin{abstract}

Cooperative behavior has been at the forefront of condensed matter physics research for the most part of the last half-century.
This complex systems era has brought about a concept called emergence, which is now overarching.
In particular, in strongly correlated electron systems, emergent phenomena lead to a variety of exotic properties that defy intuition.

The discovery of two-dimensional materials, has renewed interest in the many-electron problem since electron-electron interactions  play an important role in the description of many of the properties of these systems.
Moreover, the rapid increase in computational power, and algorithmic sophistication has made it possible to attack many problems in the field that were previously intractable.
In this work, we focus on a particular class of two-dimensional materials showing a wealth of fascinating electronic and optical properties: Transition Metal Dichalcogenides.

We investigate emerging edge-state magnetism in a type of nanostructure called a nanoribbon, so called because it resembles a ribbon, being much longer on direction than on the other.
To study this phenomenon, we consider a recently introduced symmetry-based tight-binding model that is found to capture most properties related to the edge physics of the problem.
Then, we generalize it to the interacting case by considering intra-orbital Hubbard-type interactions.

Our approach to this problem is two-fold.
We start by performing original numerical mean field calculations and build an approximate physical picture of the system at hand.
Then, we use our own implementation of the unbiased, state-of-the-art Determinant Quantum Monte Carlo algorithm to simulate this interacting, quantum many-fermion system.

\end{abstract}