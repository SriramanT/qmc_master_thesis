\section{Motivation}
\label{sec:int_motivation}

Condensed matter physics is concerned with the emergence of the properties of quantum materials from complexity. The central concept within this approach is that of symmetry breaking. When a system condenses into a phase, there is a decrease in symmetry. A simple example is the transition from a gas to a solid. Statistically, any point within a gas is equivalent, that is, on average, the surroundings of all points look similar. Formally, the system is said to be fully  translationally invariant. On the other hand, in a solid, a point is only equivalent to a discrete set of points, which can be reached starting from the original point by translations by lattice vectors. Thus, the system is invariant under a discrete set of translations, rather than a continuous one as in the case of the gas.

The broad scope of condensed matter comes from the sheer number of possibilities that this approach affords. One can study the emergence of magnetism, superconductivity, or superfluidity, just to name a few. The search for order relies on symmetry ideas beyond condensed matter. The framework that is used to carry out this search is the Landau-Ginzburg theory of phase transitions. The theory gives a prescription to discover phase transitions: identify an order parameter reflecting the underlying symmetry of the system and then minimize a free energy function to deduce conditions for the symmetry to manifest itself. The drawback of this approach is that it might be difficult to identify an order parameter in the first place. Moreover, even if we do manage to find one, the usual procedure may be impossible to perform. It can easily happen that the degree of complexity of the order parameter is simply too high.

Not all phase transitions can be described by the Landau-Ginzburg paradigm. On the one hand, there are systems where topological order arises as in fractional quantum Hall effect. On the other, for strongly correlated systems, there are phenomena which emerge specifically due to the interacting nature of the problem.