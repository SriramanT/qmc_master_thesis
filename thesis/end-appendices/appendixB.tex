\chapter{Computing the partition function for a quadratic Hamiltonian}
\label{ap:Zquadratic}

\pagebreak

Let us start by restating the result we want to prove.

If $\mathcal{H} = \bm c^\dagger \bm H \bm c$, where $\bm H$ is a $N \times N$ Hermitian matrix, then we have that

\begin{equation}\label{eq:apZquadratic}
\text{Tr} \big[ e^{-\beta \mathcal{H} } \big] = \prod_{i=1}^N ( 1 + e^{-\beta \lambda_{k_i} } ) ,
\end{equation}
where $\lambda_{k_i}$ are the eigenvalues of $\bm H$. 

We will now prove equation (\ref{eq:apZquadratic}). Without loss of generality, let us consider $\bm H$ to be diagonal. Then, its eigenvalues coincide with the diagonal entries, so that $\bm H = \text{diag}(\lambda_{k_i} )$. The quadratic Hamiltonian may then be  diagonalized

\begin{equation*}
\mathcal{H} = {\bm c}^\dagger \text{diag} (\lambda_{k_1}, \lambda_{k_2}, .., \lambda_{k_N}) \bm c = \sum_{i=1}^N \lambda_{k_i} n_{k_i}
\end{equation*}

We continue by induction. When $N=1$, we have

\begin{equation}
\text{Tr} (e^{-\beta\mathcal{H} } ) = \left\langle 0 \left| e^{-\beta \lambda_{k_1} n_{k_1}}  \right| 0 \right\rangle + \left\langle 1 \left| e^{-\beta \lambda_{k_1} n_{k_1}}   \right| 1 \right\rangle = 1 + e^{-\beta \lambda_{k_1} }
\end{equation}

Assuming that for $N-1$:

\begin{equation*}
\text{Tr} \big[ e^{-\beta \sum_{i=1}^{N-1} \lambda_{k_i} n_{k_i} } \big] = \prod_{i=1}^{N-1} ( 1 + e^{-\beta \lambda_{k_i} } )
\end{equation*}
we can compute the trace for $i$ going up to $N$.

\begin{equation*}
\begin{split}
&\text{Tr} \big[ e^{-\beta \sum_{i=1}^{N-1} \lambda_{k_i} n_{k_i} } \big] = \sum_{\{ k_i \}_{i=1}^{N} } \left\langle \psi_1^{k_1} \psi_2^{k_2} ... \psi_N^{k_N} \left| e^{-\beta \sum_{i=1}^N \lambda_{k_i} n_{k_i}}  \right| \psi_1^{k_1} \psi_2^{k_2} ... \psi_N^{k_N} \right\rangle \\
&= \sum_{\{ k_i \}_{i=1}^{N-1} } \bigg( \left\langle \{\psi_i^{k_i}\} 0 \left| e^{-\beta \sum_{i=1}^N \lambda_{k_i} n_{k_i}} e^{-\beta \lambda_{k_N} n_{k_N}} \right| \{\psi_i^{k_i}\} 0 \right\rangle + \left\langle \{\psi_i^{k_i}\} 1 \left| e^{-\beta \sum_{i=1}^N \lambda_{k_i} n_{k_i}} e^{-\beta \lambda_{k_N} n_{k_N}} \right| \{\psi_i^{k_i}\} 1 \right\rangle \bigg) \\
&= (1 + e^{-\beta \lambda_{k_N} } ) \sum_{\{ k_i \}_{i=1}^{N-1} } \left\langle \{\psi_i^{k_i}\} \left| e^{-\beta \lambda_{k_i} n_{k_i}} \right| \{\psi_i^{k_i}\} \right\rangle \\
&= (1 + e^{-\beta \lambda_{k_N} } ) \prod_{i=1}^{N-1} (1 + e^{-\beta \lambda_{k_i} } ) \\
&= \prod_{i=1}^{N} (1 + e^{-\beta \lambda_{k_i} } )
\end{split}
\end{equation*}

To complete the proof we note that for any $\bm H$, there exists a unitary matrix $\bm Q$, such that $\bm Q^T \bm H \bm Q = \bm \Lambda = \text{diag}(\lambda_{k_i})$. Let $\tilde{\bm c} = \bm Q \bm c$, and $\tilde{n_i} = \tilde{c_i}^\dagger \tilde{c_i}$. Then, we find

\begin{equation*}
\mathcal{H} = \bm c^\dagger \bm H \bm c = \bm \tilde{\bm c}^\dagger \bm \Lambda \tilde{\bm c} = \sum_{i=1}^N \lambda_{k_i} \tilde{n}_{k_i}
\end{equation*}

The trace is independent of the choice of basis functions. Thus, we have

\begin{equation*}
\begin{split}
\text{Tr} ( e^{-\beta \mathcal{H} } ) &= \text{Tr} \bigg( \prod_{i=1}^N e^{-\beta \lambda_{k_i} \tilde{n}_{k_i} } \bigg) \\
&= \prod_{i=1}^N \bigg( 1 + e^{-\beta \lambda_{k_i} } \bigg) \quad\quad \qedsymbol
\end{split}
\end{equation*}