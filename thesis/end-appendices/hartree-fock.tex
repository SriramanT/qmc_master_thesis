\chapter{Hartree-Fock Approximation and the Self Consistent Field Method}
\label{ap:hartree-fock}

\pagebreak

In the mean field approximation, the quartic term of the interaction part of the Hamiltonian

\begin{equation*}
V_{\text{int}} = \frac{1}{2} V^{\nu\mu}_{\nu'\mu'} c_\nu^\dagger c_\mu^\dagger c_{\mu'} c_{\nu'} ,
\end{equation*} 
becomes a sum of all possible 2-body terms (note that terms of the type $\left\langle cc \right\rangle$ and $\left\langle c^\dagger c^\dagger \right\rangle$ must vanish since they do not conserve the number of particles).

\begin{equation}\label{eq:c_mft}
c_\nu^\dagger c_\mu^\dagger c_{\mu'} c_{\nu'} \approx - \left\langle c_\nu^\dagger c_{\mu'} \right\rangle  c_{\mu}^\dagger c_{\nu'} - \left\langle c_{\mu}^\dagger c_{\nu'} \right\rangle c_{\nu}^\dagger c_{\mu'} + \left\langle c_{\nu}^\dagger c_{\nu'} \right\rangle  c_{\mu}^\dagger c_{\mu'} + \left\langle c_{\mu}^\dagger c_{\mu'} \right\rangle  c_{\nu}^\dagger c_{\nu'} ,
\end{equation}
where we ignored the constant terms which are unimportant in the Hamiltonian, in what concerns the dynamics. 

This Hartree-Fock, or mean field approximation is slightly tricky to obtain. It requires one to be precise about what the meaning of the mean field approximation is in terms of creation and annihilation operators. In mean field theory, we assume that the operator

\begin{equation}
\rho_{\mu\mu'} = c_{\mu}^\dagger c_{\mu'}
\end{equation}
is close to its average, so that we neglect second order terms in the fluctuations $\delta \rho_{\mu\mu'}$, i.e. $\rho_{\mu\mu'}$ is \say{large} only when its average is nonzero, otherwise it is negligibly small. Thus, for most combinations of indices, this operator will vanish. We follow the usual mean field procedure of writing the original operator as a deviation plus an average

\begin{equation}\label{eq:hartree}
c_{\nu}^\dagger \bigg( c_\mu^\dagger c_{\mu'} - \left\langle c_\mu^\dagger c_{\mu'} \right\rangle \bigg) c_{\nu'} + c_{\nu}^\dagger c_{\nu'} \left\langle c_\nu^\dagger c_{\nu'} \right\rangle
\end{equation}

Then we note that if $\nu' \neq \mu$, we can commute $c_{\nu'}$ with the parenthesis. But this is true except in a set of measure zero. In the thermodynamic limit $N \rightarrow \infty$, the number of allowed $\bm k$-states is very large, and if we take a continuum limit in which the set of possible $\bm k$-states becomes dense, then the commutation becomes exact. Repeating the procedure of writing (\ref{eq:hartree}) replacing $c_\nu^\dagger c_{\nu'} \mapsto c_\nu^\dagger c_{\nu'} - \left\langle c_\nu^\dagger c_{\nu'} \right\rangle + \left\langle c_\nu^\dagger c_{\nu'} \right\rangle $, we obtain

\begin{equation}\label{eq:mf}
\underbrace{\big( c_\nu^\dagger c_{\nu'} - \left\langle c_\nu^\dagger c_{\nu'} \right\rangle \big) \big( c_\mu^\dagger c_{\mu'} - \left\langle c_\mu^\dagger c_{\mu'} \right\rangle \big)}_{\propto \, \delta \rho_{\mu\mu'} \, \delta \rho_{\nu\nu'} \rightarrow 0} + c_\nu^\dagger c_{\nu'} \left\langle c_\mu^\dagger c_{\mu'} \right\rangle + c_\mu^\dagger c_{\mu'} \left\langle c_\nu^\dagger c_{\nu'} \right\rangle - \left\langle c_\mu^\dagger c_{\mu'} \right\rangle \left\langle c_\nu^\dagger c_{\nu'} \right\rangle
\end{equation}

But this result is not complete. This is only the so called Hartree or direct term. Due to identical nature of the interacting electrons, we must consider an analogous contribution for $\left\langle c_\nu^\dagger c_{\mu'} \right\rangle$ finite. We start by exchanging the first two operators: 

\begin{equation}
c_\nu^\dagger c_\mu^\dagger c_{\mu'} c_{\nu'} = - c_\mu^\dagger c_\nu^\dagger c_{\mu'} c_{\nu'}
\end{equation}
Then we proceed in exactly the same manner as before. The result is analogous, but a minus sign appears and we must switch $\mu \leftrightarrow \nu$:

\begin{equation}
- c_\mu^\dagger c_{\nu'} \left\langle c_\nu^\dagger c_{\mu'} \right\rangle \\
- c_\nu^\dagger c_{\mu'} \left\langle c_\mu^\dagger c_{\nu'} \right\rangle + \left\langle c_\nu^\dagger c_{\mu'} \right\rangle \left\langle c_\mu^\dagger c_{\nu'} \right\rangle
\end{equation}

Ignoring the constant terms of the type $\left\langle c^\dagger c \right\rangle \left\langle c^\dagger c \right\rangle$, we recover equation (\ref{eq:c_mft}).

Now we can simply substitute the mean field expansion of equation (\ref{eq:c_mft}) in the second term to  obtain the last term that is subtracted in equation (\ref{eq:startingHamiltonian}) (we omit the boldface on the $\bm k$'s solely in the following equation, but keep in mind that they are vectors):

\begin{equation}\label{eq:mean_field}
\begin{split}
&\frac{1}{2} \sum_{\substack{ k_1 k_2 k_1' k_2' \\ \sigma_1 \sigma_2} } V^{k_1 k_2}_{k_1' k_2'} \bigg( - \underbrace{\left\langle c_{k_1 \sigma_1}^\dagger c_{k_2' \sigma_2} \right\rangle}_{\delta_{k_1 k_2'} \delta_{\sigma_1 \sigma_2} f_{k_1} } c_{k_2 \sigma_2}^\dagger c_{k_1' \sigma_1}  - \underbrace{\left\langle c_{k_2 \sigma_2}^\dagger c_{k_1' \sigma_1}  \right\rangle}_{\delta_{k_2 k_1'} \delta_{\sigma_1 \sigma_2} f_{k_2} } c_{k_1 \sigma_1}^\dagger c_{k_2' \sigma_2} + \underbrace{\left\langle c_{k_1 \sigma_1}^\dagger c_{k_1' \sigma_1} \right\rangle}_{\delta_{k_1 k_1'} f_{k_1} } c_{k_2 \sigma_2}^\dagger c_{k_2' \sigma_2}  \\
& + \underbrace{\left\langle c_{k_2 \sigma_2}^\dagger c_{k_2' \sigma_2} \right\rangle}_{\delta_{k_2 k_2'} f_{k_2} } c_{k_1 \sigma_1}^\dagger c_{k_1' \sigma_1} \bigg)\\
\end{split}
\end{equation}

In the language of Hartree Fock theory, the first two terms give the exchange term, and the last two terms the direct term. Apart from the $\frac{1}{2}$ factor, the term in (\ref{eq:mean_field}) becomes

\begin{equation}
\begin{split}
&- \sum_{\substack{k_1 k_2 \\ k_1' \sigma_1}} V_{k_1' k_1}^{k_1 k_2} f_{k_1} c_{k_2 \sigma_1}^\dagger c_{k_1' \sigma_1}  - \sum_{\substack{k_1 k_2 \\ k_2' \sigma_1}} V_{k_2 k_2'}^{k_1 k_2} f_{k_2} c_{k_1 \sigma_1}^\dagger c_{k_2' \sigma_1} + \sum_{\substack{k_1 k_2 k_2' \\ \sigma_1 \sigma_2}} V_{k_1 k_2'}^{k_1 k_2} f_{k_1} c_{k_2 \sigma_2}^\dagger c_{k_2' \sigma_2} \\
& + \sum_{\substack{k_1 k_2 k_1' \\  \sigma_1 \sigma_2}} V_{k_1' k_2'}^{k_1 k_2} f_{k_2} c_{k_1 \sigma_1}^\dagger c_{k_1' \sigma_1} \\
&= \sum_{k_1 k_2 \sigma_1} \bigg( 4 V_{k_1 k_2}^{k_1 k_2} - 2  V_{k_2 k_1}^{k_1 k_2}  \bigg) f_{k_2} c_{k_1 \sigma_1}^\dagger c_{k_1 \sigma_1}
,
\end{split}
\end{equation}
where we used momentum conservation to eliminate a $k'$-sum. Moreover, we used that the sum on spin ($\pm 1/2$) on the last two terms gives factors of 2 , since the interaction is spin independent and thus no spin-dependent term remains after we use momentum conservation. Making $k_1 \rightarrow k , \, k_2 \rightarrow k', \, \sigma_1 \rightarrow \sigma$, and recalling the definition in equation (\ref{eq:integrals}), we obtain the result we sought.

The procedure above is meant to serve as an intuitive derivation. Now we approach the problem more formally. In fact, the argument that allowed us to perform the commutation leading to equation \ref{eq:mf} seems somewhat handwaving. We should not have to take the thermodynamic limit to perform a mean field expansion. A more systematic procedure to obtain the mean field expansion of a quartic interaction term was given by Pierre de Gennes in the context of a mean field treatment of a superconductor in a magnetic field \cite{gennes_superconductivity_1999}. Our case is actually much simpler to analyze, but we follow the same argument as de Gennes.

Consider the Hamiltonian to be given by $\mathcal{H} = \mathcal{H}_0 + \mathcal{H}_1$, where

\begin{equation}
\begin{split}
\mathcal{H}_0 &= \sum_{\bm k, \sigma} \varepsilon_{\bm k} c_{\bm k, \sigma}^\dagger c_{\bm k, \sigma} \\
\mathcal{H}_1 &= \frac{1}{2} \sum_{\substack{\bm k_1 \bm k_2 \\ \bm k_1' \bm k_2' \\  \sigma_1 \sigma_2}} V_{k_1' k_2'}^{k_1 k_2} c_{\bm k_1 \sigma_1}^\dagger c_{\bm k_2 \sigma_2}^\dagger c_{\bm k_2' \sigma_2} c_{\bm k_1' \sigma_1} 
\end{split}
\end{equation}

We would like to find an effective Hamiltonian that is quadratic in the fermion operators:

\begin{equation}
\mathcal{H}_{\text{eff}} = \sum_{\bm k, \sigma} (\varepsilon_{\bm k} + v_{\bm k} ) c_{\bm k, \sigma}^\dagger c_{\bm k, \sigma}
\end{equation}

This effective Hamiltonian is diagonal, so assuming we know $v_{\bm k}$ (which is what we are trying to determine in the first place), we can compute its eigenstates $\{ \left| \phi \right\rangle \}$, and compute the average of the actual Hamiltonian $\mathcal{H}$ using the basis $\{ \left| \phi \right\rangle \}$:

\begin{equation}\label{eq:avH}
\left\langle \mathcal{H} \right\rangle = \frac{\sum_\phi \left\langle \phi | \mathcal{H} | \phi \right\rangle e^{-\beta E_\phi} }{\sum_\phi e^{-\beta E_\phi} }
\end{equation}

Our criterion to determine $\mathcal{H}_{\text{eff}}$ is the requirement that the free energy $F = \left\langle \mathcal{H} \right\rangle - TS$, with the average computed with the eigenstates of $\mathcal{H}_{\text{eff}}$ be stationary, i.e. $\delta F = 0$. Thus, we find the mean field form of the quartic term invoking only a variational principle without any need to resort to the thermodynamic limit. In fact, we never even have to explicitly compute the average in equation (\ref{eq:avH}). In terms of pairs of fermion operator averages, we have 

\begin{equation}
\left\langle \mathcal{H} \right\rangle = \sum_{\bm k, \sigma} \varepsilon_{\bm k} \left\langle c_{\bm k, \sigma}^\dagger c_{\bm k, \sigma} \right\rangle + \frac{1}{2} \sum_{\substack{\bm k_1 \bm k_2 \\ \bm k_1' \bm k_2' \\  \sigma_1 \sigma_2}} V_{k_1' k_2'}^{k_1 k_2} \left\langle c_{\bm k_1 \sigma_1}^\dagger c_{\bm k_2 \sigma_2}^\dagger c_{\bm k_2' \sigma_2} c_{\bm k_1' \sigma_1} \right\rangle ,
\end{equation}
where the last term can be reduced to products of averages of pairs of fermion operators by Wick's theorem:

\begin{equation}
\begin{split}
&\left\langle c_{\bm k_1 \sigma_1}^\dagger c_{\bm k_2 \sigma_2}^\dagger c_{\bm k_2' \sigma_2} c_{\bm k_1' \sigma_1} \right\rangle = \left\langle c_{\bm k_1 \sigma_1}^\dagger c_{\bm k_1' \sigma_1} \right\rangle  \left\langle c_{\bm k_2 \sigma_2}^\dagger c_{\bm k_2' \sigma_2} \right\rangle - \left\langle c_{\bm k_1 \sigma_1}^\dagger c_{\bm k_2' \sigma_2} \right\rangle  \left\langle c_{\bm k_2 \sigma_2}^\dagger c_{\bm k_1' \sigma_1} \right\rangle \\
& + \left\langle c_{\bm k_1 \sigma_1}^\dagger c_{\bm k_2 \sigma_2}^\dagger \right\rangle \left\langle c_{\bm k_2' \sigma_2} c_{\bm k_1' \sigma_1} \right\rangle
\end{split}
\end{equation}

The computation is now done by using the rules (for all $\bm k$ and $\sigma$).

\begin{equation}\label{eq:rules}
\begin{split}
\left\langle c_{\bm k, \sigma}^\dagger c_{\bm k', \sigma'} \right\rangle &= \delta_{\bm k, \bm k'} \delta_{\sigma, \sigma'} f_{\bm k} \\
\left\langle c_{\bm k, \sigma}^{(\dagger)} c_{\bm k', \sigma'}^{(\dagger)} \right\rangle &= 0 ,
\end{split}
\end{equation}
where $f_{\bm k} = (e^{\beta(\varepsilon_{\bm k} - \mu)} +1 )^{-1}$ is the Fermi-Dirac function.

Since the original Hamiltonian is quadratic, again we have that terms of the type $\left\langle cc \right\rangle$ and $\left\langle c^\dagger c^\dagger \right\rangle$ do not contribute. Hence, varying the free energy, we obtain

\begin{equation}
\begin{split}
&\delta F =  \delta \left\langle \mathcal{H} \right\rangle - T \delta S = \sum_{\bm k \sigma} \varepsilon_{\bm k} \delta \left\langle c_{\bm k, \sigma}^\dagger c_{\bm k, \sigma} \right\rangle + \frac{1}{2} \sum_{\substack{\bm k_1 \bm k_2 \\ \bm k_1' \bm k_2' \\  \sigma_1 \sigma_2}} V_{k_1' k_2'}^{k_1 k_2} \bigg( \left\langle c_{\bm k_1 \sigma_1}^\dagger c_{\bm k_1' \sigma_1} \right\rangle \delta  \left\langle c_{\bm k_2 \sigma_2}^\dagger c_{\bm k_2' \sigma_2} \right\rangle + \\
&\delta \left\langle c_{\bm k_1 \sigma_1}^\dagger c_{\bm k_1' \sigma_1} \right\rangle  \left\langle c_{\bm k_2 \sigma_2}^\dagger c_{\bm k_2' \sigma_2} \right\rangle  - \left\langle c_{\bm k_1 \sigma_1}^\dagger c_{\bm k_2' \sigma_2} \right\rangle  \delta \left\langle c_{\bm k_2 \sigma_2}^\dagger c_{\bm k_1' \sigma_1} \right\rangle - \delta \left\langle c_{\bm k_1 \sigma_1}^\dagger c_{\bm k_2' \sigma_2} \right\rangle  \left\langle c_{\bm k_2 \sigma_2}^\dagger c_{\bm k_1' \sigma_1} \right\rangle \bigg) - T \delta S ,
\end{split}
\end{equation}
which can be simplified exactly in the same manner as in equation (\ref{eq:mean_field}), i.e. by using the rules of equation (\ref{eq:rules}), and that the occupation of a given momentum state $\bm k$ is given by the Fermi-Dirac function:

\begin{equation}
\delta F = \sum_{\bm k \sigma} \varepsilon_{\bm k} \delta \left\langle c_{\bm k, \sigma}^\dagger c_{\bm k, \sigma} \right\rangle + \sum_{\bm k \bm k' \sigma} \bigg( 2 V_{\bm k \bm k'}^{\bm k \bm k'} -  V_{\bm k' \bm k}^{\bm k \bm k'}  \bigg) f_{\bm k'} c_{\bm k \sigma} \delta \left\langle c_{\bm k, \sigma}^\dagger c_{\bm k, \sigma} \right\rangle
\end{equation}

We can now compare $\delta F = \delta \left\langle \mathcal{H} \right\rangle - T \delta S$ and $\delta F' = \delta \left\langle \mathcal{H}_{\text{eff}} \right\rangle - T \delta S$, which is simply given by

\begin{equation}
\delta F' =  \delta \left\langle \mathcal{H}_{\text{eff}} \right\rangle - T \delta S = \sum_{\bm k \sigma} (\varepsilon_{\bm k} + v_{\bm k}) \delta \left\langle c_{\bm k, \sigma}^\dagger c_{\bm k, \sigma} \right\rangle
\end{equation}

Requiring both free energies to be stationary, we find our desired result

\begin{equation}
v_{\bm k} = \sum_{\bm k'} \bigg( 2 V_{\bm k \bm k'}^{\bm k \bm k'} -  V_{\bm k' \bm k}^{\bm k \bm k'}  \bigg) f_{\bm k'} ,
\end{equation}
which agrees with the result obtained from our initial more intuitive, but somewhat less rigorous  argument.