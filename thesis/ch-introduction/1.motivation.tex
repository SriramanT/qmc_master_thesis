\section{Motivation}
\label{sec:int_motivation}

Condensed matter physics is concerned with the emergence of the properties of quantum materials from complexity.
The central concept within this approach is that of symmetry breaking.
When a system condenses into a phase, there is a decrease in symmetry.
A simple pictorial example is the transition from a gas to a solid.
Statistically, any point within a gas is equivalent, that is, on average, the surroundings of all points look similar.
Formally, the system is then said to be fully translationally invariant.
On the other hand, in a solid, a point is only equivalent to a discrete set of other points, which can be reached starting from the original point by translations by lattice vectors.
Thus, the system becomes invariant only under a discrete set of translations, rather than a continuous one as in the case of the gas. 

The framework that is used to identify symmetry breaking is the Landau-Ginzburg (LG) theory of phase transitions.
The theory gives a prescription to discover phase transitions.
In other words, it gives criteria for a symmetry to become manifest.
In fact, as it turns out, the search for order relies on symmetry ideas well beyond condensed matter.
Symmetry breaking gives rise to emergent phenomena.
The idea of emergence rests on a constructionist, rather than a reductionist hypothesis: the explanation for the behavior of the many does not trivially follow from a deep understanding the behavior of the few.
As P.W. Anderson puts it, \say{The ability to reduce everything to simple fundamental laws does not imply the ability to start from those laws and reconstruct the universe.}

The broad scope of condensed matter comes from the sheer number of possibilities that the Landau-Ginzburg approach affords.
One can study the emergence of magnetism, superconductivity, or superfluidity, just to name a few.
The Landau-Ginzburg procedure can be summarized as follows: identify an order parameter reflecting the underlying symmetry of the system, and minimize a free energy function to deduce conditions for the symmetry to become manifest, leading to a phase transition.
The drawback of this approach is that it might be difficult to identify an order parameter in the first place.
Moreover, even if we do manage to find one, the usual procedure may be impossible to perform.
It can easily happen that the degree of complexity of the order parameter is simply too high.
Additionally, not all phase transitions can be described by the LG paradigm.

On the one hand, there are systems where a different kind of order arises.
A prominent example is that of fractional quantum Hall effect, where (rather surprisingly!) the \emph{quasi-particles} describing the excitations of the quantum Hall fluid carry \emph{fractions} of the electron charge.
There is an intimate connection between charge fractionalization and topology, which may be understood in terms of the properties of the Laughlin states describing the quantum Hall fluid. However, while it is tempting to try to characterize the latter in terms of the LG paradigm, it must actually be regarded as a distinct type of matter, where \say{topological order} arises.
The proposal put forward by Wen rest on characterizing quantum states by their ground state degeneracy, and investigating how they change under operations defined on specific manifolds.

For strongly correlated systems, there are phenomena which emerge specifically due to the interacting nature of the problem.