\section{Introduction to \acl{QMC}}
\label{sec:introQMC}

Solving the many-body problem remains one of the greatest challenges in physics.
Following the wealth of attempts at such pursuit, certain phenomena arising due to the strong interactions in quantum systems are explained in different theoretical frameworks, namely superconductivity, the Mott metal-insulator transition, and fractional quantum Hall effect.
All of these breakthroughs represented revolutions in their respective fields with significant scientific and technological impact.

Only in very limited cases does an actual analytical solution exist for the  Schr\"odinger equation for a system of interacting particles.
One must resort to sophisticated approximation methods to obtain  information about the role played by the competing interactions under various conditions in the aforementioned cases.
It is then natural that numerical methods have become prominent as a tool for extracting useful information about this type of systems.
\ac{QMC} is amongst the most accurate and extensively studied ones.
The idea of all \ac{QMC} methods is to reduce the interacting problem to solving a set of integrals, which can be evaluated numerically through a standard stochastic procedure.
These integrals are arrived at upon formulating the quantum many-body description of the system using the Schr\"odinger equation.
Hence the name \acl{QMC}, which is used to distinguish it from Classical Monte Carlo.
In the classical version, one measures thermal averages, while in the quantum version, one measures expectations of operators over the Hilbert space of the system, corresponding to physical observables that fluctuate with a dynamics given by the Schr\"odinger equation.

The dynamics of a quantum system are encoded in the Hamiltonian operator.
In the case of graphene-like \acl{2D} materials, one usually uses a tight-binding type of model.
It is found that the dynamics given by the tight-binding Hamiltonian is sufficient to describe most properties of graphene.
However, in other materials, such as \acp{TMD}, electron-electron interactions are stronger, and we must consider  Hubbard-type models.

In principle, the properties of a quantum many-fermion system can all be deduced by solving an extremely complicated Schr\"odinger equation that takes into account the coupling of all (identical) particles of the system.
However, for the majority of systems the resulting integrals have no analytic solution, so we solve the problem by numerical integration.
But there is a myriad of methods to evaluate integrals numerically.
How do we pick the best one for this case? 
Multi-dimensional integrals are plagued by the curse of dimensionality.
Although the Newton-Cotes quadrature formulas (including, for example the Newton method, and Simpson's rules), Gaussian quadrature formulas, or Romberg's method all scale polynomially with the number of integration points, they become impractical as the dimension increases.
To use them, one would invoke Fubini's theorem to reduce the multi-dimensional integral to a series of one-dimensional integrals.
However, the number of function evaluations required to compute the whole integral grows exponentially with its dimension.
The Monte Carlo method preserves the polynomial scaling, thus yielding comparable accuracy with far less function evaluations.
It is natural to use it since typically the state space of our quantum system is huge, leading to high dimensional integrals.

The Monte Carlo method is ubiquitous.
Its central idea is to use randomness to produce accurate estimates of deterministic integrals.
The term was coined by Nicolas Metropolis in 1949, first appearing in a seminal paper, in which it was described as a \say{statistical approach to the study of differential equations, or more generally, of integro-differential equations that occur in various branches of sciences}\cite{metropolis_monte_1949}.
Although it was used as early as 1777 in an experiment known as Buffon's needle - where one obtains an estimate of the constant $\pi$ by repeatedly throwing a needle randomly onto a sheet of paper with evenly spaced lines - it was crucially developed in the Los Alamos National Laboratory during World War II where the development of the first atomic bomb was completed, the primary objective of the Manhattan Project.
The method is particularly useful when one wants to sample from a probability distribution in an exponentially large state space, but it can, in principle, be used to solve any problem allowing a probabilistic formulation.

A variety of \ac{QMC} methods exists, using a sampling scheme based on the Metropolis algorithm, and variations thereof.
Variational and Diffusion \ac{QMC} are the simplest \ac{QMC} methods that allow one to capture the properties of correlated systems.
Although they already contain the main concepts used in this type of simulations, it is not always ideal or even possible to use them. 
We will discuss their flaws and show how further refinement leads to the auxiliary field method we ultimately used.

Using the Monte Carlo approach to study a many-fermion system implies overcoming a significant obstacle common to all \ac{QMC} methods - the so called \emph{fermion-sign problem}.
Pauli's exclusion principle implies that the many-fermion wave function is anti-symmetric, which leads to a sign oscillation that greatly impedes the accurate evaluation of averages of quantum observables.
The anti-symmetry constraint implies that a  straightforward weight interpretation of the wave function is not possible.
In the case of the finite temperature algorithm, the cancellations that occur when computing the average of any physical observable lead to poor statistical properties of the corresponding estimators.
This means that a massive amount of samples requiring enormous computer time are needed to obtain meaningful results.
In the case of the zero temperature algorithms, the situation is even worse.
It might not even be possible to design a stochastic process carrying the system to its ground state, as normally is done in \say{projective} methods\footnote{Methods that iteratively project a trial wave function onto the ground state.}: the wave function that is used as an initial proposal turns out to converge to a bosonic one, and the fermionic character of the system is lost.

As was proven by Troyer, the \emph{fermion-sign problem} has NP\footnote{NP or nondeterministic polynomial time, meaning that one can devise an algorithm that verifies the "yes" answer to a decision problem in polynomial time in the system size.
Note that the class $P$ - of polynomial time algorithms - is a subclass of NP.} computational complexity \cite{troyer_computational_2005}.
One of the greatest open questions in computer science is whether $P = NP$.
Solving the \emph{fermion-sign problem} would imply finding a solution to $P = NP$, which would constitute a major breakthrough.

\subsection{Variational Monte Carlo}

Variational techniques rely on an educated guess for the wave function of the system.
One introduces a set of variational parameters $\bm \alpha$ that are then tuned according to a variational principle.
Then, we may use the optimized trial wave function to compute physical quantities of interest using Monte Carlo.
The method is used to obtain zero temperature properties of a given model.
Note that it requires prior knowledge about the system to propose an approximate wave function in the first place.

A particularly relevant observable is the variational energy $E_V$ associated to a trial ground state.
Let $\bm r$ be the $3N$ spatial coordinates of the $N$ electrons.
For simplicity, let us ignore all other degrees of freedom, such as spin.
Given the Hamiltonian of the system $\mathcal{H}$, and a trial wave function $\psi (\bm r)$ - a guess of the wave function representing the ground state - one can compute the corresponding variational energy.

\begin{equation}\label{eq:variational_energy}
E_V = \frac{\left\langle \psi | \mathcal{H} | \psi \right \rangle}{\left\langle \psi | \psi \right \rangle} = \frac{ \int d\bm r |\psi (\bm r)|^2 E_L (\bm r)}{\int d\bm r | \psi (\bm r)|^2 } = \int d\bm r\rho (\bm r) E_L (\bm r) ,
\end{equation}
where

\begin{equation}\label{eq:local_energy}
E_L = \frac{\mathcal{H} \psi (\bm r) }{\psi (\bm r)}   \quad \text{and} \quad \rho (\bm r) = \frac{ | \psi (\bm r) |^2}{ \int d\bm r' | \psi (\bm r') |^2}
\end{equation}

Note that we managed to recast the variational energy as an average of the \emph{local} energy, $\left\langle E_L \right\rangle $, over the the distribution $\rho$.
This may be computed using the Monte Carlo method by sampling $M$ points $\bm r_k$ from the distribution $\rho (\bm r)$.
Denoting the sample mean of the random variable $X$ as $\overline {X}$:

\begin{equation}\label{eq:average}
E_V \approx \overline{E}_L = \frac{1}{M} \sum_{k= 1}^{M} E_L (\bm r_k) ,
\end{equation}

Let the ground state energy be $E_0$.
Then, states are optimized according to the variational principle:

\begin{equation}
E_V(\bm \alpha) = \frac{\left\langle \psi_{\bm \alpha} | \mathcal{H} | \psi_{\bm \alpha} \right\rangle}{\left\langle\psi_{\bm \alpha} | \psi_{\bm \alpha} \right\rangle} \ge E_0,
\end{equation}
where $\psi_{\bm \alpha}$ is the trial ground state wave function for the set of variational parameters ${\bm \alpha}$.

By varying $\bm \alpha$ we aim to obtain a variational energy that is as close as possible to the true ground state energy.
Since $E_V(\bm \alpha)$ is bounded from below, this is equivalent to minimizing it in the hope that $E_V(\bm \alpha_{min}) \gtrsim E_0$, i.e. the bound is tight.
The finite sampling size $M$, of course, introduces a statistical error common to all Monte Carlo methods. 
However, the use of an approximate wave function introduces a systematic error that is hard to control since trial wave functions are generally introduced based on approximate, or heuristic arguments.

\subsection{Diffusion Monte Carlo and projective methods}\label{subsec:dmc}

Variational Monte Carlo is severely limited by the use of a trial wave function $\psi_{\bm \alpha} (\bm r)$ because we may even not have enough information to even construct a reliable variational wave function in the first place.
Diffusion \ac{QMC} allows the simulation of a many-body system while having only a limited knowledge of the system's physical properties.
While it is exact for many-boson systems, it is only approximate for many-fermion systems.
The idea is to map the Schr\"odinger equation into  an imaginary-time diffusion equation.
Excited states are then filtered out by a diffusion process as we advance in imaginary-time.
In imaginary-time $\tau = i t$, the solution to the Schr\"odinger equation in terms of a formal series expansion in the eigenfunctions of the Hamiltonian becomes a series of \say{transient} wavefunctions weighted by $e^{-E_n \tau}, \, n \in \mathbb{N}$.
Within precision and accuracy constraints, the longest lasting of these is the ground state \cite{kosztin_introduction_1996}.
Thus, the idea of the diffusion method is to generate samples using the exact ground state wave function $\psi_0 (\bm r)$ \cite{toulouse_chapter_2016}.
The associated exact energy $E_0$ is the matrix element of the hamiltonian calculated using a trial wave function and the ground state wave function.

\begin{equation}
E_0 = \frac{ \left\langle \psi_0 |E_0 \mathbbm{1} | \psi \right\rangle}{\left\langle \psi_0 | \psi \right\rangle} = \frac{\left\langle \psi_0 | \mathcal{H} | \psi \right\rangle}{ \left\langle\psi_0 | \psi \right\rangle} = \frac{\int d\bm r \psi_0^\star (\bm r) \psi (\bm r) E_L (\bm r)}{\int d\bm r\psi_0^\star (\bm r) \psi (\bm r)}
\end{equation}

Note that using this trick we avoid the computation of $\mathcal{H} \psi_0 = E_0 \psi_0$, that is, the ground state energy.
Instead, we approximate the integral by considering $M$ configuration samples $\bm r_{k = 1,..., M}$ in a similar spirit to that of Variational \ac{QMC}.
Notice that the integral consists of a local energy of the trial wave function $E_L (\bm r) = \frac{\mathcal{H} \psi (\bm r)}{\psi (\bm r)}$ averaged over a mixed distribution from which we draw a sample:

\begin{equation}
f(\bm r) = \frac{\psi_0^\star (\bm r) \psi (\bm r) }{ \int d\bm r  \psi_0 (\bm r) \psi (\bm r)}
\end{equation}

Although the method is, of course, aimed at probing many-body systems, let us consider a single particle in \acs{1D}, for simplicity, to illustrate the method.
Performing a Wick rotation - effectively going to imaginary time - and shifting the energy, the Schr\"odinger equation becomes (with $\hbar = 1$)

\begin{equation}
\frac{\partial \psi ( x, \tau )}{\partial\tau}  = -\frac{1}{2m}\frac{\partial^2 \psi ( x, \tau )}{\partial x^2} - \bigg[ V(x) - E_T \bigg] \psi( x, \tau ) 
\end{equation}

The exact ground state wave function $\psi_0 ( x )$ is obtained as the longest lasting transient state in imaginary time: we are interested in the asymptotic behavior of the series expansion constituting the formal solution of the Schr\"odinger equation

\begin{equation}
\psi (x, \tau) = \sum_{n=0}^{\infty} c_n \psi_n (x) e^{-(E_n - E_T)\tau}
\end{equation}

Imaginary time evolution is governed by

\begin{equation}\label{eq:im_ev}
\left| \psi (t) \right\rangle = \lim_{\tau \rightarrow \infty} \sum_n e^{-(E_n - E_T) \tau} \left|\psi_n \right\rangle \left\langle \psi_n | \psi \right\rangle = \lim_{\tau \rightarrow \infty} e^{-(E_0 - E_T)\tau} \left| \psi_0 \right\rangle \left\langle \psi_0 | \psi \right\rangle 
\end{equation}


If $E_T > E_0$ the wave function diverges exponentially fast: $\lim_{\tau \rightarrow \infty} \psi ( x, \tau) = \infty$.
Similarly, for $E_T < E_0$ it vanishes exponentially fast: $\lim_{\tau \rightarrow \infty} \psi ( x, \tau) = 0$.
However, if $E_T = E_0$ the wave function converges to the ground state one up to a constant factor, $c_0 = \left\langle \psi_0 | \psi \right\rangle$.

\begin{equation}\label{eq:dmc}
\lim_{\tau \rightarrow \infty} \psi ( x, \tau) = c_0 \psi_0 (x) \quad \text{or} \quad \lim_{\tau \rightarrow \infty} \left|\psi (\tau) \right\rangle \propto \left| \psi_0 \right\rangle
\end{equation}

Diffusion \ac{QMC} makes use of Eq. (\ref{eq:dmc}), approximating $\psi_0(x)$ by $\psi (x, \tau)$ for sufficiently long time.
The only requirement is that $\psi (x, \tau)$ and $\psi_0(x)$ overlap significantly so that $c_0$ is large enough to be numerically measurable, and we can always center a positive trial wave function in a region where $\psi_0(x)$ is large enough and positive.
If the latter condition does not hold, the wave function converges to a bosonic, instead of a fermionic one.
Of course, these conditions can always be met for a single particle, but note that they might fail for a many-fermion system, for which the wave function crosses a number of nodes due to its anti-symmetric nature.

\subsection{Auxiliary Field \acs{QMC} and the Fermion Sign Problem}
\label{subsec:introAFQMC}

As we have seen, the major drawback of the variational method was that it demanded \emph{a priori} knowledge of a reasonable variational wave function describing, at least partly, some of the physics of the problem.
Diffusion \acs{QMC} demands less: we need only propose a trial wave function that overlaps with the ground state.
However, none of these methods allow us to probe systems at finite temperature.
Moreover, they both require some prior knowledge about the system, which may not always be available.

An alternative method is based on introducing an additional lattice bosonic field that mediates the electron-electron interaction.
The interacting problem then becomes a problem of independent fermions coupled to an external field, and the fermionic part of the partition function can be traced out explicitly, leaving the contribution of a \emph{discrete}\footnote{Although, there is a finite number of field configurations, the number grows exponentially with the number of sites on the lattice.} bosonic field.
This contribution can be evaluated numerically by employing importance sampling over the field configurations.
Auxiliary field \acs{QMC} relies on a mapping to a \say{classical} system (in quotes because there is no actual classical analogue):

\begin{equation}
Z = \Tr [ e^{-\beta \mathcal{H} } ] = \sum_c p_c ,
\end{equation}
but some of the \say{probabilities} can actually be negative $p_c < 0$.
This occurs due to the antisymmetry of the many-electron wavefunction under electron exchange, and is at the root of the sign problem.

The negative weight problem may easily be circumvented when computing averages of observables:

\begin{equation}\label{eq:signSampling}
\left\langle A \right\rangle = \frac{\sum_c A ( c ) p ( c )}{\sum_c p ( c ) } = \frac{\sum_c A ( c )|  p ( c ) | \text{sign}[p(c)] / \sum_c | p ( c ) | }{\sum_c  |  p ( c ) | \text{sign}[p(c)] /  \sum_c | p ( c ) |} \equiv \frac{\left\langle A s \right\rangle_{|p|}}{\left\langle s \right\rangle_{|p|}} ,
\end{equation}
where $s(c) = \text{sign} [ p ( c ) ]$, and $| p ( c ) | $ corresponds to an auxiliary bosonic system (also coupled to the bosonic field) corresponding to the original fermionic system, and for which there is no sign problem.

The relative error $\Delta s / \left\langle s \right\rangle$ increases exponentially with the number of particles, with inverse temperature, and possibly with other parameters of the specific model to be studied \cite{troyer_computational_2005, hou_numerical_2009}.
To see this, we start by noting that the average sign is the ratio between the partition functions of the fermionic ($Z = \sum_c p(c)$) and bosonic systems ($Z = \sum_c | p ( c ) |$).
In terms of the difference in free energy densities, $\left\langle s \right\rangle = Z / Z' = e^{-\beta N_p \Delta f}$, implying that for $M$ samples, the error of the denominator of Eq. (\ref{eq:signSampling}) becomes

\begin{equation}
\frac{\Delta s}{\left\langle s \right\rangle} = \frac{\sqrt{(\left\langle s^2 \right\rangle - \left\langle s \right\rangle^2 )/ M }}{\left\langle s \right\rangle} = \frac{ \sqrt{ 1 - \left\langle s \right\rangle^2}  }{\sqrt{M} \left\langle s \right\rangle} \propto \frac{e^{\beta N_p \Delta f}}{\sqrt{M}} ,
\end{equation}
and similarly for the numerator of Eq. (\ref{eq:signSampling}).

Auxiliary field \acs{QMC} can also be formulated to probe ground state properties, and a sign problem arises similarly.
In fact, this problem plagues all \acs{QMC} methods, even though we showed it only for auxiliary field \acs{QMC}.
The latter is the most robust, unbiased, and reliable method, with a generally modest sign problem, hence we choose it to carry out our simulations.
Note that, in particular, it is certainly more powerful than the variational and diffusion methods outlined before since it requires much less \emph{a priori} information about the system.
Perhaps more importantly, recent findings suggest that it can be used in conjunction with neural networks to discover quantum phase transitions in correlated systems  \cite{broecker_machine_2017} in what could be a revolution in the field.