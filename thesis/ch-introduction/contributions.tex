\section{Original Contributions}
\label{sec:int_contributions}

The major contribution of this work has been the study of the properties of a \ac{TMD} nanoribbon, in particular providing insight into its magnetic properties.
The results were compared with those obtained in the mean field approximation and benchmarked against existing, more efficient implementations (namely \texttt{ALF} \cite{bercx_alf_2017} and \texttt{QUEST} \cite{noauthor_quest_2012}), and early seminal studies \cite{hirsch_discrete_1983,white_numerical_1989}.

To carry out this study, we implemented a general auxiliary \ac{QMC} code in \texttt{C++}, which can be used to simulate low-dimensional Hubbard-like models with different geometries.
Additionaly, using our code, we characterized and compared different options to stabilize the matrix products needed to perform the simulations.
Lastly, we gave a contribution to circumvent the fermion sign problem in an attempt to extract the maximum amount of information out of the Monte Carlo measurements.