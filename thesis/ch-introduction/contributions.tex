\section{Original Contributions}
\label{sec:int_contributions}

In this work, we focus on the study of emerging magnetic ordering in \ac{TMD} nanoribbons due to electron-electron interactions.
To carry out this study, we use our own original implementation of the auxiliary field \ac{QMC} algorithm in \texttt{C++}.
To validate the built software, we consider previously well known models (such as the Hubbard model on the \acs{1D} chain, and the square lattice), and benchmark our results by comparing them with those of existing implementations (namely \texttt{ALF} \cite{bercx_alf_2017} and \texttt{QUEST} \cite{noauthor_quest_2012}) and early seminal studies \cite{hirsch_discrete_1983,white_numerical_1989}.
We obtain original mean field results, which we use as a guide to approach the problem via \acs{QMC}.
The code we wrote can be used to simulate low-dimensional Hubbard-like models with different geometries, thus extending this work further.
Additionally, using our code, we compare different options to stabilize the matrix products needed to perform the simulations, and characterize the fermion sign problem.

To study \ac{TMD} nanoribbons, we extend a minimal three-band tight binding model \cite{liu_three-band_2013} of \acp{TMD} to the interacting case.
Then, we find emerging edge-state magnetism in \ac{TMD} nanoribbons via original mean field calculations, and we compare those with \acs{QMC} results.