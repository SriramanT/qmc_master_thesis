\section{Original Contributions}
\label{sec:int_contributions}

The major contribution of this work has been the development of a general auxiliary \ac{QMC} code in \texttt{C++}, which can be used to simulate low-dimensional Hubbard-like models with different geometries.

The code was then applied to study the properties of a \ac{TMD} nanoribbon, in particular providing insight into its magnetic properties.
The results were compared with those obtained in the mean field approximation.

Additionaly, we characterized and compared different options to stabilize the matrix products needed to perform the simulations.

Lastly, we gave a contribution to circumvent the fermion sign problem in an attempt to extract the maximum amount of information out of the Monte Carlo measurements.