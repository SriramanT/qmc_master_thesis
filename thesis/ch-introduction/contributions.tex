\section{Original Contributions}
\label{sec:int_contributions}

In this work we focus mainly on the study of the magnetic properties of \ac{TMD} nanoribbons.
We compare our \ac{QMC} results with those obtained in the mean field approximation and benchmark them  against existing, \say{tried and true}  implementations (namely \texttt{ALF} \cite{bercx_alf_2017} and \texttt{QUEST} \cite{noauthor_quest_2012}), and early seminal studies \cite{hirsch_discrete_1983,white_numerical_1989}.

To carry out this study, we use \texttt{QUEST} and our own original implementation of the auxiliary field \ac{QMC} algorithm in \texttt{C++}.
The code we wrote can be used to simulate low-dimensional Hubbard-like models with different geometries to extend this work.
Additionaly, using our code, we characterize and compare different options to stabilize the matrix products needed to perform the simulations.
Lastly, we give a contribution to circumvent the fermion sign problem in an attempt to extract the maximum amount of information out of the Monte Carlo measurements.