\section{Strongly correlated electron systems}
\label{sec:strongly_correlated}

Condensed matter physics is concerned with the emergence of the properties of quantum materials from complexity.
The central concept within this approach is that of symmetry breaking.
When a phase transition occurs, a system is said to condense into a phase of lower symmetry.
A simple pictorial example is the transition from a gas to a solid.
%Statistically, any point within a gas is equivalent, that is, on average, the surroundings of every point look similar.
%Formally, the system is said to be invariant under any translation.
%On the other hand, in a crystal, a point is only equivalent to a discrete set of other points.
%In fact, a simplified view of a crystal consists of a periodic arrangement of atoms occupying the points of a Bravais lattice.
%Any point on the lattice can be reached starting from any other point upon translation by a lattice vector.
%Thus, a system which makes a transition from the gaseous to the solid state becomes invariant only under a discrete subgroup of all translations. 

A framework that is commonly used to identify symmetry breaking is the \ac{LG} theory of phase transitions.
This theory gives a prescription to discover phase transitions.
More precisely, it gives criteria for a symmetry to become manifest.
Symmetry breaking gives rise to emergent phenomena.
The idea of emergence rests on a constructionist, rather than a reductionist hypothesis: that the behavior of the many does not trivially follow from the behavior of the few.
As P.W. Anderson puts it, \say{The ability to reduce everything to simple fundamental laws does not imply the ability to start from those laws and reconstruct the universe.} \cite{anderson_more_1972}

The broad scope of condensed matter comes from the sheer number of possibilities that the symmetry breaking approach affords.
For the specific case of the \acs{LG} theory, one can study the emergence of magnetism, superconductivity, or superfluidity, just to name a few.
However, as we shall see, sometimes the \acs{LG} theory fails to capture a system's behavior, and we must resort to other theories to identify these, or other eventual properties that might arise.
The \acl{LG} procedure can be summarized as follows: identify an order parameter reflecting the underlying symmetry of the system, and minimize the free energy in order to deduce conditions for the symmetry to become manifest, leading to a phase transition.
The drawback of this \emph{variational} approach is that it might be difficult to identify an order parameter in the first place.
Moreover, even if we do manage to find one, the usual procedure may be impossible to perform.
It can easily happen that the degree of complexity of the order parameter is simply too high.
Additionally, and perhaps more importantly, \emph{not all} phase transitions can be described by the LG paradigm.

On the one hand, there are systems where a different kind of order arises.
A prominent example is that of fractional quantum Hall effect, where (rather surprisingly!) the \emph{quasi-particles} describing the excitations of the quantum Hall fluid carry \emph{fractions} of the electron charge.
There is an intimate connection between charge fractionalization and topology, which may be understood in terms of the properties of the Laughlin states describing the quantum Hall fluid. However, while it is tempting to try to characterize the latter in terms of the \acs{LG} paradigm, it must actually be regarded as a distinct type of matter, where \say{topological order} arises \cite{wen_topological_1990}.
%The proposal put forward by Wen \cite{wen_topological_1990} rests on characterizing quantum states by their ground state degeneracy, and investigating how they change under operations defined on specific manifolds. 

On the other hand, for the so called \emph{strongly correlated} systems we shall focus on in this work, there are phenomena which emerge specifically due to the interacting nature of the problem.
They are elusive because a description within the \acs{LG} paradigm does not yield a behavior consistent with what is observed.
%Instead, order emerges from the complexity created by the interactions among all the constituents.
The \acs{LG} theory fails because it ignores these interactions by disregarding fluctuations in the microscopic configuration of the system.
This approximation consists of reducing the complex interactions to an effective \emph{mean field}, which is normally determined self-consistently.
Strongly correlated systems require an approach beyond mean field, which makes them both extremely interesting and notoriously difficult to treat.
In fact, the failure of mean field theory is not limited to correlated systems, and its success in describing a given system depends, for example, on the dimensionality\footnote{Normally, there is an upper critical dimension $d_c$ above which mean field is exact. Below $d_c$, its predictions might be useful qualitatively, but not quantitatively.} and on the range of the particular type of interaction that is considered.

In many cases, mean field theory is too extreme an approximation.
Nonetheless, its occasional failure at capturing the whole of a system's properties does not deem it  useless.
Actually, it is quite the contrary.
Mean field is often used as a first approach to build an intuitive physical picture for the general properties of the system, while keeping in mind that the description it provides is intrinsically insufficient.
In this work, we will use mean field theory to gain intuition about the problem at hand.
%Clearly, to extract the features of a correlated system we must extend it to the fully interacting case.
%Strongly correlated quantum matter is ubiquitous and is at the heart of today's most advanced electronic materials, namely organic conductors, high $T_c$ (cuprate) superconductors, colossal  magnetoresistance materials, and \say{heavy-fermion}\footnote{The quasi-particles describing excitations in these materials behave like much heavier electrons, hence the name.} compounds. 
%Actually, the problem of strong correlations has now expanded beyond condensed matter physics. Quark-gluon plasmas, believed to have been formed just a few microseconds after the Big Bang, also belong to this class of systems.
%Another example comes from atomic physics: ultracold atoms in optical lattices behave in a very similar way to correlated electrons.
%In fact, the behavior is so similar that these systems are being used as \emph{de facto} quantum simulators of correlated electron systems \cite{quintanilla_strong-correlations_2009}.

A central piece in the understanding of correlated matter is the Hubbard model, a model which we will use extensively in this thesis.
It was introduced to bridge a gap between metals and magnetic insulators, building on the earlier work of Mott.
The model is extremely simple.
Electrons hop from atom to atom on a lattice, paying an energy penalty when they occupy the same site.
This repulsive effect results in correlations beyond those that are always present due to the fermionic nature of the particles obeying the Pauli exclusion principle.
In the limit of weak repulsion, the electrons are nearly free, and the system behaves like a metal.
Otherwise, the electrons become localized at fixed atomic positions resulting in magnetic insulating behavior.
The model is simple to formulate, but already includes highly nontrivial  correlation effects between all electrons in the solid.
Thus, it is not surprising that an exact solution exists only in \acs{1D} \cite{lieb_absence_1968}, and higher dimensional versions are still being studied more than 50 years after the model appeared \cite{hubbard_electron_1963}.