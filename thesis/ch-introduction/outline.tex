\section{Outline}
\label{sec:int_outline}

We started this introductory chapter with the concept of emergence in strongly correlated electron systems.
Then, we proceeded to discuss the particular example we study in this thesis: the \acs{2D} \acs{TMD} nanoribbon.
The strong electron correlations in this system suggest the existence of emergent edge-state magnetism, which to our knowledge was unexplored numerically before this work.
To tackle this interacting fermion system, we resort to a numerical iterative solution of the mean field self consistent equation and a state-of-the-art determinant \ac{QMC} algorithm.

In chapter \ref{cap:hubbard}, we introduce the Hubbard model, a ubiquitous model of electron correlations.
We discuss analytical solutions for simple limiting cases, outline some approximation methods, and introduce Green's functions, which turn out to be the main object of our simulations.
Moreover, we formulate the mean field theory of the Hubbard model.

In chapter \ref{cap:afqmc}, we start by summarizing the main ideas about how to apply the Monte Carlo method to statistical physics problems.
In this context, we use original results of our simulations to illustrate the concepts in the specific context of our problem.
Still in chapter \ref{cap:afqmc}, we introduce the auxiliary field method, and its various challenges, namely low temperature, and large size stabilization.

In chapter \ref{cap:applications}, we apply the code we implemented for a variety of systems, benchmarking our code, and carrying out some original calculations both at the mean field level and using \acs{QMC} for \acp{TMD}.

Finally, in chapter \ref{cap:conclusions}, we conclude by discussing the results obtained in the previous chapter in the context of the literature, and propose future work to be done on the topic.