\section{Outline}
\label{sec:int_outline}

We started this introductory chapter with the concept of emergence in strongly correlated electron systems.
Then, we proceeded to discuss the particular example, which we will tackle in this thesis: the \acs{2D} \acs{TMD} nanoribbon.
In this system, electron correlations give rise to interesting forms of magnetism, which, as far as we know, were unexplored numerically before this work.
To tackle this interacting fermion system, we resort to a  state-of-the-art algorithm that has been in continuous development over the last 30 years.

In the chapter (), we will discuss the Hubbard mode, an ubiquitous model of electron correlations.
We discuss analytical solutions of simple limiting cases, outline some approximation methods, and introduce Green's functions, which will turn out to be the main object of our simulations.
Moreover, we formulate the mean field theory of the Hubbard model.

Then, we proceed to the simulation method.
In chapter (), we summarize the main ideas about how to apply the Monte Carlo method to statistical physics problems.
In this context, we use original results of our simulations to illustrated the concepts in the specific context of our problem.

Chapter () introduces the auxiliary field method, and its various challenges, namely low temperature, and large size stabilization.

In the following chapter (), we apply the code we implemented for a variety of systems, benchmarking our code, and carrying out some original calculations.

Finally, in chapter (), we conclude by discussing the results obtained in the previous chapter, and propose future work to be done on the topic.