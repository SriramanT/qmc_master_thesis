
\fancychapter{Introduction}
\label{cap:int}

\slshape

The isolation of graphene in 2004 has led to a growing interest of the scientific community in \ac{2D} materials revealing extraordinary properties.
Among them, are \acp{TMD} nanostructures, which are examples of systems where electrons are strongly correlated, and one cannot neglect the interactions between them.
The increased complexity of the models describing such materials, compared to their graphene counterparts, calls for sophisticated computer simulation methods.
A numerical treatment is required to solve these problems since analytical approaches are either hopeless, or rely on unrealistic approximations.
In this introductory chapter, we start by  reviewing the literature on the physics of \acp{TMD}, focusing on their basic properties.
Then, we present a survey of simulation methods belonging to the \acl{QMC} class to introduce some basic concepts, and motivate the choice of the particular method we used.
Finally, we summarize our original contributions, and outline the structure of the thesis.

\normalfont

\section{Motivation}
\label{sec:motivation}

It might seem surprising that \ac{2D} systems were not considered as a real possibility before the discovery of graphene since they are often idealized in thought experiments, for example when investigating toy models of more complex higher dimensional systems.
In fact, while thin film deposition on comparably thicker substrates was commonplace long before 2004, \ac{2D} layers were thought not to exist independently from their 3D base.
Their existence was not expected \emph{a priori} because at first sight they seem to violate the Mermin-Wagner-Hohenberg theorem \cite{mermin_absence_1966, coleman_there_1973, hohenberg_existence_1967}, a no-go theorem that forbids ordering below three dimensions at finite temperature\footnote{On graphene sheets, ripples appear, which implies that the material is not strictly \ac{2D}, and thus can be stabilized. This issue is subtle, and is beyond the scope of this work.}.

The discovery of graphene paved the way for the search for similarly stable \ac{2D} materials and a plethora of these has been discovered since \cite{ajayan_two-dimensional_2016}.
A vast set of open problems remains to be solved within the realm of the fascinating and counterintuitive properties of the now huge variety of existing \ac{2D} systems.
In particular, in some of these, the effect of electron interactions is not negligible, leading to emergent phenomena.
These are collective effects which emerge as a result of the interactions between the individual components of a macroscopic system.
The properties of the system's components do not directly percolate up; instead, they shape the interactions that dictate the system's properties sometimes in rather unexpected ways, leading to unusual behavior.

Interacting electron systems are often tackled by carrying out computer simulations.
\ac{QMC} is a family of numerical methods that are  amply applicable to condensed matter physics problems, and that are particularly well suited to study strongly correlated electrons.
Despite the system size being constrained due to limited simulation time, reliable, accurate and unbiased solutions are provided to the otherwise intractable quantum many-body problem.
The class of \acs{QMC} algorithms that is used in this work was introduced in the 1980's in a series of seminal papers by Hirsch and \acl{BSS}\footnote{After whom the \ac{BSS} algorithm, on which we based the implementation used in this work, is named.} \cite{hirsch_discrete_1983, hirsch_monte_1982, blankenbecler_monte_1981, hirsch_two-dimensional_1985, hirsch_monte_1983, hirsch_stable_1988, hirsch_antiferromagnetism_1989}, but it saw a recent surge \cite{dumitrescu_superconductivity_2016, berg_monte_2018, beyl_revisiting_2018, chang_recent_2015, esterlis_breakdown_2018, mondaini_determinant_2012, meng_characterization_2014, kung_characterizing_2016, johnston_determinant_2013, rademaker_determinant_2013, ying_determinant_2014, scalettar_numerical_2007, zhou_quantum_2014} due to the increase in computational power, and algorithmic development.
Method optimization can prove crucial in applications to widely studied physical models of electron interactions.
In particular, recent computational and algorithmic developments opened the door to study both larger and lower temperature systems \cite{jiang_fast_2016, lee_parallelization_2010, chang_recent_2015, bai_stable_2011}.
In this work, an implementation of determinant \acs{QMC} based on the \ac{BSS} algorithm is used to simulate a \ac{TMD} zigzag-edged nanoribbon, a nanostructure made of this recent member of the \acs{2D} materials family.
Early mean field studies show that this type of nanostructures have a tendency towards magnetism in graphene \cite{yazyev_emergence_2010}, which makes them good candidates for use in nanospintronics.
Our mean field calculations for \acp{TMD} show a similar trend, motivating our subsequent \acs{QMC} study, in an attempt to test how realistic the mean field predictions are.
\acs{QMC} is a complementary, more accurate, and unbiased approach that can shed light upon not only magnetic, but also  other phenomena, like the formation of charge density waves and superconductivity in the context of generic interacting electron models. Hence \acs{QMC} has acquired a far-reaching importance as a flexible, and accurate numerical tool.
\section{Strongly correlated electron systems}
\label{sec:strongly_correlated}

Condensed matter physics is concerned with the emergence of the properties of quantum materials from complexity.
The central concept within this approach is that of symmetry breaking.
When a phase transition occurs, a system is said to condense into a phase of lower (or higher) symmetry.
A simple pictorial example is the transition from a gas to a solid.
Statistically, any point within a gas is equivalent, that is, on average, the surroundings of all points look similar.
Formally, the system is then said to be fully translationally invariant.
On the other hand, in a solid, a point is only equivalent to a discrete set of other points.
In fact, a simplified view of a solid consists of a periodic arrangement of atoms occupying the points of a lattice.
Any point on the lattice can be reached starting from any other point upon translation by a lattice vector.
Thus, a system that makes a transition from the gaseous to the solid state becomes invariant only under a discrete set of translations, rather than a continuous one. 

A framework that is commonly used to identify symmetry breaking is the \ac{LG} theory of phase transitions.
The theory gives a prescription to discover phase transitions.
More precisely, it gives criteria for a symmetry to become manifest.
Although this framework is very useful, it turns out that the search for order relies on symmetry ideas well beyond condensed matter.
Symmetry breaking gives rise to emergent phenomena.
The idea of emergence rests on a constructionist, rather than a reductionist hypothesis: that the behavior of the many does not trivially follow from the behavior of the few.
As P.W. Anderson puts it, \say{The ability to reduce everything to simple fundamental laws does not imply the ability to start from those laws and reconstruct the universe.} \cite{anderson_more_1972}

The broad scope of condensed matter comes from the sheer number of possibilities that the symmetry breaking approach affords.
For the specific case of the \acs{LG} theory, one can study the emergence of magnetism, superconductivity, or superfluidity, just to name a few.
However, as we shall see, sometimes the \acs{LG} theory fails to capture a system's behavior, and we must resort to other theories to identify these, or other eventual properties that might arise.
The \acl{LG} procedure can be summarized as follows: identify an order parameter reflecting the underlying symmetry of the system, and minimize the free energy in order to deduce conditions for the symmetry to become manifest, leading to a phase transition.
The drawback of this \emph{variational} approach is that it might be difficult to identify an order parameter in the first place.
Moreover, even if we do manage to find one, the usual procedure may be impossible to perform.
It can easily happen that the degree of complexity of the order parameter is simply too high.
Additionally, and perhaps more importantly, not all phase transitions can be described by the LG paradigm.

On the one hand, there are systems where a different kind of order arises.
A prominent example is that of fractional quantum Hall effect, where (rather surprisingly!) the \emph{quasi-particles} describing the excitations of the quantum Hall fluid carry \emph{fractions} of the electron charge.
There is an intimate connection between charge fractionalization and topology, which may be understood in terms of the properties of the Laughlin states describing the quantum Hall fluid. However, while it is tempting to try to characterize the latter in terms of the \acs{LG} paradigm, it must actually be regarded as a distinct type of matter, where \say{topological order} arises \cite{wen_topological_1990}.
%The proposal put forward by Wen \cite{wen_topological_1990} rests on characterizing quantum states by their ground state degeneracy, and investigating how they change under operations defined on specific manifolds. 

On the other hand, for the so called strongly correlated systems we shall focus on in this work, there are phenomena which emerge specifically due to the interacting nature of the problem.
They are elusive because a description in terms of the \acs{LG} paradigm does not yield a behavior consistent with what is observed empirically.
Instead, order emerges from the complexity created by the interactions among all the constituents.
The \acs{LG} theory fails because it ignores these interactions by disregarding fluctuations in the microscopic configuration of the system.
This approximation consists of reducing the complex interactions to an effective \emph{mean field}, which is normally determined self consistently.
Strongly correlated systems require an approach beyond mean field, which makes them both extremely interesting and notoriously difficult to tackle.
The mean field view fails to describe them because it considers each constituent to interact only with an external entity representing the interactions with all other constituents, underestimating collective behavior.
In fact, the failure of mean field theory is not limited to correlated systems, and its success in describing a given system depends, for example, on the dimensionality\footnote{Normally, there is an upper critical dimension $d_c$ above which mean field is exact. Below $d_c$, its predictions might be useful qualitatively, but not quantitatively.} and on the range of the particular type of interaction that is considered.

In many cases, mean field theory is too extreme an approximation.
Nonetheless, its occasional failure at capturing the whole of a system's properties does not deem it  useless.
Actually, it is quite the contrary.
Mean field is often used as a first approach to build an intuitive physical picture for the general properties and behavior of the system.
Of course, this is done while keeping in mind that the description it provides is intrinsically insufficient.
Clearly, to extract the features of a correlated system we must extend it to the fully interacting case.

Strongly correlated quantum matter is ubiquitous and is at the heart of today's most advanced electronic materials, namely organic conductors, high $T_c$ (cuprate) superconductors, colossal  magnetoresistance materials, and \say{heavy-fermion}\footnote{The quasi-particles describing excitations in these materials behave like much heavier electrons, hence the name.} compounds. 
Actually, the problem of strong correlations has now expanded beyond condensed matter physics. Quark-gluon plasmas, believed to have been formed just a few microseconds after the Big Bang, also belong to this class of systems.
Another example comes from atomic physics: ultracold atoms in optical lattices behave in a very similar way to correlated electrons.
In fact, the behavior is so similar that these systems are being used as \emph{de facto} quantum simulators of correlated electron systems \cite{quintanilla_strong-correlations_2009}.

A central piece in the understanding of correlated matter is the Hubbard model.
It was introduced to bridge a gap between metals and magnetic insulators, building on the earlier work of Mott.
The model is extremely simple.
Electrons hop from atom to atom on a lattice, paying an energy penalty when they occupy the same site.
This repulsive effect results in correlations beyond those that are always present due to the fermionic nature of the particles obeying the Pauli exclusion principle.
In the limit of weak repulsion, the electrons are nearly free, and the system behaves like a metal.
Otherwise, the electrons become localized at fixed atomic positions resulting in magnetic insulating behavior.
The model is simple to formulate, but already includes highly nontrivial  correlation effects between all electrons in the solid.
Thus, it is not surprising that an exact solution exists only in \acs{1D} \cite{lieb_absence_1968}, and higher dimensional versions are still being studied more than 50 years after the model appeared \cite{hubbard_electron_1963}.
\section{State of The Art}
\label{sec:int_state}

Solving the many-body problem remains one of the greatest challenges in physics.
Following the wealth of attempts at such pursuit, certain phenomena arising due to the strong interactions in quantum systems are explained in different theoretical frameworks, namely superconductivity, the Mott metal-insulator transition, and fractional quantum Hall effect.
All of these breakthroughs represented revolutions in their respective fields with significant scientific and technological impact.

Only in very limited cases does an actual analytical solution exist for the  Schr\"odinger equation for a system of interacting particles.
One must resort to sophisticated approximation methods to obtain  information about the role played by the competing interactions under various conditions in the aforementioned cases.
It is then natural that numerical methods have become prominent as a tool for extracting useful information about this type of systems.
\ac{QMC} is amongst the most accurate and extensively studied ones.

The idea of all \ac{QMC} methods is to reduce the interacting problem to solving a set of integrals, which can be evaluated numerically through a standard stochastic procedure.
These integrals are arrived at upon formulating the quantum many-body description of the system using the Schr\"odinger equation.
Hence the name \acl{QMC}, which is used to distinguish it from Classical Monte Carlo.
In the classical version, one measures thermal averages, while in the quantum version, one measures expectations of operators over the Hilbert space of the system, corresponding to physical observables that fluctuate with a dynamics given by the Schr\"odinger equation.

\subsection{Beyond graphene: TMD nanoribbons}

\ac{2D} materials have steadily been attracting more and more attention since graphene was experimentally isolated from a graphite sample by mechanical exfoliation, yielding a system constituted by a single layer of atoms (Figure \ref{fig:graphene}, left).
Since then, numerous studies have been made due to the promising properties of these materials, and the interesting as-yet-unseen phenomena occurring within them, for example: unconventional quantum Hall effect, absence of localization, and electrons behaving like massless relativistic particles (Figure \ref{fig:graphene}, right), providing a bridge between condensed matter physics and quantum electrodynamics \cite{katsnelson_graphene:_2007}.

\begin{figure}[H]
\hspace{1cm}
\includegraphics[width = 4.7cm]{Introduction/graphene.png}
\hspace{2cm}
\includegraphics[width = 8.5cm]{Introduction/disp_rel.png}
\caption[Graphene monolayer; graphene's dispersion relation.]{Left: \acf{AFM} picture of a graphene monolayer. The black area is a substrate used for fabrication purposes. The dark orange area is a monolayer of graphene. Right: Dispersion relation of graphene. The black line represents the Fermi energy. Close to it, the dispersion relation is linear, corresponding to massless excitations (taken from \cite{noauthor_nobel_nodate}). }
\label{fig:graphene}
\end{figure}	

On the other hand, \acp{TMD} are a recent member of the \ac{2D} materials family \cite{wang_electronics_2012, roldan_electronic_2014, xu_spin_2014}.
\acp{TMD} have been attracting interest because they seem to overcome some of the drawbacks of graphene in technological applications.
For example, monolayer graphene is gapless, while its bilayer counterpart has only a tunable, but small gap of the order of a tenth of an $eV$.
Contrastingly, \acp{TMD} have an intrinsic gap in excess of $1 \, eV$, being more promising in designing, for example, transistors.
Hole-doped \acp{TMD} are expected to show topological superconductivity \cite{hsu_topological_2017}, while the superconducting phase of graphene has been predicted, but is not easily attained.
Superconductivity in graphene-like \ac{2D} materials is important because it could boost high speed nanoelectronics.
Moreover, the presence of transition metal atoms in \acp{TMD} suggests the possibility of magnetic ordering \cite{braz_valley_2017}, which could be very relevant in nanospintronics applications.
Both topological superconductivity and magnetic ordering arise due to the effect of strong electron correlations.
Thus, to investigate these properties of \acp{TMD} when performing simulations, we need a computational method that is robust enough to capture the effects of electron interactions.

A nanoribbon consists of a \ac{2D} layer that can be regarded as infinitely long on one direction, but not on the other (Figure \ref{fig:fabrication}), so that edge states become relevant, and can be controlled to yield interesting properties.
For simulation purposes, it is natural to assume translational invariance along the ribbon's longitudinal direction, and use \acp{PBC}.
On the other direction, we use \acp{OBC}, effectively considering zigzag edges (Figure \ref{fig:nanoribbons}, left).

\begin{figure}[H]
\centering
\includegraphics[scale = 0.35]{Introduction/nanoribbons}
\caption[Fabrication of \ac{TMD} nanoribbons]{Fabrication of \ac{TMD} nanoribbons. From left to right, we see \ac{AFM} images showing the appeareance of nanostructures ranging from \ac{2D} nanoislands to nanoribbons, as the temperature of the substrate is increased. The nanoribbons are grown by taking advantage of the temperature dependence of shape transformations occuring during the nonequilibrium growth of this kind of surface-based nanostructures. (taken from \cite{chen_fabrication_2017})}
\label{fig:fabrication}
\end{figure}
   
A high density of low-energy electronic states is localized at the zigzag edges, decaying quickly in the bulk, which suggests the possibility of magnetic ordering.
In fact, a mean field solution of the Hubbard model for a graphene nanoribbon shows that magnetic moments are localized at the edges \cite{yazyev_emergence_2010} (Figure \ref{fig:nanoribbons}, right).
QMC has been used to investigate edge-state magnetism beyond mean field in graphene \cite{feldner_dynamical_2011, golor_quantum_2013, cheng_strain-induced_2015, raczkowski_interplay_2017, yang_strain-tuning_2017}.
However, edge magnetism in TMD nanoribbons remains unexplored \cite{davelou_nanoribbon_2017}.
 
\begin{figure}[H]
\hspace{2cm}
\begin{minipage}[c]{0.1\textwidth}
\includegraphics[scale = 0.22]{Introduction/zigzag}
\end{minipage} \hspace{6cm}
\begin{minipage}[c]{0.1\textwidth}
\includegraphics[scale = 0.23]{Introduction/edge_states}
\end{minipage}
 \caption[Zigzag edges of a nanoribbon and magnetism.]{Left: Two possible terminations of a \ac{TMD} nanoribbon condensing in a honeycomb lattice. Right: Local magnetic moments exist on the zig zag edges. The area of the circles corresponds to the magnitude of the magnetic moment, while the color red corresponds to a spin up density, and blue to a spin down density. The accumulation of electronic edge states leads to an \ac{AF} ground state (opposite edges with opposite magnetic moment). (taken from \cite{yazyev_emergence_2010}) \label{fig:nanoribbons}}
\end{figure}

While the zigzag graphene nanoribbon antiferromagnetic ground state is semiconducting, a state with interedge ferromagnetic orientation is a metal.
An example of an application based on the switching between the two states is a magnetorresistive sensor.
This device allows switching between low and high-resistance configurations, corresponding, respectively, to parallel, and antiparallel configurations of ferromagnetic leads at the ends of a nanoribbon.
An important application of this project is precisely the investigation of the possibility of edge-state magnetism, as is observed in graphene nanoribbons, for TMD nanoribbons, which could yield similarly innovative applications.


\subsection{Introduction to \acl{QMC}}

In principle, the properties of a quantum many-fermion system can all be deduced by solving an extremely complicated Schr\"odinger equation that takes into account the coupling of all (identical) particles of the system.
However, for the majority of systems the resulting integrals have no analytic solution, so we solve the problem by numerical integration.
But there is a myriad of methods to evaluate integrals numerically.
How do we pick the best one for this case? 
Multi-dimensional integrals are plagued by the curse of dimensionality.
Although the Newton-Cotes quadrature formulas (including, for example the Newton method, and Simpson's rules), Gaussian quadrature formulas, or Romberg's method all scale polynomially with the number of integration points, they become impractical as the dimension increases.
To use them, one would invoke Fubini's theorem to reduce the multi-dimensional integral to a series of one-dimensional integrals.
However, the number of function evaluations required to compute the whole integral grows exponentially with its dimension.
Monte Carlo methods preserve the polynomial scaling, thus yielding comparable accuracy with far less function evaluations. It is natural to use them since typically the state space of our quantum system is huge, leading to high dimensional integrals.

The Monte Carlo method is ubiquitous.
Its central idea is to use randomness to produce accurate estimates of deterministic integrals.
The term was coined by Nicolas Metropolis in 1949, first appearing in a seminal paper, in which it was described as a \say{statistical approach to the study of differential equations, or more generally, of integro-differential equations that occur in various branches of sciences}\cite{metropolis_monte_1949}.
Although it was used as early as 1777 in an experiment known as Buffon's needle - where one obtains an estimate of the constant $\pi$ by repeatedly throwing a needle randomly onto a sheet of paper with evenly spaced lines - it was crucially developed in the Los Alamos National Laboratory during World War II where the development of the first atomic bomb was completed, the primary objective of the Manhattan Project.
The method is particularly useful when one wants to sample from a probability distribution in an exponentially large state space.
In fact, it can in principle be used to solve any problem allowing a probabilistic formulation.

A variety of \acf{QMC} methods exists, using a sampling scheme based on the Metropolis algorithm, and variations thereof.
Variational and Diffusion \ac{QMC} are the simplest \ac{QMC} methods that allow one to capture some properties of correlated systems.
Although they already contain the main concepts used in this type of simulations, it is not always possible to use them. 
We will discuss their flaws and show how further refinement leads to the determinant, or auxiliary field method we ultimately used.

Using the Monte Carlo approach to study a many-fermion system implies overcoming a significant obstacle common to all \ac{QMC} methods - the so called \emph{fermion-sign problem}.
Pauli's exclusion principle implies that the many-fermion wave function is anti-symmetric, which leads to a sign oscillation that greatly impedes the accurate evaluation of averages of quantum observables.
The anti-symmetry constraint implies that a  straightforward weight interpretation of the wave function is not possible.
In the case of the finite temperature algorithm, the cancellations that occur when computing the average of any physical observable lead to poor statistical properties of the corresponding estimators.
This means that a massive amount of samples requiring enormous computer time are needed to obtain meaningful results.
In the case of the zero temperature algorithms, the situation is even worse.
It might not even be possible to design a stochastic process carrying the system to its ground state, as normally is done in \say{projective} methods\footnote{Methods that iteratively project a trial wave function onto the ground state.}: the wave function that is used as an initial proposal turns out to converge to a bosonic one, and the fermionic character of the system is lost.

As was proven by Troyer, the \emph{fermion-sign problem} has NP\footnote{NP or nondeterministic polynomial time, meaning that one can devise an algorithm that verifies the "yes" answer to a decision problem in polynomial time in the system size.
Note that the class $P$ - of polynomial time algorithms - is a subclass of NP.} computational complexity \cite{troyer_computational_2005}.
One of the greatest open questions in computer science is whether $P = NP$.
Solving the \emph{fermion-sign problem} would imply finding a solution to $P = NP$, which would constitute a major breakthrough.

\subsection{Monte Carlo Method in  Classical Statistical Physics}

Monte Carlo methods form the largest and arguably most useful class of numerical methods used to approach statistical physics problems.
Statistical physics often deals with computing quantities that describe the behavior of condensed matter systems.
The main difficulty one faces when doing so has to do with the collective nature of these systems.
Many identical components comprise these systems, and while the equations that govern the behavior of the whole may be easy to write down, their solution is in general a remarkably laborious mathematical problem.

The exponentially large number of configurations of a typical condensed matter system can be daunting.
Analytical solutions are more often than not hopeless and even numerical solutions are seemingly challenging.
However, they give valuable information lying between theory and experiment, and connecting them.

Suppose you try to sample uniformly from the probability distribution of all possible configurations of one of the aforementioned systems.
Changes are your algorithm will not end before the Universe does.
This is the computational complexity hurdle.
A related issue is that of finite size effects.
We are far from being able to simulate a macroscopically sized system. 
At best we can simulate a system that has only a minuscule fraction of the actual size of the corresponding real world system.
Amazingly there are techniques that allow us to efficiently extract information out of relatively small size simulations.
Nonetheless, increasing the system size systematically improves the reliability of a simulation.
Thus, the more efficient the algorithm is, the larger the system we can simulate in a fixed time.

The sheer number of equations describing a condensed matter system, and sometimes the strong coupling between them deems the task of finding an exact solution either very tough or even impossible.
It is not even clear whether an analytical solution would be of any use in many cases, and a statistical and numerical treatment often allows us to study more effectively the key properties of a system.

Strikingly, we are able to describe a system that is governed by a macroscopically large number of equations in terms of only a few variables.
The loss of information in doing so is only apparent.
The statistical description is so effective because most of the possible states of the system are extremely improbable when compared to the relevant very narrow part of configuration space.
The success of the field is largely attributed to the averaging out that naturally occurs when we measured a property of a macroscopic system.

The law of large numbers affords an approximation to integrals which can be written as an expectation of a random variable. Upon drawing enough independent samples from the corresponding distribution, the sample mean gets arbitrarily close to the integral at stake.

\begin{equation}\label{eq:int_mean}
\mathbb{E} [f(X)] = \int dx f(x) p(x),
\end{equation}
where $p(x)$ is the distribution of $X$. 

We could simply draw $M$ independent and identically distributed samples $x_{1,...M}$ from $p(x)$ and approximate the integral as

\begin{equation}
\frac{1}{M} \sum_{k=1}^M f (x_k) , 
\end{equation}
which in most cases converges to the desired expectation, as long as $M$ is large enough. How large?

\begin{equation}\label{eq:variance}
\text{Var}\bigg( \frac{1}{M} \sum_{k=1}^M f(x_k) \bigg) = \frac{1}{M} \text{Var}\bigg( f(x_1) \bigg) \sim \mathcal{O}\bigg(\frac{1}{M}\bigg)
\end{equation}

Thus, the correction to the sample mean is of order $\mathcal{O}(\frac{1}{\sqrt{M}})$ as long as $\text{Var}\big( f(x_1) \big) \sim 1$.

How do we sample from an arbitrary distribution $p(X)$? The idea is to first make an educated choice of a Markov Chain with the prescribed stationary distribution from which we ultimately desire to sample from, $p(X)$. After a sufficiently high number of steps, a Markov Chain Monte Carlo (MCMC) algorithm generates samples from the target distribution. Imposing some conditions on this Markov Chain, namely that it should be irreducible, aperiodic and positive recurrent, the ergodic theorem guarantees that the empirical measures of the aforementioned sampler approach the target stationary distribution. Another important condition to impose on this Markov Chain is detailed balance. Let the transition matrix be $\bm P = [P_{\mu \rightarrow \nu}]$, and the state space $\Omega$ be $\{\pi_\mu | \mu=1, ..., |\Omega| \}$, where $|\Omega|$ is the total number of possible states. Then, the condition of detailed balance is defined for all $\mu, \nu$ as

\begin{equation}\label{eq:detBal}
\pi_\mu P_{\mu \rightarrow \nu} = P_{\nu \rightarrow \mu} \pi_\nu
\end{equation}

Consider a system in state $\mu$ that makes transitions to state $\nu$ at a rate $R_{\mu \rightarrow  \nu}$ (that specifies the system's dynamics) and vice-versa.
The probability that a system is in state $\mu$ at time $t$, $p_\mu (t)$, such that $\sum_\mu p_\mu (t) = 1$, is given by the master equation(s):

\begin{equation}\label{eq:master}
\frac{d p_\mu}{dt} = \sum_\nu \big[ p_\nu (t) R_{\nu \rightarrow \mu} - p_\mu (t) R_{\mu \rightarrow \nu} \big] \quad \forall \mu \in \Omega
\end{equation}

The equilibrium occupation probabilities at finite temperature $T$ follow the Boltzmann distribution.

\begin{equation}
\pi_\mu = \lim_{t \rightarrow \infty} p_\mu (t) = \frac{1}{Z} e^{ - E_\mu / k_B T} ,
\end{equation}
where $E_\mu$ is the energy of state $\mu$, $k_B$ is Boltzmann's constant, and $Z$ is the partition function, from which we can extract thermodynamic functions in terms of expectations of physical quantities $\left\langle Q \right\rangle$, and response functions in terms of their variance $\sigma_Q^{\,\, 2}$.

Imposing the condition of stationarity on equation (\ref{eq:master}), $d_t p_\mu = 0$ , and noting that $ P_{\mu\rightarrow \nu} = R_{\mu\rightarrow \nu}  dt$, we obtain the equilibrium condition

\begin{equation}\label{eq:equilibrium}
\sum_\nu \pi_\mu P_{\mu \rightarrow \nu} = \sum_\nu P_{\nu \rightarrow \mu} \pi_\nu \iff \pi_\mu \sum_\nu P_{\mu \rightarrow \nu} = \sum_\nu P_{\nu \rightarrow \mu} \pi_\nu \iff \pi_\mu = \sum_\nu P_{\nu \rightarrow \mu} \pi_\nu
\end{equation}

This condition is enough to ensure convergence to an equilibrium of the dynamics of the Markov process.
However, it does not guarantee that the reached distribution is our desired one, $\bm \pi$, after running the process for long enough.
The probability of a state evolves according to

\begin{equation}
\pi_\nu ( t + 1 ) = \sum_\mu P_{\mu\rightarrow\nu}  \pi_\mu ( t ) \iff \bm \pi ( t + 1 ) = \bm P \bm \pi ( t )
\end{equation}

The stationary distribution of a Markov chain obeys

\begin{equation}
\bm \pi ( \infty ) = \bm P \bm \pi ( \infty ) ,
\end{equation}
however, condition (\ref{eq:equilibrium}) also allows  for limit cycles of length $n$, where $\bm \pi$ rotates around a number of configurations:

\begin{equation}
\bm \pi ( \infty ) = \bm P^n \bm \pi ( \infty ) ,
\end{equation}
where $\bm P^n$ is the n-th power of $\bm P$.

Detailed balance is a stronger requirement than the equilibrium condition, which eliminates limit cycles, thus ensuring that our sampler draws configurations from the desired distribution.
Intuitively, detailed balance corresponds to incorporating time-reversal symmetry in a simulation.
The condition imposes a constraint on the Markov transition probabilities:

\begin{equation}\label{eq:markovCondition}
\frac{P_{\mu\rightarrow\nu}}{P_{\nu\rightarrow\mu}} = \frac{\pi_\nu}{\pi_\mu} = e^{-\beta ( E_\nu - E_\mu ) }
\end{equation}

Crucially, Monte Carlo methods employ \emph{importance sampling}.
It turns out that we can improve upon our estimate of $\mathbb{E} [f(X)]$ by introducing a separate distribution $q(x)$, and defining the weight function as $w(x) = p(x)/ q(x)$.
Then, we can rewrite equation (\ref{eq:int_mean}):

\begin{equation}
\mathbb{E} [f(X)] = \int dx f(x) q(x) w(x) = \mathbb{E} [f(Y) w(Y)],
\end{equation}
with $Y \sim q$, i.e. the random variable $Y$ follows the distribution $q(Y)$.

It appears as though we didn't gain anything. However, by choosing $q$ wisely, we can actually reduce the variance we computed in equation (\ref{eq:variance}):

\begin{equation}
\text{Var}\bigg( \frac{1}{M} \sum_{k=1}^M f(y_k) w(y_k) \bigg) = \frac{1}{M} \text{Var}\bigg( f(y_1) w(y_1) \bigg)
\end{equation}

Since we didn't make any assumptions about $q(Y)$, it may be chosen so as to minimize the variance, hence the error of the Monte Carlo estimator, improving the approximation of the expectation. However, note that the error remains proportional to $\frac{1}{\sqrt{M}}$.
In practice, we devise a method to select the portion of state space which contains states contributing more significantly to the average.
This procedure ensures that $\text{Var}\big( f(y_1) w(y_1) \big) \sim 1$, improving the efficiency of our sampler.
The choice of the weight function translates to the averaging process by changing the estimator.
Explicitly computing the average

\begin{equation}
\left\langle Q \right\rangle = \frac{ \sum_\mu Q_\mu e^{-\beta E_\mu} }{ \sum_\mu e^{-\beta E_\mu}}
\end{equation}
is only tractable for very small systems.
In practice, we choose a subset of M states $\{\mu_1, \mu_2, ..., \mu_M \} $, and estimate the average as

\begin{equation}
Q_M = \frac{ \sum_{i=1}^M Q_{\mu_i} \pi_{\mu_i}^{-1} e^{ -\beta E_{\mu_i} } }{ \sum_{j=1}^M \pi_{\mu_j}^{-1} e^{ -\beta E_{\mu_j} }  }
\end{equation}

The estimate improves as $N$ increases, and when $N\rightarrow \infty$, $Q_M \rightarrow \left\langle Q \right\rangle$.
The accuracy of the estimator depends on the choice of the probabilities $\bm \pi$, which is related to the aforementioned variance.
For example, if $\bm \pi$ corresponds to the uniform distribution, i.e. $\pi_\mu = \frac{1}{| \Omega |} \forall \mu \in \Omega$, we have

\begin{equation}
Q_M = \frac{ \sum_{i=1}^M Q_{\mu_i} e^{ -\beta E_{\mu_i} } }{ \sum_{j=1}^M e^{ -\beta E_{\mu_j} }  } ,
\end{equation}
which turns out to be a poor choice since most of the visited states contribute negligibly to the average, leading to an inaccurate estimate.
The sum is dominated by a small subset of states, which we would like to access.
The idea of the Quantum (Classical) Monte Carlo method is to simulate the random quantum (thermal) fluctuations of a system, as it oscillates between states in a given time frame \cite{newman_monte_1999}. Instead of visiting these states uniformly, the most relevant part of the phase space is sampled more frequently, overcoming the seemingly exponential complexity of computing a sample mean numerically.
Even though only a small fraction of the system's states are sampled, we then obtain an accurate estimate of physical quantities of interest, namely energy, and correlation functions. This is implemented via a proposal-acceptance scheme.

To exploit the freedom given by condition (\ref{eq:markovCondition}), we note that we can always introduce a non-zero \say{stay-at-home} probability $P_{\mu \rightarrow \mu} \in [0, 1] $.
Regardless of its value, detailed balance is satisfied.
Similarly, any adjustment in $P_{\mu\rightarrow \nu}$ must be compensated by changing $P_{\nu\rightarrow \mu}$ to preserve their ratio.
Break the transition probability into a selection probability and an acceptance ratio, respectively:

\begin{equation}
\frac{P_{\mu\rightarrow\nu}}{P_{\nu\rightarrow\mu}}= \frac{G_{ \mu\rightarrow\nu} A_{\mu\rightarrow\nu}}{G_{ \nu\rightarrow\mu} A_{\nu\rightarrow\mu}}
\end{equation}

The Markov process now consists of generating a chain of states according to $G_{ \mu\rightarrow\nu}$, which are then accepted or rejected depending on $A_{\mu\rightarrow\nu}$.
Since we want to make the algorithm as efficient as possible, we want to make the acceptance ratio as close to one as possible to avoid useless steps.	
The most common way to do this is to fix the largest of them to one, and ajust the other accordingly.
The acceptance ratio will be close to one more often if $G_{ \mu\rightarrow\nu}$ includes most of the dependence of $P_{\mu\rightarrow\nu}$ on the characteristics of the states $\mu, \nu$.
Ideally, states would always be selected with the correct transition probability, and the acceptance ratio would be fixed to unity.
Good algorithms approach this situation, and much effort has been directed at optimizing them to do so.

By far, the most common sampling scheme choice is the Metropolis-Hastings algorithm, which we now  describe.

We select the transition probability to be uniform, and impose detailed balance through the choice of the acceptance ratios:

\begin{equation}
\frac{ P_{\mu\rightarrow\nu }}{ P_{\nu\rightarrow\mu }} = \frac{ A_{\mu\rightarrow\nu }}{ A_{\nu\rightarrow\mu } } = e^{-\beta ( E_\nu - E_\mu )}
\end{equation}

Suppose that $E_\mu < E_\nu $.
Then, $A ( \nu \rightarrow \mu ) > A ( \mu \rightarrow \nu ) $, and since only the acceptance ratio is fixed, we may freely set $A ( \nu \rightarrow \mu ) = 1$, which fixes $A ( \mu \rightarrow \nu ) = e^{-\beta ( E_\nu - E_\mu ) }$.
This choice maximizes the efficiency of the algorithm.
In short, we propose a random new state uniformly, and then we accept it with probability $A_{\mu\rightarrow \nu} = \min (1,  e^{-\beta ( E_\nu - E_\mu )})$.

Before we can use the states generated by our sampler to measure averages of physical quantities, we must reach the stationary distribution of the Markov process.
We consider this condition to be satisfied after a time $\tau_{\text{eq}}$, measured in steps of the algorithm.
When we consider a lattice model with a discrete set of states at each site $i = 1, 2, ..., N$, we say that a \emph{sweep} is completed whenever $N$ Monte Carlo steps are performed.
Thus, the number of \say{warm-up} sweeps is $W = \tau_{\text{eq}} / N$.

Before running a simulation, we need to decide how many sweeps we need to get an accurate estimate of the average.
The problem is that we need uncorrelated samples to average over.
To clarify, let us choose a specific model.
The paradigmatic model of statistical physics is the Ising model, a classical model of a magnet, which consists of considering spins-$1/2$ on a lattice, interacting only with their nearest neighbors.
Since each spin can only take on two values, say $\pm 1$, there are $2^N$ possible states.
The Hamiltonian reads

\begin{equation}
H = - J \sum_{\left\langle i, j \right\rangle } s_i s_j - B \sum_i s_i ,
\end{equation}
where $\left\langle i, j \right\rangle$ means that $i, j $ are nearest neighbors on the lattice.

A simple strategy to sample configurations of the Ising model is single-spin-flip dynamics.
We start with a random configuration of the spins, and then propose new configurations at each step by flipping a single spin at a given site.
A sweep is completed after we propose a spin flip at every site on the lattice.

Consecutive configurations generated by this chain differ only slightly.
Thus, it takes some time for the system to reach a configuration which is significantly different from the initial one.
This characteristic time is called the correlation time $\tau_c$.
A rigorous manner to estimate $\tau_c$ is through the time-displaced auto-correlation function associated to some quantity being measured.
An example of a relevant quantity for the case of the Ising model is the magnetization per site:

\begin{equation}
m = \frac{1}{N} \sum_i s_i
\end{equation}

Its associated time-displaced auto-correlation is

\begin{equation}
\chi_m ( t ) = \int dt' \bigg( m ( t' ) - \left\langle m \right\rangle \bigg) \bigg( m ( t' + t ) - \left\langle m \right\rangle \bigg) = \int dt' \bigg( m ( t' ) m ( t' + t ) - \left\langle m \right\rangle^2 \bigg)
\end{equation}
giving a measure of how correlated two measurements of the magnetization separated by a simulation time $t$ are.

The typical time-scale on which $\chi_m (t)$ falls off is a measure of the correlation time of the simulation.
In particular, at long times it falls off exponentially.
The definition of $\tau_c$ stems from this characteristic long-time behavior: $\chi_m (t) \sim e^{-t / \tau_c}$.
In practice, after waiting for $2\tau_c$, the measurements are virtually uncorrelated.
Let $A$ be the number of sweeps roughly  corresponding to $2\tau_c$ steps.
Then, if we make $S$ sweeps of the lattice during the simulation, the number of measurements (waiting for $A$ sweeps between them) is

\begin{equation}
M = \frac{S - W}{A}
\end{equation}

There are many ways to estimate $\tau_c$ from $\chi_m (t)$.
The simplest consists of making an exponential fit in a given range of times.
However, this might be unreliable since the estimate depends strongly on the chosen range.
An alternative is to compute the \say{integrated} correlation time:

\begin{equation}
\int_0^\infty dt \frac{\chi_m ( t ) }{\chi_m ( 0 ) } = \int_0^\infty dt e^{-t / \tau_c} = \tau_c  ,
\end{equation}
which is less sensitive, but not perfect either since the error that is introduced when the assumption  that \say{long-time} behavior has been reached is arbitrary and introduces an uncontrolled error.
Moreover, the very long-time behavior of the auto-correlation is rather noisy and must be excluded.

Using measured data for the magnetization at evenly-spaced times, we may construct the time-displaced auto-correlation function up to an unimportant constant, which does not affect the estimate of the correlation time:

\begin{equation}
\chi_m (t) = \frac{1}{ t_{\text{max}} - t } \sum_{t' = 0}^{t_{\text{max}} - t } m (t') m(t' + t) - \frac{1}{ t_{\text{max}} - t } \sum_{t' = 0}^{t_{\text{max}} - t } m (t') \frac{1}{ t_{\text{max}} - t } \sum_{t' = 0}^{t_{\text{max}} - t } m(t' + t) ,
\end{equation}
where $t_{\text{max}}$ is the total simulation time.

One should be careful when using this expression at very long times.
As $t$ approaches $t_{\text{max}}$, the upper limit of the sums decreases, and the integration interval becomes narrower.
Since $m (t)$ fluctuates randomly at very long times, the statistical error associated to $\chi_m (t)$ becomes more prominent as $t$ approaches $t_{\text{max}}$.
This turns out not to be problematic since typical simulations run for many correlation times.
Thus, the tails of the auto-correlation may safely be neglected because the correlations will have already vanished, by definition.

To finish our discussion on the issue of correlations, we note that if we have a total of $n_s$ samples of, for instance, magnetization data, the complexity of computing $\chi_m$ is $\mathcal{O}(n_s^2)$.
It is possible to speed up this process by computing its Fourier transform $\tilde{\chi}_m(\omega)$, and inverting to recover $\chi_m (t)$.
This can be done via a standard \ac{FFT} algorithm in $\mathcal{O}(2 n_s \log n_s )$ flops.
To do this, we apply the following trick

\begin{equation}
\begin{split}
\tilde{\chi}_m ( \omega ) &= \int dt e^{i\omega t} \int dt' \bigg( m ( t' ) - \left\langle m \right\rangle \bigg) \bigg( m ( t' + t ) - \left\langle m \right\rangle \bigg) \\
&= \int dt \int dt' e^{-i\omega t'} \bigg( m ( t' ) - \left\langle m \right\rangle \bigg) e^{i\omega ( t' + t )} \bigg( m ( t' + t ) - \left\langle m \right\rangle \bigg) \\
&= \tilde{m}' (\omega) \tilde{m}' (- \omega) = | \tilde{m}' (\omega) |^2 ,
\end{split} 
\end{equation}
where $\tilde{m}' (\omega)$ is the Fourier transform of $m' (t) = m(t) - \left\langle m \right\rangle$\footnote{The only difference between $\tilde{m}' (\omega)$ and $\tilde{m} (\omega)$, is that $\tilde{m}' (0) = 0$, while $\tilde{m} (0) \neq 0$. Thus, one can also compute $\tilde{m} (\omega)$ and then set its $\omega = 0$ component to zero.}.

\subsection{Variational Monte Carlo}

Variational techniques rely on an educated guess for the wave function of the system.
One introduces a set of variational parameters $\bm \alpha$ that are then tuned according to a variational principle.
Then, we may use the optimized trial wave function to compute physical quantities of interest using Monte Carlo.
The method is used to obtain zero temperature properties of a given model.
Note that it requires prior knowledge about the system to propose an approximate wave function in the first place.

A particularly relevant observable is the variational energy $E_V$ associated to a trial ground state.
Let $\bm r$ be the $3N$ spatial coordinates of the $N$ electrons.
Given the Hamiltonian of the system $\mathcal{H}$, and a trial wave function $\psi (\bm r)$ - a guess of the wave function representing the ground state - one can compute the corresponding variational energy.

\begin{equation}\label{eq:variational_energy}
E_V = \frac{\left\langle \psi | \mathcal{H} | \psi \right \rangle}{\left\langle \psi | \psi \right \rangle} = \frac{ \int d\bm r |\psi (\bm r)|^2 E_L (\bm r)}{\int d\bm r | \psi (\bm r)|^2 } = \int d\bm r\rho (\bm r) E_L (\bm r) ,
\end{equation}
where the local energy $E_L (\bm r)$ is defined as

\begin{equation}\label{eq:local_energy}
E_L = \frac{\mathcal{H} \psi (\bm r) }{\psi (\bm r)}
\end{equation}
and the probability distribution $\rho (\bm r)$ is defined as

\begin{equation}\label{eq:rho}
\rho (\bm r) = \frac{ | \psi (\bm r) |^2}{ \int d\bm r' | \psi (\bm r') |^2}
\end{equation}

Note that we managed to recast the variational energy as an average of the \emph{local} energy, $\left\langle E_L \right\rangle $, over the the distribution $\rho$.
This may be computed using the Monte Carlo method by sampling $M$ points $\bm r_k$ from distribution $\rho (\bm r)$:

\begin{equation}\label{eq:average}
E_V \approx \overline{E}_L = \frac{1}{M} \sum_{k= 1}^{M} E_L (\bm r_k) ,
\end{equation}
where $\overline {X}$ denotes a sample mean of the random variable $X$.

Let the ground state energy be $E_0$.
Then, states are optimized according to the variational principle:

\begin{equation}
E_V(\bm \alpha) = \frac{\left\langle \psi_{\bm \alpha} | \mathcal{H} | \psi_{\bm \alpha} \right\rangle}{\left\langle\psi_{\bm \alpha} | \psi_{\bm \alpha} \right\rangle} \ge E_0,
\end{equation}
where $\psi_{\bm \alpha}$ is the trial ground state wave function for the set of variational parameters ${\bm \alpha}$.

By varying $\bm \alpha$ we aim to obtain a variational energy that is as close as possible to the true ground state energy.
Since $E_V(\bm \alpha)$ is bounded from below, this is equivalent to minimizing it in the hope that $E_V(\bm \alpha_{min}) \gtrsim E_0$, i.e. the bound is tight.

The finite sampling size $M$, of course, introduces a statistical error common to all Monte Carlo methods. 
However, the use of an approximate wave function introduces a systematic error that is hard to control since trial wave functions are generally introduced based on approximate, or heuristic arguments.

\subsection{Diffusion Monte Carlo}\label{subsec:dmc}

Variational Monte Carlo is severely limited by the use of a trial wave function $\psi_{\bm \alpha} (\bm r)$ because we may even not have enough information to even construct a reliable variational wave function.

Diffusion \ac{QMC} allows the simulation of a many-body system while having only a limited knowledge of the system's physical properties.
While it is exact for many-boson systems, it is only approximate for many-fermion systems.
The idea is to map the Schr\"odinger equation into  an imaginary-time diffusion equation.
Excited states are then filtered out by a diffusion process as we advance in imaginary-time.
In imaginary-time $\tau = - i t$, the solution to the Schr\"odinger equation in terms of a formal series expansion in the eigenfunctions of the hamiltonian becomes a series of transients $e^{-E_n \tau}, \, n \in \mathbb{N}$.
The longest lasting of these is the ground state  \cite{kosztin_introduction_1996}.

The idea of the diffusion method is to generate samples using the exact ground state wave function $\psi_0 (\bm r)$ \cite{toulouse_chapter_2016}.
The associated exact energy $E_0$ is the matrix element of the hamiltonian calculated using a trial wave function and the ground state wave function.

\begin{equation}
E_0 = \frac{ \left\langle \psi_0 |E_0 \mathbbm{1} | \psi \right\rangle}{\left\langle \psi_0 | \psi \right\rangle} = \frac{\left\langle \psi_0 | \mathcal{H} | \psi \right\rangle}{ \left\langle\psi_0 | \psi \right\rangle} = \frac{\int d\bm r \psi_0^\star (\bm r) \psi (\bm r) E_L (\bm r)}{\int d\bm r\psi_0^\star (\bm r) \psi (\bm r)}
\end{equation}

Note that using this trick we avoid the computation of $\mathcal{H} \psi_0 = E_0 \psi_0$, that is, the ground state energy.
Instead, we approximate the integral by considering $M$ configuration samples $\bm r_{k = 1,..., N}$ in a similar spirit to that of Variational \ac{QMC}.
Notice that the integral consists of a local energy of the trial wave function $E_L (\bm r) = \frac{\mathcal{H} \psi (\bm r)}{\psi (\bm r)}$ averaged over a mixed distribution from which we draw a sample of points $\bm r_{k=1,...M}$:

\begin{equation}
f(\bm r) = \frac{\psi_0^\star (\bm r) \psi (\bm r) }{ \int d\bm r  \psi_0 (\bm r) \psi (\bm r)}
\end{equation}

Although the method is, of course, aimed at probing many-body systems, let us consider a single particle in \acl{1D} for simplicity for illustrating the method.
Performing a Wick rotation - effectively going to imaginary time - and shifting the energy, the Schr\"odinger equation becomes

\begin{equation}
\frac{\partial \psi ( x, \tau )}{\partial\tau}  = -\frac{1}{2m}\frac{\partial^2 \psi ( x, \tau )}{\partial x^2} - \bigg[ V(x) - E_T \bigg] \psi( x, \tau ) 
\end{equation}

The exact ground state wave function $\psi_0 ( x )$ is obtained as the longest lasting transient state in imaginary time: we are interested in the asymptotic behavior of the series expansion constituting the formal solution of the Schr\"odinger equation

\begin{equation}
\psi (x, \tau) = \sum_{n=0}^{\infty} c_n \psi_n (x) e^{-(E_n - E_T)\tau}
\end{equation}

Imaginary time evolution is governed by

\begin{equation}\label{eq:im_ev}
\begin{split}
&\left| \psi (t) \right\rangle = \lim_{\tau \rightarrow \infty} \sum_i e^{-(E_i - E_T) \tau} \left|\psi_i \right\rangle \left\langle \psi_i | \psi \right\rangle = \\
&= \lim_{\tau \rightarrow \infty} e^{-(E_0 - E_T)\tau} \left| \psi_0 \right\rangle \left\langle \psi_0 | \psi \right\rangle 
\end{split}
\end{equation}


If $E_T > E_0$ the wave function diverges exponentially fast: $\lim_{\tau \rightarrow \infty} \psi ( x, \tau) = \infty$.
Similarly, for $E_T < E_0$ it vanishes exponentially fast: $\lim_{\tau \rightarrow \infty} \psi ( x, \tau) = 0$.
However, if $E_T = E_0$ the wave function converges to the ground state one up to a constant factor.

\begin{equation}\label{eq:dmc}
\lim_{\tau \rightarrow \infty} \psi ( x, \tau) = c_0 \psi_0 (x) \,\,\, \text{, or} \quad \lim_{\tau \rightarrow \infty} \left|\psi (\tau) \right\rangle \propto \left| \psi_0 \right\rangle
\end{equation}

Diffusion \ac{QMC} makes use of equation (\ref{eq:dmc}), approximating $\phi_0(x)$ by $\psi (x, \tau)$ for sufficiently long time.
The only requirement is that $\psi (x, \tau)$ and $\psi_0(x)$ overlap significantly so that $c_0$ is large enough to be numerically measurable, and we can always center a positive trial wave function in a region where $\psi_0(x)$ is large enough.
Of course, this is always possible for a single particle, but note that it might fail for a many-fermion system, for which the wave function crosses a number of nodes due to its anti-symmetric nature.
\section{Introduction to \acl{QMC}}
\label{sec:introQMC}

Solving the many-body problem remains one of the greatest challenges in physics.
Following the wealth of attempts at such pursuit, certain phenomena arising due to the strong interactions in quantum systems are explained in different theoretical frameworks, namely superconductivity, the Mott metal-insulator transition, and fractional quantum Hall effect.
All of these breakthroughs represented revolutions in their respective fields with significant scientific and technological impact.
However, only in very limited cases does an actual analytical solution exist for the  Schr\"odinger equation for a system of interacting particles.
One must resort to sophisticated approximation methods to obtain  information about the role played by the competing interactions under various conditions in the aforementioned cases.
It is then natural that numerical methods have become prominent as a tool for extracting useful information about this type of systems.
\ac{QMC} is amongst the most accurate and extensively studied ones.
The idea of all \ac{QMC} methods is to reduce the interacting problem to solving a set of integrals, which can be evaluated numerically through a standard stochastic procedure.
These integrals are arrived at upon formulating the quantum many-body description of the system using the Schr\"odinger equation.
Hence the name \acl{QMC}, which is used to distinguish it from Classical Monte Carlo.
In the classical version, one measures thermal averages, while in the quantum version, one measures expectations of operators over the Hilbert space of the system, corresponding to physical observables that fluctuate with a dynamics given by the Schr\"odinger equation (and, of course, can also have thermal fluctuations).
In fact, the dynamics of a quantum system are encoded in the Hamiltonian operator.
In the case of graphene-like \ac{2D} materials, one usually uses a tight-binding model.
It is found that the dynamics given by the tight-binding Hamiltonian is sufficient to describe most properties of graphene.
However, in other materials, such as \acp{TMD}, electron-electron interactions are stronger, and Hubbard-type models could give us a more accurate picture of the phenomena that occur within them.

In principle, the properties of a quantum many-fermion system can all be deduced by solving an extremely complicated Schr\"odinger equation that takes into account the coupling of all (identical) particles of the system.
However, for the majority of systems the resulting integrals have no analytic solution, so we solve the problem by numerical integration.
But there is a myriad of methods to evaluate integrals numerically.
How do we pick the best one for this case? 
Multi-dimensional integrals are plagued by the curse of dimensionality.
Although the Newton-Cotes quadrature formulas (including, for example the Newton method, and Simpson's rules), Gaussian quadrature formulas, or Romberg's method all scale polynomially with the number of integration points, they become impractical as the dimension increases.
To use them, one would invoke Fubini's theorem to reduce the multi-dimensional integral to a series of one-dimensional integrals.
However, the number of function evaluations required to compute the whole integral grows exponentially with its dimension.
The Monte Carlo method preserves the polynomial scaling, thus yielding comparable accuracy with far less function evaluations.
It is natural to use it since typically the state space of our quantum system is huge, leading to high dimensional integrals.

The Monte Carlo method is ubiquitous.
Its central idea is to use randomness to produce accurate estimates of deterministic integrals.
The term was coined by Nicolas Metropolis in 1949, first appearing in a seminal paper, in which it was described as a \say{statistical approach to the study of differential equations, or more generally, of integro-differential equations that occur in various branches of sciences}\cite{metropolis_monte_1949}.
Although it was used as early as 1777 in an experiment known as Buffon's needle - where one obtains an estimate of the constant $\pi$ by repeatedly throwing a needle randomly onto a sheet of paper with evenly spaced lines - it was crucially developed in the Los Alamos National Laboratory during World War II where the development of the first atomic bomb was completed, the primary objective of the Manhattan Project.
The method is particularly useful when one wants to sample from a probability distribution in an exponentially large state space (like the huge Hilbert space of an interacting electron system), but it can, in principle, be used to solve any problem allowing a probabilistic formulation.
A variety of \ac{QMC} methods exists, using a sampling scheme based on the Metropolis algorithm, and variations thereof.
Variational and Diffusion \ac{QMC} are the simplest \ac{QMC} methods that allow one to capture some properties of correlated systems, but it is not always ideal or even possible to use them. 
We will discuss their flaws and show how further refinement leads to the auxiliary field method we ultimately used.

Using the Monte Carlo approach to study a many-fermion system implies overcoming a significant obstacle common to all \ac{QMC} methods - the so called \emph{fermion sign problem}.
Pauli's exclusion principle implies that the many-fermion wave function is anti-symmetric, which leads to a sign oscillation that greatly impedes the accurate evaluation of averages of quantum observables.
The anti-symmetry constraint implies that a  straightforward weight interpretation of the wave function is not possible.
In the case of the finite temperature algorithm, the cancellations that occur when computing the average of any physical observable lead to poor statistical properties of the corresponding estimators.
This means that a massive amount of samples requiring enormous computer time are needed to obtain meaningful results.
In the case of the zero temperature algorithms, the situation is even worse.
It might not even be possible to design a stochastic process carrying the system to its ground state, as normally is done in \say{projective} methods\footnote{Methods that iteratively project a trial wave function onto the ground state.}: the wave function that is used as an initial proposal turns out to converge to a bosonic one, and the fermionic character of the system is lost.
As was proven by Troyer, the \emph{fermion sign problem} has NP\footnote{NP or nondeterministic polynomial time, meaning that one can devise an algorithm that verifies the "yes" answer to a decision problem in polynomial time in the system size.
Note that the class $P$ - of polynomial time algorithms - is a subclass of NP.} computational complexity \cite{troyer_computational_2005}.
One of the greatest open questions in computer science is whether $P = NP$.
Solving the \emph{fermion sign problem} would imply finding a solution to $P = NP$, which would constitute a major breakthrough.

\subsection{Variational Monte Carlo}

Variational techniques rely on an educated guess for the wave function of the system.
One introduces a set of variational parameters $\bm \alpha$ that are then tuned according to a variational principle.
Then, we may use the optimized trial wave function to compute physical quantities of interest using Monte Carlo.
The method is used to obtain zero temperature properties of a given model.
Note that it requires prior knowledge about the system to propose an approximate wave function in the first place.

A particularly relevant observable is the variational energy $E_V$ associated to a trial ground state.
Let $\bm r$ be the $3N$ spatial coordinates of the $N$ electrons.
For simplicity, let us ignore all other degrees of freedom, such as spin.
Given the Hamiltonian of the system $\mathcal{H}$, and a trial wave function $\psi_T (\bm r)$ - a guess of the wave function representing the ground state - one can compute the corresponding variational energy by averaging over a \say{local} energy:

\begin{equation}\label{eq:variational_energy}
E_V = \frac{\left\langle \psi_T | \mathcal{H} | \psi_T \right \rangle}{\left\langle \psi_T | \psi_T \right \rangle} = \frac{ \int d\bm r |\psi_T (\bm r)|^2 E_L (\bm r)}{\int d\bm r | \psi_T (\bm r)|^2 } = \int d\bm r\rho (\bm r) E_L (\bm r) , \text{where}
\end{equation}

\begin{equation}\label{eq:local_energy}
E_L = \frac{\mathcal{H} \psi_T (\bm r) }{\psi_T (\bm r)}   \quad \text{and} \quad \rho (\bm r) = \frac{ | \psi_T (\bm r) |^2}{ \int d\bm r' | \psi_T (\bm r') |^2}
\end{equation}

Note that we managed to recast the variational energy as an average of the \emph{local} energy, $\left\langle E_L \right\rangle $, over the the distribution $\rho$.
This may be computed using the Monte Carlo method by sampling $M$ points $\bm r_k$ from the distribution $\rho (\bm r)$.
Denoting the sample mean of the random variable $X$ as $\overline {X}$:

\begin{equation}\label{eq:average}
E_V \approx \overline{E}_L = \frac{1}{M} \sum_{k= 1}^{M} E_L (\bm r_k) ,
\end{equation}

Let the ground state energy be $E_0$.
Then, states are optimized according to the variational principle:

\begin{equation}
E_V(\bm \alpha) = \frac{\left\langle \psi_{\bm \alpha} | \mathcal{H} | \psi_{\bm \alpha} \right\rangle}{\left\langle\psi_{\bm \alpha} | \psi_{\bm \alpha} \right\rangle} \ge E_0,
\end{equation}
where $\psi_{\bm \alpha}$ is the trial ground state wave function for the set of variational parameters ${\bm \alpha}$.
By varying $\bm \alpha$ we aim to obtain a variational energy that is as close as possible to the true ground state energy, and use the corresponding trial wave function to compute averages of other observables.
Since $E_V(\bm \alpha)$ is bounded from below, this is equivalent to minimizing it in the hope that $E_V(\bm \alpha_{min}) \gtrsim E_0$, i.e. the bound is tight.
The finite sampling size $M$, of course, introduces a statistical error common to all Monte Carlo methods. 
However, the use of an approximate wave function introduces a systematic error that is hard to control since trial wave functions are generally introduced based on approximate, or heuristic arguments.

\subsection{Diffusion Monte Carlo and projective methods}\label{subsec:dmc}

Variational Monte Carlo is severely limited by the use of a trial wave function $\psi_T (\bm r)$ because we may not even have enough information to even construct a reliable variational wave function in the first place.
Diffusion \ac{QMC} allows the simulation of a many-body system while having only a limited knowledge of the system's physical properties.
As a projective method, it is exact for many-boson systems, while being only approximate for many-fermion systems.
The idea is to map the Schr\"odinger equation onto an imaginary-time diffusion equation.
Excited states are then filtered out by a diffusion process as we advance in imaginary-time.
In imaginary-time $\tau = i t$, the solution to the Schr\"odinger equation in terms of a formal series expansion in the eigenfunctions of the Hamiltonian becomes a series of \say{transient} wavefunctions weighted by $e^{-E_n \tau}, \, n \in \mathbb{N}$.
Within precision and accuracy constraints, the longest lasting of these is the ground state \cite{kosztin_introduction_1996}.
Thus, the idea of the diffusion method is to generate samples using the exact ground state wave function $\psi_0 (\bm r)$ \cite{toulouse_chapter_2016}.
The associated exact energy $E_0$ is the matrix element of the hamiltonian calculated using a trial wave function and the ground state.

\begin{equation}
E_0 = \frac{ \big( \left\langle \psi_0 |E_0 \big) \big( \mathbbm{1} | \psi_T \right\rangle \big)}{\left\langle \psi_0 | \psi_T \right\rangle} = \frac{\left\langle \psi_0 | \mathcal{H} | \psi_T \right\rangle}{ \left\langle\psi_0 | \psi_T \right\rangle} = \frac{\int d\bm r \psi_0^\star (\bm r) \psi_T (\bm r) E_L (\bm r)}{\int d\bm r\psi_0^\star (\bm r) \psi_T (\bm r)}
\end{equation}

Note that using this trick we avoid the computation of $\mathcal{H} \psi_0 = E_0 \psi_0$, that is, the ground state energy.
Instead, we approximate the integral by considering $M$ configuration samples $\bm r_{k = 1,..., M}$ in a similar spirit to that of Variational \ac{QMC}.
Notice that the integral consists of a local energy of the trial wave function $E_L (\bm r) = \frac{\mathcal{H} \psi (\bm r)}{\psi (\bm r)}$ averaged over a mixed distribution from which we draw a sample:

\begin{equation}
f(\bm r) = \frac{\psi_0^\star (\bm r) \psi_T (\bm r) }{ \int d\bm r  \psi_0 (\bm r) \psi_T (\bm r)}
\end{equation}

Although the method is, of course, aimed at probing many-body systems, let us consider a single particle in \acs{1D}, for simplicity, to illustrate the method.
Performing a Wick rotation - effectively going to imaginary time - and shifting the energy, the Schr\"odinger equation becomes (with $\hbar = 1$)

\begin{equation}
\frac{\partial \psi_T ( x, \tau )}{\partial\tau}  = -\frac{1}{2m}\frac{\partial^2 \psi_T ( x, \tau )}{\partial x^2} - \bigg[ V(x) - E_T \bigg] \psi_T( x, \tau ) 
\end{equation}

The exact ground state wave function $\psi_0 ( x )$ is obtained as the longest lasting transient state in imaginary time: we are interested in the asymptotic behavior of the series expansion constituting the formal solution of the Schr\"odinger equation

\begin{equation}
\psi_T (x, \tau) = \sum_{n=0}^{\infty} c_n \psi_n (x) e^{-(E_n - E_T)\tau}
\end{equation}

Imaginary time evolution is governed by

\begin{equation}\label{eq:im_ev}
\left| \psi_T (t) \right\rangle = \lim_{\tau \rightarrow \infty} \sum_n e^{-(E_n - E_T) \tau} \left|\psi_n \right\rangle \left\langle \psi_n | \psi_T \right\rangle = \lim_{\tau \rightarrow \infty} e^{-(E_0 - E_T)\tau} \left| \psi_0 \right\rangle \left\langle \psi_0 | \psi_T \right\rangle 
\end{equation}

If $E_T > E_0$ the wave function diverges exponentially fast: $\lim_{\tau \rightarrow \infty} \psi_T ( x, \tau) = \infty$.
Similarly, for $E_T < E_0$ it vanishes exponentially fast: $\lim_{\tau \rightarrow \infty} \psi_T ( x, \tau) = 0$.
However, if $E_T = E_0$ the wave function converges to the ground state one up to a constant factor, $c_0 = \left\langle \psi_0 | \psi_T \right\rangle$.

\begin{equation}\label{eq:dmc}
\lim_{\tau \rightarrow \infty} \psi_T ( x, \tau) = c_0 \psi_0 (x) \quad \text{or} \quad \lim_{\tau \rightarrow \infty} \left|\psi_T (\tau) \right\rangle \propto \left| \psi_0 \right\rangle
\end{equation}

Diffusion \ac{QMC} makes use of Eq. (\ref{eq:dmc}), approximating $\psi_0(x)$ by $\psi_T (x, \tau)$ for sufficiently long time.
The only requirement is that $\psi_T (x, \tau)$ and $\psi_0(x)$ overlap significantly so that $c_0$ is large enough to be numerically measurable, and we can always center a positive trial wave function in a region where $\psi_0(x)$ is large enough and positive.
If the latter condition does not hold, the wave function converges to a bosonic, instead of a fermionic one.
Of course, these conditions can always be met for a single particle, but note that they might fail for a many-fermion system, for which the wave function crosses a number of nodes due to its anti-symmetric nature.

\subsection{Auxiliary Field \acs{QMC} and the Fermion Sign Problem}
\label{subsec:introAFQMC}

As we have seen, the major drawback of the variational method was that it demanded \emph{a priori} knowledge of a reasonable variational wave function describing, at least partly, some of the physics of the problem.
Diffusion \acs{QMC} demands less: we need only propose a trial wave function that overlaps with the ground state.
However, none of these methods allow us to probe systems at finite temperature.
Moreover, they both require some prior knowledge about the system, which may not always be available.

An alternative method is based on introducing an additional lattice bosonic field that mediates the electron-electron interaction.
The interacting problem then becomes a problem of independent fermions coupled to an external field, and the fermionic part of the partition function can be traced out explicitly, leaving the contribution of a \emph{discrete}\footnote{The introduced field is discrete (and \emph{binary}) because each fermionic state can only have occupations $n = 0, 1$. Although, there is a finite number of field configurations, the number grows exponentially with the number of sites on the lattice.} bosonic field, $\bm h$.
This contribution can be evaluated numerically by employing importance sampling over the field configurations.
Auxiliary field \acs{QMC} relies on a mapping to a so called \say{classical} system (in quotes because there may be no actual classical analogue):

\begin{equation}\label{eq:Zsign}
Z = \Tr [ e^{-\beta \mathcal{H} } ] = \sum_{\{ \bm h\} } \sum_{\text{fermionic}} e^{-S} = \sum_c p_c ,
\end{equation}
but some of the \say{probabilities} can actually be negative $p_c < 0$.
This occurs due to the antisymmetry of the many-electron wavefunction under electron exchange, and is at the root of the sign problem.
Here, $S$ is a fermion-boson action that we shall write out explicitly later.
For a fixed configuration of the bosonic field, we sum over the fermionic part exactly to obtain the weight of each configuration $p_c$.
The sum over $\bm h$ is carried out stochastically.

The negative weight problem may easily be circumvented when computing averages of observables:

\begin{equation}\label{eq:signSampling}
\left\langle A \right\rangle = \frac{\sum_c A ( c ) p ( c )}{\sum_c p ( c ) } = \frac{\sum_c A ( c )|  p ( c ) | \text{sign}[p(c)] / \sum_c | p ( c ) | }{\sum_c  |  p ( c ) | \text{sign}[p(c)] /  \sum_c | p ( c ) |} \equiv \frac{\left\langle A s \right\rangle_{|p|}}{\left\langle s \right\rangle_{|p|}} ,
\end{equation}
where $s(c) = \text{sign} [ p ( c ) ]$, and $| p ( c ) | $ corresponds to an auxiliary bosonic system (also coupled to the bosonic field) corresponding to the original fermionic system, and for which there is no sign problem.

The relative error $\Delta s / \left\langle s \right\rangle$ increases exponentially with the number of particles, with inverse temperature, and possibly with other parameters of the specific model to be studied \cite{troyer_computational_2005, hou_numerical_2009}.
To see this, we start by noting that the average sign is the ratio between the partition functions of the fermionic ($Z = \sum_c p(c)$) and bosonic systems ($Z' = \sum_c | p ( c ) |$).
In terms of the difference in free energy densities, $\left\langle s \right\rangle = Z / Z' = e^{-\beta N_p \Delta f}$, implying that for $M$ samples, the error of the denominator of Eq. (\ref{eq:signSampling}) becomes

\begin{equation}
\frac{\Delta s}{\left\langle s \right\rangle} = \frac{\sqrt{(\left\langle s^2 \right\rangle - \left\langle s \right\rangle^2 )/ M }}{\left\langle s \right\rangle} = \frac{ \sqrt{ 1 - \left\langle s \right\rangle^2}  }{\sqrt{M} \left\langle s \right\rangle} \propto \frac{e^{\beta N_p \Delta f}}{\sqrt{M}} ,
\end{equation}
and similarly for the numerator of Eq. (\ref{eq:signSampling}).

Auxiliary field, or determinant \acs{QMC} can also be formulated to probe ground state properties, and a sign problem arises similarly.
In fact, this problem plagues all \acs{QMC} methods, even though we showed it only for the determinant method\footnote{So called because, as we shall show later, $p_c$ boils down to a product of determinants that depends on the energy scales of the problem.}.
The latter is the most robust, unbiased, and reliable method, with a generally modest sign problem, hence we choose it to carry out our simulations.

Furthermore, in general, it suffices to use the finite temperature auxiliary field  method with $\beta$ large enough to probe ground state properties (for example, this is shown numerically for the Hubbard model on the square lattice in \cite{white_numerical_1989}).
In this case, the inverse temperature may be regarded as being analogous to a  projective parameter $\Theta$, characterizing convergence to the ground state, within statistical uncertainty.
Projector \ac{QMC}, the zero temperature version of auxiliary field \ac{QMC} is based on an equation similar to Eq.(\ref{eq:dmc}).
Any observable $A$ is computed by use of a trial wave function with some overlap with the ground state $\left\langle \psi_T | \psi_0 \right\rangle \neq 0$ (see \cite{f._assaad_quantum_2002} for more details on the projector method; in this work we focus on the finite temperature version since it is more general):

\begin{equation}
\left\langle A \right\rangle = \lim_{\Theta \rightarrow \infty} \frac{\left\langle \psi_T | e^{-\Theta \mathcal{H} } A e^{-\Theta \mathcal{H} } | \psi_T \right\rangle }{\left\langle \psi_T | e^{- 2 \Theta \mathcal{H} } | \psi_T \right\rangle}
\end{equation}

Note that auxiliary field \ac{QMC} is more powerful than the variational and diffusion methods outlined before since it requires much less \emph{a priori} information about the system.
Perhaps more importantly, recent work suggests that it can be used in conjunction with neural networks to discover quantum phase transitions in correlated systems  \cite{broecker_machine_2017} in what could be a revolution in the field.
\section{Original Contributions}
\label{sec:int_contributions}

In this work we focus mainly on the study of the emergence of magnetic ordering in \ac{TMD} nanoribbons.
To carry out this study, we use our own original implementation of the auxiliary field \ac{QMC} algorithm in \texttt{C++}.
To validate the built software, we consider previously well known models (such as the Hubbard model on the \acs{1D} chain, and the square lattice), and compare our \ac{QMC} results with those obtained in the mean field approximation and benchmark them against existing, \say{tried and true}  implementations (namely \texttt{ALF} \cite{bercx_alf_2017} and \texttt{QUEST} \cite{noauthor_quest_2012}), and early seminal studies \cite{hirsch_discrete_1983,white_numerical_1989}.
The code we wrote can be used to simulate low-dimensional Hubbard-like models with different geometries to extend this work further.
Additionaly, using our code, we compare different options to stabilize the matrix products needed to perform the simulations, and characterize the fermion sign problem.

To study \ac{TMD} nanoribbons, we extend a minimal three-band tight binding model \cite{liu_three-band_2013} of \acp{TMD} to the interacting case.
Then, we show the emergence of edge-state magnetism in \ac{TMD} nanoribbons via original mean field and \ac{QMC} calculations, which we compare with one another.
\section{Outline}
\label{sec:int_outline}

We started this introductory chapter with the concept of emergence in strongly correlated electron systems.
Then, we proceeded to discuss the particular example, which we will tackle in this thesis: the \acs{2D} \acs{TMD} nanoribbon.
In this system, electron correlations give rise to interesting forms of magnetism, which, as far as we know, were unexplored numerically before this work.
To tackle this interacting fermion system, we resort to a  state-of-the-art algorithm that has been in continuous development over the last 30 years.

In the chapter (), we will discuss the Hubbard mode, an ubiquitous model of electron correlations.
We discuss analytical solutions of simple limiting cases, outline some approximation methods, and introduce Green's functions, which will turn out to be the main object of our simulations.
Moreover, we formulate the mean field theory of the Hubbard model.

Then, we proceed to the simulation method.
In chapter (), we summarize the main ideas about how to apply the Monte Carlo method to statistical physics problems.
In this context, we use original results of our simulations to illustrated the concepts in the specific context of our problem.

Chapter () introduces the auxiliary field method, and its various challenges, namely low temperature, and large size stabilization.

In the following chapter (), we apply the code we implemented for a variety of systems, benchmarking our code, and carrying out some original calculations.

Finally, in chapter (), we conclude by discussing the results obtained in the previous chapter, and propose future work to be done on the topic.

\cleardoublepage