\section{State of The Art}
\label{sec:int_state}

These notes aim to present the basic aspects of a numerical technique used to simulate the Hubbard model in a self-contained, tutorial form. The idea is to reduce our problem to solving a set of integrals, which we evaluate numerically through a standard stochastic procedure, known as the Monte Carlo method. These integrals are arrived at upon formulating the quantum many-body description of the system using the Schr\"odinger equation. Hence the name Quantum Monte Carlo, which is used to distinguish it from Classical Monte Carlo. In the classical version, one measures thermal averages. In the quantum version, one measures expectations of operators over the Hilbert space of the system, corresponding to physical observables that fluctuate with a dynamics given by Schr\"odinger's equation.
