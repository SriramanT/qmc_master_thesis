\section{Nanoribbons}
\label{sec:nanoribbon}

In this section, we apply our code to the case of honeycomb lattices with boundary conditions corresponding to nanoribbons. These structures are much longer on one direction than on the other, resembling a ribbon, hence their name.
The low energy electronic states on the edges of this ribbon might lead to interesting magnetic behavior in \ac{TMD} nanostructures (as indeed they do in graphene nanostructures \cite{yazyev_emergence_2010}) and it is this possibility was unexplored numerically before this work \cite{feldner_dynamical_2011, golor_quantum_2013}, as was mentioned on chapter \ref{cap:int}.
A nanoribbon is much longer on one direction than on the other, i.e. $l \gg w$. 
This condition corresponds to taking $N_x \gg N_y$ in our conventions (see Fig.(\ref{fig:nanoribbon}), where this condition is, of course, not obeyed solely for the sake of giving a good visual representation of the boundary conditions, and the numbering system).

\subsection{Graphene}
\label{sec:graphene}

We use three coordinates to label each site on the honeycomb lattice, by taking advantage of its bipartite nature.
Regarding the honeycomb lattice as two interpenetrating triangular sublattices $\mathcal{A}$ and $\mathcal{B}$, we take the axes $x$ and $y$ to be along the primitive vectors of each triangular sublattice.
Along the $x$-direction, the ribbon is supposed to be very long, which justifies the fact that we take \acp{PBC}.
In contrast, in the narrow $y$-direction we take \acp{OBC}.
To number the sites on the ribbon, we introduce an additional coordinate labeling the sublattice: $z = 0$, if the site is in sublattice $\mathcal{A}$, and $z = 1$ if the site is in sublattice $\mathcal{B}$.
We then adopt the numbering convention for the sites $i = 0,1, ..., 2 N_x N_y - 1$ of the lattice $\mathcal{L}$:

\begin{equation}\label{eq:numbering}
i (x, y, z) = N_x N_y z + N_x y + x,
\end{equation}
where $x = 0, ..., N_x - 1$, $y = 0, ..., N_y - 1$, and $z = 0, 1$ define each element $\bm r = (x, y, z) \in \mathcal{L}$.

\begin{figure}[H]
	\centering
\includegraphics[trim={0 8cm 0 2.5cm},clip, width=0.8\linewidth]{Applications/nanoribbon.pdf}
	\caption[Boundary conditions on the nanoribbon.]{Boundary conditions on the nanoribbon for $N_x = 5, \, N_y = 4$. The orange circles correspond to sublattice $\mathcal{A}$, and the blue circles correspond to sublattice $\mathcal{B}$.
	The counting starts at 1 (i.e. the numbers correspond to $i+1$, since $i$ starts at 0).}
	\label{fig:nanoribbon}
\end{figure}

The geometry of the system appears through the hopping matrix $\bm K$ in our code.
This numbering system makes it straightforward to find the neighbors of each site.
Let us begin by considering a site that is not on a zigzag edge.
There are two possible cases. For example, for 

\begin{equation*}
z_i = 0, y_i \neq N_y - 1, x_i \neq 0 ,
\end{equation*}
we have that the nearest neighbors of $i$ are $ j (i) = \{ j ( \bm r) \}$, with $\bm r$ in
\begin{equation*}
\bigg\{ \bm r_j \in \mathcal{L} \bigg| z_j = 1 \,\land\, \bigg[ \bigg( y_j = y_i  \,\land\, ( x_j = x_i \,\lor\, x_j = x_i - 1) \bigg) \,\lor\, \bigg( y_j = y_i + 1  \,\land\, x_j = x_i - 1  \bigg)  \bigg] \bigg\}
\end{equation*}

As opposed to the sites of a honeycomb lattice with \acp{PBC}, which have 3 neighbors, the sites of the zigzag edges have only 2 neighbors.
We summarize all possible cases in the following table.

\begin{table}[H]
\centering
	\caption{Nearest neighbors on the graphene nanoribbon.
	The neighbors in red are only for sites that are not on the edges.}
	\begin{tabular}{|c|c|c|c|} \hline
	\multicolumn{4}{|c|}{\textbf{\acp{OBC} \color{silver}{(\acp{PBC})} }}							\\ \hline
		Case 				& $z_j$	& $y_j$	& $x_j$ 	\\ \hline
		\multicolumn{1}{|c|}{\multirow{3}{*}{$z_i = 0$}}	 &	\multicolumn{1}{c|}{\multirow{3}{*}{1}} & \multicolumn{1}{c|}{\multirow{2}{*}{$y_i$}} & $x_i$   \\ \cline{4-4}
	   	\multicolumn{1}{|c|}{}	& \multicolumn{1}{c|}{\multirow{3}{*}{}} & \multicolumn{1}{c|}{\multirow{2}{*}{}}& \multicolumn{1}{c|}{\multirow{2}{*}{$N_x - 1 - (N_x - x_i) \% N_x$}} \\ \cline{3-3}
	   	\multicolumn{1}{|c|}{}	& \multicolumn{1}{c|}{} & \color{silver}{$y_i +1$} & \multicolumn{1}{c|}{\multirow{2}{*}{}} \\ \hline
%		\multicolumn{1}{|c|}{\multirow{3}{*}{$z_i = 0, y_i \neq N_y - 1, x_i = 0$}}	 &	\multicolumn{1}{c|}{\multirow{3}{*}{1}} & \multicolumn{1}{c|}{\multirow{2}{*}{$y_i$}} & 0   \\ \cline{4-4}
%	   	\multicolumn{1}{|c|}{}	& \multicolumn{1}{c|}{\multirow{3}{*}{}} & \multicolumn{1}{c|}{\multirow{2}{*}{}}& \multicolumn{1}{c|}{\multirow{2}{*}{$N_x - 1$}} \\ \cline{3-3}
%	   	\multicolumn{1}{|c|}{}	& \multicolumn{1}{c|}{} & $y_i +1$ & \multicolumn{1}{c|}{\multirow{2}{*}{}} \\ \hline
		\multicolumn{1}{|c|}{\multirow{3}{*}{$z_i = 1$}}	 &	\multicolumn{1}{c|}{\multirow{3}{*}{0}} & \multicolumn{1}{c|}{\multirow{2}{*}{$y_i$}} & $x_i$   \\ \cline{4-4}
	   	\multicolumn{1}{|c|}{}	& \multicolumn{1}{c|}{\multirow{3}{*}{}} & \multicolumn{1}{c|}{\multirow{2}{*}{}}& \multicolumn{1}{c|}{\multirow{2}{*}{$(x_i + 1) \% N_x$}} \\ \cline{3-3}
	   	\multicolumn{1}{|c|}{}	& \multicolumn{1}{c|}{} & \color{silver}{$y_i -1$} & \multicolumn{1}{c|}{\multirow{2}{*}{}} \\ \hline
%		\multicolumn{1}{|c|}{\multirow{3}{*}{$z_i = 1$, $y_i \neq 0$, $x_i = N_x - 1$}}	 &	\multicolumn{1}{c|}{\multirow{3}{*}{0}} & \multicolumn{1}{c|}{\multirow{2}{*}{$y_i$}} & $N_x - 1$   \\ \cline{4-4}
%	   	\multicolumn{1}{|c|}{}	& \multicolumn{1}{c|}{\multirow{3}{*}{}} & \multicolumn{1}{c|}{\multirow{2}{*}{}}& \multicolumn{1}{c|}{\multirow{2}{*}{0}} \\ \cline{3-3}
%	   	\multicolumn{1}{|c|}{}	& \multicolumn{1}{c|}{} & $y_i -1$ & \multicolumn{1}{c|}{\multirow{2}{*}{}} \\ \hline
	\end{tabular}
%	\hspace{-0.1cm}
%	\begin{tabular}{|c|c|c|c|} \hline
%	\multicolumn{4}{|c|}{\textbf{\acp{OBC}}}							\\ \hline
%		Case 				& $z_j$	& $y_j$	& $x_j$ 	\\ \hline
%	\multicolumn{1}{|c|}{\multirow{2}{*}{$z_i = 0, y_i = N_y - 1$}}	 &	\multicolumn{1}{c|}{\multirow{2}{*}{1}} & \multicolumn{1}{c|}{\multirow{2}{*}{$y_i$}} & $x_i$   \\ \cline{4-4}
%	   	\multicolumn{1}{|c|}{}	& \multicolumn{1}{c|}{\multirow{2}{*}{}}  & \multicolumn{1}{c|}{}& $N_x - 1 - (N_x - x_i) \% N_x$ \\ \hline
%	   	\multicolumn{1}{|c|}{\multirow{2}{*}{$z_i = 0, y_i = N_y - 1, x_i = 0$}}	 &	\multicolumn{1}{c|}{\multirow{2}{*}{1}} & \multicolumn{1}{c|}{\multirow{2}{*}{$y_i$}} & 0   \\ \cline{4-4}
%	   	\multicolumn{1}{|c|}{}	& \multicolumn{1}{c|}{\multirow{2}{*}{}}  & \multicolumn{1}{c|}{}& $N_x - 1$ \\ \hline
%	   	\multicolumn{1}{|c|}{\multirow{2}{*}{$z_i = 1, y_i = 0$}}	 &	\multicolumn{1}{c|}{\multirow{2}{*}{0}} & \multicolumn{1}{c|}{\multirow{2}{*}{$y_i$}} & $x_i$   \\ \cline{4-4}
%	   	\multicolumn{1}{|c|}{}	& \multicolumn{1}{c|}{\multirow{2}{*}{}}  & \multicolumn{1}{c|}{}& $(x_i + 1) \% N_x$  \\ \hline
%	   	\multicolumn{1}{|c|}{\multirow{2}{*}{$z_i = 1, y_i = 0, x_i = N_x - 1$}}	 &	\multicolumn{1}{c|}{\multirow{2}{*}{0}} & \multicolumn{1}{c|}{\multirow{2}{*}{$y_i$}} & 0   \\ \cline{4-4}
%	   	\multicolumn{1}{|c|}{}	& \multicolumn{1}{c|}{\multirow{2}{*}{}}  & \multicolumn{1}{c|}{}& $N_x - 1$ \\ \hline
%	\end{tabular}
	\label{tab:dummytable}
\end{table}

\subsection{\acp{TMD}}
\label{subsec:apTMD}

\begin{figure}[H]
\includegraphics[scale=0.8]{Applications/fillingVsE.png}
\includegraphics[scale=0.8]{Applications/fillingVsEnanoribbon.png}
	\caption[Filling factor as a function of the Fermi energy for \acs{TMD} monolayers and nanoribbons.]{Left: Filling factor as a function of the Fermi energy for a system with \acp{PBC}, as computed by diagonalizing the input matrix of our code, and by the hopping matrix in $\bm k$-space.
	Right: Comparison between the filling factors of a nanoribbon and a periodic system as a function of the Fermi energy, as computed by diagonalizing our input matrix. }
	\label{fig:fillingVsE}
\end{figure}