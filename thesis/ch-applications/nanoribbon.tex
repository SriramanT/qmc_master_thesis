\section{Nanoribbons}
\label{sec:nanoribbon}

In this section, we apply our code to the case of the honeycomb lattice of graphene and of the triangular lattice with nonuniform hoppings considered in the 3-band minimal model of \acs{TMD}s with intra-orbital on-site interactions.
We take boundary conditions corresponding to nanoribbons.
These structures are much longer on one direction than on the other, , i.e. $l \gg w$, resembling a ribbon, hence their name (see Fig.(\ref{fig:bcRibbon}))\footnote{In Fig.(\ref{fig:bcRibbon}), this condition is, of course, not obeyed solely for the sake of giving a good visual representation of the boundary conditions, and the numbering system).}.
Along the $x$-direction, a ribbon is normally very long, which justifies the fact that we take \acp{PBC}.
In contrast, in the narrow $y$-direction we take \acp{OBC}.
The presence of low energy electronic states on the edges of these ribbons might lead to nontrivial magnetic behavior in \ac{TMD} nanostructures (as indeed they do in graphene nanostructures \cite{yazyev_emergence_2010}) and it is this possibility was unexplored numerically before this work \cite{feldner_dynamical_2011, golor_quantum_2013}, as was mentioned in chapter \ref{cap:int}.

\subsection{Graphene}
\label{sec:graphene}

We use three coordinates to label each site on the honeycomb lattice, by taking advantage of its bipartite nature.
Regarding the honeycomb lattice as two interpenetrating triangular sublattices $\mathcal{A}$ and $\mathcal{B}$, we take the axes $x$ and $y$ to be along the primitive vectors of each triangular sublattice.
To number the sites on the ribbon, we introduce an additional coordinate labeling the sublattice: $z = 0$, if the site is in sublattice $\mathcal{A}$, and $z = 1$ if the site is in sublattice $\mathcal{B}$.
We then adopt the numbering convention for the sites $i = 0,1, ..., 2 N_x N_y - 1$ of the lattice $\mathcal{L}$:
$
i (x, y, z) = N_x N_y z + N_x y + x,
$
 where $x = 0, ..., N_x - 1$, $y = 0, ..., N_y - 1$, and $z = 0, 1$ define each element $\bm r = (x, y, z) \in \mathcal{L}$.

The geometry of the system appears through the hopping matrix $\bm K$ in our code.
This numbering system makes it straightforward to find the neighbors of each site.
Let us begin by considering a site that is not on a zigzag edge.
There are two possible cases. For example, for 
$
z_i = 0, y_i \neq N_y - 1, x_i \neq 0 ,
$
 we have that the nearest neighbors of $i$ are $ j (i) = \{ j ( \bm r) \}$, with $\bm r$ in
\begin{equation*}
\bigg\{ \bm r_j \in \mathcal{L} \bigg| z_j = 1 \,\land\, \bigg[ \bigg( y_j = y_i  \,\land\, ( x_j = x_i \,\lor\, x_j = x_i - 1) \bigg) \,\lor\, \bigg( y_j = y_i + 1  \,\land\, x_j = x_i - 1  \bigg)  \bigg] \bigg\}
\end{equation*}

As opposed to the sites of a honeycomb lattice with \acp{PBC}, which have 3 neighbors, the sites of the zigzag edges have only 2 neighbors.
We summarize all possible cases in the following table.

\begin{table}[H]
\centering
	\caption{Nearest neighbors on the graphene nanoribbon.
	The neighbors in gray are only for sites that are not on the edges.
	$\%$ refers to the remainder of integer division.}
	\begin{tabular}{|c|c|c|c|} \hline
	\multicolumn{4}{|c|}{\textbf{\acp{OBC} \color{silver}{(\acp{PBC})} }}							\\ \hline
		Case 				& $z_j$	& $y_j$	& $x_j$ 	\\ \hline
		\multicolumn{1}{|c|}{\multirow{3}{*}{$z_i = 0$}}	 &	\multicolumn{1}{c|}{\multirow{3}{*}{1}} & \multicolumn{1}{c|}{\multirow{2}{*}{$y_i$}} & $x_i$   \\ \cline{4-4}
	   	\multicolumn{1}{|c|}{}	& \multicolumn{1}{c|}{\multirow{3}{*}{}} & \multicolumn{1}{c|}{\multirow{2}{*}{}}& \multicolumn{1}{c|}{\multirow{2}{*}{$N_x - 1 - (N_x - x_i) \% N_x$}} \\ \cline{3-3}
	   	\multicolumn{1}{|c|}{}	& \multicolumn{1}{c|}{} & \color{silver}{$y_i +1$} & \multicolumn{1}{c|}{\multirow{2}{*}{}} \\ \hline
		\multicolumn{1}{|c|}{\multirow{3}{*}{$z_i = 1$}}	 &	\multicolumn{1}{c|}{\multirow{3}{*}{0}} & \multicolumn{1}{c|}{\multirow{2}{*}{$y_i$}} & $x_i$   \\ \cline{4-4}
	   	\multicolumn{1}{|c|}{}	& \multicolumn{1}{c|}{\multirow{3}{*}{}} & \multicolumn{1}{c|}{\multirow{2}{*}{}}& \multicolumn{1}{c|}{\multirow{2}{*}{$(x_i + 1) \% N_x$}} \\ \cline{3-3}
	   	\multicolumn{1}{|c|}{}	& \multicolumn{1}{c|}{} & \color{silver}{$y_i -1$} & \multicolumn{1}{c|}{\multirow{2}{*}{}} \\ \hline
	\end{tabular}
	\label{tab:dummytable}
\end{table}

Recall the mean field result obtained for a graphene nanoribbon we presented in chapter \ref{cap:int}.
For each sublattice, ferromagnetic order is induced by the on-site interaction along each row of the ribbon, i.e. along the longitudinal direction $x$.
The average spin density has its maximum on the edge, decaying in the bulk.
It reaches another (smaller) maximum on the other side of the ribbon, i.e., in the row next to the other sublattice's edge. (see Fig.(\ref{fig:nanoGraphVsTMD})).
This type of globally (considering the two sublattices) antiferromagnetic ordering is confirmed by \ac{QMC} \cite{feldner_dynamical_2011, raczkowski_interplay_2017}.
Other recent auxiliary field \ac{QMC} studies point at the possibility of strain-tuning this type of edge-magnetism in zig-zag graphene nanoribbons \cite{yang_strain-tuning_2017}.
The authors start by introducing a reduction in the hopping along $x$ to model strain, i.e. changing the hopping between atoms connected by red bonds in Fig.(\ref{fig:bcRibbon}): $-t \mapsto -t + \Delta$.
Then, they identify phase transitions to edge-magnetized states for relatively small values of the hopping reduction ($\Delta = 0-0.5t$) and on-site interaction ($U = 1 t- 4t$).
To do so, they fit the magnetic susceptibility at the edge $\chi_e$ to the Curie-Weiss law:
\begin{equation}
\frac{1}{\chi_e} = \frac{T - T_c}{A} ,
\end{equation}
and identify the critical temperature.
By repeating this procedure for different simulations, one can find critical values of $U_c$ and $\Delta$ for the transition to occur, and find the critical temperatures of the transitions for different values of these parameters.

\begin{figure}[H]
\hspace{-0.6cm}
\includegraphics[trim={0 8cm 0 2.5cm},clip, scale = 0.27]{Applications/nanoribbon.pdf}
\includegraphics[scale = 0.49]{Applications/graphene-strain/CorrelationsDots.png}
	\caption[Boundary conditions on the nanoribbon. Spin-spin correlations of a strained zig-zag graphene nanoribbon.]{Left: Boundary conditions on the nanoribbon for $N_x = 5, \, N_y = 4$. The orange circles correspond to sublattice $\mathcal{A}$, and the blue circles correspond to sublattice $\mathcal{B}$.
	Spin-spin correlations of a strained zig-zag graphene nanoribbon, with reduced hopping along $x$: $t \mapsto t - \Delta$, for $\Delta = 0.3t$. The label $\bm 0$ refers to the lower left site.
}
	\label{fig:bcRibbon}
\end{figure}

A mean field treatment tends to overestimate long range order.
In this case, it predicts spin correlations to decay rapidly from the edge rows inward, and then to remain constant along the $x$ direction.
The result of Fig.(\ref{fig:bcRibbon}) shows us that it decays as we penetrate the sample, and then increases again, with the profile shown in Fig.(\ref{fig:longProf}).

A peak at $\bm q = \bm 0$ in the magnetic structure factor $S ( \bm q )$ suggests sublattice ferromagnetic ordering, and indeed, by studying the magnetic structure factor of each row separately, we see that it is enhanced at $\bm q = \bm 0$ on the edges compared to the bulk.
Moreover, we study the behavior of $M_{\text{row}} \equiv \frac{1}{N_x}\sqrt{ \sum_{(i, j) \in \text{row}} \left\langle S_i S_j \right\rangle}$, and we conclude that it is maximum in one of the edge rows, decreases as we penetrate the bulk, and then increases again, having a smaller local maximum on the other edge row.
Of course, since the system size is small, the effect is much more pronounced on one of the edges than the other, because the decay of the average spin density from the edge to the bulk is relatively slow.
If we had more CPU time available, we could do larger scale simulations to see a more dramatic effect on \emph{both} edges.

\begin{figure}[H]
\includegraphics[scale = 0.56]{Applications/graphene-strain/S(q)pcolor.png}
\hspace{-0.7cm}
\includegraphics[scale = 0.55]{Applications/graphene-strain/edgeStFactor_Graphene.png}
	\caption[Magnetic structure factor $S(\bm q)$ for a strained graphene nanoribbon: bulk and edges.]{Magnetic structure factor $S(\bm q)$ for a strained graphene nanoribbon: bulk (left) and edges compared with a row of the bulk (right).}
	\label{fig:edgeStFactor}
\end{figure}
\begin{figure}[H]
\includegraphics[scale = 0.55]{Applications/graphene-strain/LongitudinalProfile.png}
\includegraphics[scale = 0.55]{Applications/graphene-strain/order_parameter_Graphene.png}
	\caption[Spin-spin correlation function profile along the ribbon's ]{}
	\label{fig:longProf}
\end{figure}
\begin{figure}[H]
\hspace{-0.2cm}
\includegraphics[scale = 0.55]{Applications/graphene-strain/chiTyang2017.png}
\includegraphics[scale = 0.55]{Applications/graphene-strain/fityang2017.png}
	\caption[]{}
	\label{fig:chiFit}
\end{figure}

\subsection{\acp{TMD}}
\label{subsec:apTMD}

We start by approaching the \acs{TMDNR} problem at the mean field level.
Such a procedure is very useful to obtain a physical picture of the system's behavior.
In particular, at a given temperature, if there is a transition between a configuration with magnetic order, and a disordered one, there is a critical interaction parameter $U = U_c$ at which the transition occurs, and it can be estimated in mean field, and compared with the more precise, unbiased \acs{QMC} result.
In general, our mean field formulation would involve diagonalizing an $N \times N$ matrix at each step, where $N = N_{\text{orb}} N_x N_y$ is the size of the system times the number of orbitals.
However, since we consider \acs{PBC}s along the $x$-direction, we can partially diagonalize  the Hamiltonian analytically, reducing the size of the matrix to be diagonalized to $N_{\text{orb}} N_y \times N_{\text{orb}} N_y$, where $N_y$ is the width of the ribbon, i.e. the number of $\text{M}\text{X}_2$ formula units.
Consider the spinless 3-band tight binding model, with the lattice constant: $a \equiv 1$.
\begin{equation}
\begin{split}
\mathcal{H}_0 &= \sum_{\substack{m, n \\ \alpha, \beta}} \bigg( c_{m,n, \alpha}^\dagger t_{\alpha\beta}^0 c_{m, n, \beta} + \delta_{0, N_x}  c_{m,n, \alpha}^\dagger t_{\alpha\beta}^1 c_{m+1, n, \beta} + \delta_{-\sqrt{3}/2, (N_y -1)\sqrt{3}/2}  c_{m,n, \alpha}^\dagger t_{\alpha\beta}^4 c_{m-1, n, \beta} \\
& + \delta_{0, N_m} \delta_{-1, (N_y -1)\sqrt{3}/2} c_{m+1/2,n-\sqrt{3}/2, \alpha}^\dagger t_{\alpha\beta}^2 c_{m, n, \beta} + \delta_{-1, (N_y -1)\sqrt{3}/2} c_{m-1/2,n-\sqrt{3}/2, \alpha}^\dagger t_{\alpha\beta}^3 c_{m, n, \beta} \\
& + \delta_{N_y\sqrt{3}/2, 0} c_{m+1/2,n+\sqrt{3}/2, \alpha}^\dagger t_{\alpha\beta}^6 c_{m, n, \beta} + \delta_{-1, N_x -1} \delta_{N_y\sqrt{3}/2, 0} c_{m-1/2,n+\sqrt{3}/2, \alpha}^\dagger t_{\alpha\beta}^5 c_{m, n, \beta} \bigg)
\end{split}
\end{equation}

Fourier transforming along $m$: $c_{m, n, \alpha} = \frac{1}{\sqrt{N_x}}\sum_{k} e^{-i k m } c_{k, n, \alpha} $, with $k = \frac{2\pi}{N_x} \{ -\frac{N_x}{2} + 1, -\frac{N_x}{2}, ..., \frac{N_x}{2} \}$:
\begin{equation}
\begin{split}
\mathcal{H}_0 &= \sum_{ \substack{k, y \\ \alpha, \beta} } \bigg( c_{k, y, \alpha}^\dagger (t_{\alpha \beta}^0  + e^{ik} t_{\alpha \beta}^1 + e^{-ik} t_{\alpha \beta}^4 )  c_{k, y, \beta} + \delta_{-1, N_y -1} c_{k, y - 1, \alpha}^\dagger ( e^{ik/2} t_{\alpha \beta}^2 + e^{-ik/2} t_{\alpha \beta}^3 ) c_{k, y, \beta} \\
& + \delta_{N_y, 0} c_{k, y, \alpha}^\dagger ( e^{ik/2} t_{\alpha \beta}^6 + e^{-ik/2} t_{\alpha \beta}^5 ) c_{k, y+1, \beta} \bigg) , \,\, \text{with} \, y \, \text{defined as in Fig.(4.1)}
\end{split}
\end{equation}
leading to a tridiagonal block $3 N_y \times 3 N_y$ hopping matrix $\bm H (k)$ with three different types of matrix elements: $\bm h_1 = \bm H_{y,y}$, $\bm h_2 = \bm H_{y,y-1}$, $\bm h_2^\dagger = \bm H_{y, y+1}$.
\begin{equation}
[ H_{(\alpha y) (\beta y')} (k) ] = 
\begin{pmatrix}
\bm h_1 & \bm h_2^\dagger & & & \\
\bm h_2 & \bm h_1 & \bm h_2^\dagger & & \\
& \bm h_2 & \bm h_1 & \ddots & \\
& & \ddots & \ddots & \bm h_2^\dagger \\
& & & \bm h_2 & \bm h_1
\end{pmatrix}, \, 
\bm h_1 = 
\begin{pmatrix}
\varepsilon_1 + 2 t_0 \cos k & 2 i t_1 \sin k & 2 t_2 \cos k \\
-2 i t_1 \sin k & \varepsilon_2 + 2 t_{11} \cos k & 2 i t_{12} \sin k \\
2 t_2 \cos k& -2 i t_{12} \sin k & \varepsilon_2 + 2 t_{22} \cos k \\
\end{pmatrix}
\end{equation}
\begin{equation*}
\bm h_2 =
\begin{pmatrix}
2 t_0 \cos ( k / 2 ) & i \sin ( k / 2 ) \bigg( t_1 - \sqrt{3} t_2 \bigg) & - \cos (k /2 ) \bigg( \sqrt{3} t_1 + t_2 \bigg) \\
-i \sin ( k / 2 ) \bigg(t_1 + \sqrt{3} t_2 \bigg) & \frac{1}{2} \cos (k / 2) \bigg( t_{11} + 3 t_{22} \bigg) & -i \sin (k / 2) \bigg( \frac{\sqrt{3}}{2} (t_{11} -  t_{22} ) + 2 t_{12} \bigg) \\
\cos ( k / 2) \bigg( \sqrt{3} t_1 - t_2 \bigg) & -i \sin (k / 2) \bigg( \frac{\sqrt{3}}{2} ( t_{11} - t_{22} ) - 2 t_{12} \bigg) & \frac{1}{2} \cos (k / 2) \bigg( 3 t_{11	} + t_{22} \bigg)
\end{pmatrix}
\end{equation*}
\begin{figure}[H]
\includegraphics[scale=0.55]{Applications/fillingVsE.png}
\includegraphics[scale=0.55]{Applications/BandStructureNanoribbonTMD.png}
	\caption[Filling $\nu$ as a function of the Fermi energy $\varepsilon_F$ for a \acs{TMD} monolayer and a nanoribbon. $\text{Mo}\text{S}_2$.  \acs{TMDNR} band structure obtained by the 3-band model.]{Left: Filling $\nu$ as a function of the Fermi energy $\varepsilon_F$ for a system with \acp{PBC}, as computed by diagonalizing the input matrix of our code, and by the hopping matrix in $\bm k$-space. Comparison between the fillings of a nanoribbon and a periodic system.
	For the nanoribbon, edge states appear on the gap of the periodic system.
	Right: 3-band model $\text{Mo}\text{S}_2$ zigzag edged nanoribbon energy bands (red dots and orange curve), for $N_y = 8$, using the GCA parameters. The first principles bands show the contribution from orbitals that are not considered in the 3-band model (blue:  $d_{z^2}$, $d_{xy}$, $d_{x^2-y^2}$, green: others). The 3-band model reproduces the bands that correspond to the orbitals taken into account reasonably (1 and 2 correspond to the edge states from the $d_{z^2}$, $d_{xy}$, $d_{x^2-y^2}$ orbitals of the $\text{Mo}$ atoms, while 3 and 4 correspond to the $d_{yz}$ orbital at the $\text{Mo}$-terminated edge, and $p_{y, z}$ orbitals from the $\text{S}$-terminated edge, and are not taken into account in the 3-band model).\cite{liu_three-band_2013}.}
	\label{fig:fillingVsE}
\end{figure}

By applying our mean field approach to solve the 3-band model with Hubbard-type interactions, we obtain solutions that are independent of $x$, which motivates us to reduce the number of MF parameters by choosing a translationally invariant ansatz.
This is equivalent to taking $\left\langle n_{x, y,\alpha, \sigma}\right\rangle = \left\langle n_{y,\alpha, \sigma}\right\rangle  \forall x$ ($6 N_y$ parameters).
By reducing the number of parameters, convergence is facilitated, which allows us to evaluate whether the solution of Fig.(\ref{fig:nanoGraphVsTMD}) is robust, i.e. whether it corresponds to a metastable or not.
The mean field form of the interaction term with the reduced number of parameters changes, implying that the self-consistent relation of Eq.(\ref{eq:selfConsistent}) from which the local densities are computed changes as well.
\begin{equation}
\mathcal{H}_{\text{MF}} = \mathcal{H}_0 + \mathcal{H}_1 + \mathcal{C} , \,\text{where} \,\, \mathcal{H}_1 = U \sum_{m, n, \alpha}  \sum_\sigma n_{\substack{m, \sigma \\ n, \alpha}} \big\langle n_{\substack{m, -\sigma \\ n, \alpha}} \big\rangle  , \,\, \mathcal{C} = -U  \sum_{m, n, \alpha} \big\langle n_{\substack{m, \uparrow \\ n, \alpha}} \big\rangle \big\langle n_{\substack{m, \downarrow \\ n, \alpha}} \big\rangle
\end{equation}
\begin{equation}
\begin{split}
\mathcal{H}_1 + \mathcal{C} &= \frac{U}{N_x^{\,2}} \sum_{\substack{n \alpha \\ k_1 k_2 \\ k_3 k_4}} \underbrace{\sum_m e^{i [ (k_1 + k_3) - (k_2 + k_4) ] m}}_{N_x \delta_{k_4, k_1 + k_3 - k_2}} \bigg( \sum_\sigma c_{\substack{k_1, \sigma \\ n, \alpha}}^\dagger c_{\substack{k_2, \sigma \\ n, \alpha}} \underbrace{\big\langle c_{\substack{k_3, -\sigma\\ n, \alpha}}^\dagger c_{\substack{k_4, -\sigma \\ n, \alpha}} \big\rangle}_{\delta_{k_3, k_4} \big\langle n_{\substack{k_3,-\sigma \\ n, \alpha}} \big\rangle} - \underbrace{\big\langle c_{\substack{k_1, \uparrow \\ n, \alpha}}^\dagger c_{\substack{k_2  \uparrow \\n, \alpha}} \big\rangle}_{\delta_{k_1, k_2} \big\langle n_{\substack{k_2,\uparrow \\ n, \alpha}} \big\rangle} \underbrace{\big\langle c_{\substack{k_3, \downarrow \\ n, \alpha}}^\dagger c_{\substack{k_4, \downarrow \\ n, \alpha}} \big\rangle}_{\delta_{k_3, k_4} \big\langle n_{\substack{k_3,\downarrow \\ n, \alpha}} \big\rangle}  \bigg) \\
&= \frac{U}{N_x} \sum_{\substack{n \alpha \\ k_2 k_3}} \bigg( \sum_\sigma n_{\substack{k_2, \sigma \\ n, \alpha}} \big\langle n_{\substack{k_3, -\sigma \\ n, \alpha}} \big\rangle - \big\langle n_{\substack{k_2, \uparrow \\ n, \alpha}} \big\rangle \big\langle n_{\substack{k_3, \downarrow \\ n, \alpha}} \big\rangle \bigg) \equiv
U \sum_{k, \mu} \bigg( \sum_\sigma n_{k,\mu, \sigma} \big\langle n_{\mu, -\sigma} \big\rangle - \big\langle n_{\mu, \uparrow} \big\rangle \big\langle n_{\mu, \downarrow} \big\rangle \bigg)
\end{split}
\end{equation}
where we collapsed the indexes $(n, \alpha)$ into a single index $\mu$.
The self consistent relation allowing us to compute the new MF parameters at each step emerges by diagonalizing $\mathcal{H}_1$ in the $\mu$-subspace:
\begin{equation}
\big\langle n_{\mu, \sigma} \big\rangle = \frac{1}{N_x}\sum_{q, \nu} | Q_{q \sigma \mu, \nu} |^2 \rho ( \varepsilon_{q \nu \sigma} ) , \, \text{where} \,\, d_{q, \sigma, \nu} = \sum_\nu Q_{q \sigma \mu, \nu}^\star c_{q ,\sigma, \mu} ,  \, \text{and} \,\, \mathcal{H}_{\text{MF}} = \sum_{q, \nu, \sigma} \varepsilon_{q, \nu, \sigma} d_{q, \nu, \sigma}^\dagger d_{q, \nu, \sigma} + \mathcal{C}
\end{equation}
\begin{figure}[H]
\hspace{0.1cm}
\includegraphics[scale=1.023]{Applications/MFnanoribbon.png}
\hspace{0.05cm}
\includegraphics[scale=0.55]{Applications/lattice_Nx=512_Ny=16_U=20_beta=100.png}
\hspace{1cm}
\includegraphics[scale=0.462]{Applications/magProf.png}
	\caption[Comparison between the MF solutions of the Hubbard model for a graphene nanoribbon (GNR) and a \acs{TMDNR}. Spin density profile along the ribbon's transverse direction.]{Left: Comparison between the MF solutions of the Hubbard model at half filling for a graphene nanoribbon (GNR) at $U=1.2$ with $\beta t = 20$ (left) and a \acs{TMDNR}.
	Right: Comparison between the spin density profile along the ribbon's transverse direction $\left\langle m \right\rangle (y)$ (the obtained solution is constant along $x$) for the GNR and the \acs{TMDNR}).}
	\label{fig:nanoGraphVsTMD}
\end{figure}
\begin{figure}[H]
\centering
\includegraphics[trim={0cm 1.5cm 0cm 1.5cm},clip, scale =0.23]{Applications/tmd-mf/edge-mag.pdf}
	\caption[]{}
	\label{fig:zeroTphaseDiagram}
\end{figure}
\begin{figure}[H]
\centering
\includegraphics[trim={0cm 1.5cm 0cm 1.5cm},clip, scale =0.3]{Applications/tmd-mf/bandsZoomed.pdf}
	\caption[]{}
	\label{fig:bandsZoomed}
\end{figure}
\begin{figure}[H]
\centering
\includegraphics[scale=0.4]{Applications/tmd-mf/superposedBands05} \\
\includegraphics[scale=0.4]{Applications/tmd-mf/superposedBands075} \\
\includegraphics[scale=0.4]{Applications/tmd-mf/superposedBands4} \\
\includegraphics[scale=0.4]{Applications/tmd-mf/superposedBands100} \\
	\caption[]{}
	\label{fig:band-structures}
\end{figure}
\begin{figure}[H]
\hspace{1cm}
\includegraphics[scale=0.5]{Applications/tmd-mf/wfAroundK1}
\includegraphics[scale=0.5]{Applications/tmd-mf/magProfU14}
	\caption[]{}
	\label{fig:wfs}
\end{figure}
\begin{figure}[H]
\hspace{1.2cm}
\includegraphics[scale=0.35]{Applications/tmd-mf/bands152.png}
\hspace{4mm}
\includegraphics[scale=0.35]{Applications/tmd-mf/bands154.png}
	\caption[]{}
	\label{fig:wfs}
\end{figure}
\begin{figure}[H]
\hspace{1cm}
\includegraphics[scale=0.51]{Applications/tmd-mf/wfAroundK2}
\includegraphics[scale=0.5]{Applications/tmd-mf/magProfU154}
	\caption[]{}
	\label{fig:wfs}
\end{figure}
\begin{figure}[H]
\centering
\includegraphics[scale=0.62]{Applications/tmd-mf/edge-mag-phase-diagram}
	\caption[]{}
	\label{fig:wfs}
\end{figure}