% %%%%%%%%%%%%%%%%%%%%%%%%%%%%%%%%%%%%%%%%%%%%%%%%%%%%%%%%%%%%%%%%%%%%%%
% Dummy Chapter:
% %%%%%%%%%%%%%%%%%%%%%%%%%%%%%%%%%%%%%%%%%%%%%%%%%%%%%%%%%%%%%%%%%%%%%%

% %%%%%%%%%%%%%%%%%%%%%%%%%%%%%%%%%%%%%%%%%%%%%%%%%%%%%%%%%%%%%%%%%%%%%%
% The Introduction:
% %%%%%%%%%%%%%%%%%%%%%%%%%%%%%%%%%%%%%%%%%%%%%%%%%%%%%%%%%%%%%%%%%%%%%%
\fancychapter{Applications and original results}
\label{cap:applications}

\slshape

We simulate the Hubbard model on the \acs{1D} chain and on the \acs{2D} square lattice, for which results are well established from either an analytical or numerical perspective.
Then, we consider the honeycomb lattice with  nanoribbon boundary conditions, and reproduce some results in the literature.
We compare the obtained results and the run time with \texttt{QUEST}, the flagship implementation of auxiliary field \acs{QMC}.
Finally, we carry out original calculations, applying our code to a minimal model of \acs{TMD} nanoribbons.
We compare our \acs{QMC} results with those obtained by us in the mean field (MF)  approximation.
The latter is formulated allowing a site and orbital-dependent mean field, and the self-consistency relation is solved iteratively.
We characterize the convergence issues arising in this iterative procedure.
Our original MF results suggest that in the model we consider \acs{TMD} nanoribbons could host edge-magnetism at finite temperature.
The obtained phase diagram shows transitions between non-magnetized and two distinct edge-magnetized configurations.
We use our unbiased \acs{QMC} simulations to further probe the system, searching for these or other eventual phase transitions not captured by mean field theory.

\normalfont

\section{One-dimensional Chain}
\label{sec:1d-chain}

The analytical solution of the $T= 0$ \acs{1D} Hubbard model indicates that there is no Mott transition \cite{lieb_absence_1968}.
Consider the $frac{U}{t} \gg 1$ limit, where the Hubbard model becomes an atomic Heisenberg model defined in the Hilbert subspace with one electron per site, and antiferromagnetic order sets in.
It is found that upon decreasing $U$, the system remains an antiferromagnetic insulator down to $U \rightarrow 0$, and becomes a conductor only at $U = 0$.
Thus, we expect antiferromagnetic order to set in for all $U$ for high enough $\beta$.
We identify it by studying the spin-spin correlator $\left\langle S^z_i S^z_j \right\rangle$.
Fourier transforming, we obtain a peak at $q = \pi$ in the magnetic structure factor $S ( q ) $, and in the magnetic susceptibility $\chi (q)$.
\section{Square lattice}
\label{sec:square}

Similarly to what happens for the \acs{1D} case, mean field results on the square lattice suggest that at half filling, antiferromagnetic order persists in the ground state for any value of the on-site interaction \cite{claveau_mean-field_2014, gouveia_magnetic_2015}, a result that we confirm with \acs{QMC}, reproducing the results of \cite{white_numerical_1989, hirsch_two-dimensional_1985}.

\begin{figure}[H]
\label{fig:mfHubbardPhaseDiagram}
\hspace{-0.18cm}
\includegraphics[scale=0.245]{Applications/mf-phase-diagram-hubbard}
\includegraphics[scale=0.493]{Applications/square/nUpnDwSquare}
\caption[Mean field phase diagram of the Hubbard model. \ac{QMC} data showing the decrease of the double occupancy with increasing $U$.]{Left: MF phase diagram of the Hubbard model \cite{gouveia_magnetic_2015}.
Right: \ac{QMC} data showing the decrease of the double occupancy with increasing $U$, reproducing the results of \cite{white_numerical_1989}.}
\end{figure}
Our results for a half-filled $4 \times 4$ lattice at $\beta = 16 t$ clearly show antiferromagnetic spin-spin correlations - Fig.(\ref{fig:corrSq},left), which results in a peak of the magnetic structure factor $S (\bm q)$ at $\bm q = (\pi, \pi)$ - Fig.(\ref{fig:corrSq},right).
\begin{figure}[H]
\hspace{-0.3cm}
\includegraphics[scale=0.55]{Applications/square/CorrelationsDots}
\hspace{0.4cm}
\includegraphics[scale=0.65]{Applications/square/S(q)}
\caption[Spin-spin correlations on the square lattice.
Magnetic structure factors showing a peak at $\bm q = \bm \pi$.]{Spin-spin correlations with respect to the point on the lower left corner of the lattice (which we label as $\bm 0$).
3D Plot of the magnetic structure factor $S ( \bm q)$ displaying the $\bm q = \bm \pi$ peak \label{fig:corrSq}.}
\end{figure}
The susceptibility $\chi ( \bm q) $ is also strongly peaked at $\bm q = \bm \pi$, again revealing long range \ac{AF} order.
\begin{figure}[H]
\hspace{0.8cm}
\includegraphics[scale=0.5]{Applications/square/S(q)pcolor.png}
\hspace{0.2cm}
\includegraphics[scale=0.5]{Applications/square/chi(q)pcolor.png}
\caption[Color map of the structure factor $S ( \bm q)$, and susceptibility $\chi ( \bm q)$, both showing peaks at $\bm q = \bm\pi$.]{Color map of the structure factor $S ( \bm q)$, and susceptibility $\chi ( \bm q)$, both showing peaks at $\bm q = \bm\pi$.\label{fig:Schi3d}}
\end{figure}
\vspace{-0.5cm}
We finish this section by characterizing \ac{AF} order for varying system size and temperature, reproducing one of the main results of the seminal \ac{QMC} study of White, Scalapino and Sugar \cite{white_numerical_1989}.
In this paper, the authors report that by $\beta = 20 t$, our finite temperature algorithm is already measuring ground state properties.
In fact, for $\beta = 20 t$ the peaks at $\bm \pi$ of $S ( \bm \pi )$ coincide with those obtained using the projective variant of auxiliary field \ac{QMC} mentioned in chapter \ref{cap:int}.
Extrapolating to the infinite system, we verify that long range order exists, i.e. the antiferromagnetic order parameter, or staggered magnetization is finite for $N \rightarrow \infty$.
Looking at Fig(\ref{fig:spipi}), it is clear that the zero temperature extrapolation of $\bm S (\pi)$ increases with lattice size.
In fact, as per \cite{hirsch_two-dimensional_1985}, for a sufficiently long lattice, at $T= 0$, we have that
\begin{equation}
S(\bm \pi) = N m_{\text{st}}^2 + S_c ( \bm \pi ) ,
\end{equation}
where $S_c ( \bm \pi )$ is the connected structure factor, obtained by replacing the average $\left\langle S^z_i S^z_j \right\rangle$ in the Fourier transform by $\left\langle S^z_i S^z_j \right\rangle - \left\langle S^z_i \right\rangle \left\langle S^z_j \right\rangle$, and $m_{\text{st}}$ is the \ac{AF} order parameter, or staggered magnetization:
\begin{equation}
m_{\text{st}} = \frac{1}{N} \sum_i (-1)^{R_i} \left\langle n_{i,\uparrow} - n_{i,\downarrow} \right\rangle
\end{equation}
Thus, to extrapolate the long range order, we fit $S ( \bm \pi ) / N$ by linear regression and obtain the estimate of the \ac{AF} order parameter as $m_{\text{st}} = 0.09$.
As we increase $U$, this value becomes larger.
According to \cite{hirsch_two-dimensional_1985} (and references therein), on a finite lattice, the \ac{AF} long range order is $50 \%$ reduced from the classical Néel state due to quantum fluctuations.
Thus, at $U \rightarrow \infty$, we obtain the maximum value $m_{\text{st}}^2 = 0.25$, and the estimate we get for $U=4$ amounts to $36\%$ of the maximum of $m_{\text{st}}^2$.
\vspace{-0.3cm}
\begin{figure}[H]
\hspace{-0.4cm}
\includegraphics[scale=0.55]{Applications/square/Spipi.png}
\includegraphics[scale=0.55]{Applications/square/extrapolationSpi.png}
\caption[$S ( \bm \pi ) $ for varying system size and inverse temperature.
Infinite system extrapolation of long range order.]{Left: $S ( \bm \pi ) $ for varying system size and inverse temperature.
Right: Infinite system extrapolation of long range order. \label{fig:spipi} }
\end{figure}
\section{Nanoribbons}
\label{sec:nanoribbon}

In this section, we apply our code to the case of honeycomb lattices with boundary conditions corresponding to nanoribbons. These structures are much longer on one direction than on the other, resembling a ribbon, hence their name.
The low energy electronic states on the edges of this ribbon might lead to interesting magnetic behavior in \ac{TMD} nanostructures (as indeed they do in graphene nanostructures \cite{yazyev_emergence_2010}) and it is this possibility was unexplored numerically before this work \cite{feldner_dynamical_2011, golor_quantum_2013}, as was mentioned on chapter \ref{cap:int}.
A nanoribbon is much longer on one direction than on the other, i.e. $l \gg w$. 
This condition corresponds to taking $N_x \gg N_y$ in our conventions (see Fig.(\ref{fig:nanoribbon}), where this condition is, of course, not obeyed solely for the sake of giving a good visual representation of the boundary conditions, and the numbering system).

\subsection{Graphene}
\label{sec:graphene}

We use three coordinates to label each site on the honeycomb lattice, by taking advantage of its bipartite nature.
Regarding the honeycomb lattice as two interpenetrating triangular sublattices $\mathcal{A}$ and $\mathcal{B}$, we take the axes $x$ and $y$ to be along the primitive vectors of each triangular sublattice.
Along the $x$-direction, the ribbon is supposed to be very long, which justifies the fact that we take \acp{PBC}.
In contrast, in the narrow $y$-direction we take \acp{OBC}.
To number the sites on the ribbon, we introduce an additional coordinate labeling the sublattice: $z = 0$, if the site is in sublattice $\mathcal{A}$, and $z = 1$ if the site is in sublattice $\mathcal{B}$.
We then adopt the numbering convention for the sites $i = 0,1, ..., 2 N_x N_y - 1$ of the lattice $\mathcal{L}$:

\begin{equation}\label{eq:numbering}
i (x, y, z) = N_x N_y z + N_x y + x,
\end{equation}
where $x = 0, ..., N_x - 1$, $y = 0, ..., N_y - 1$, and $z = 0, 1$ define each element $\bm r = (x, y, z) \in \mathcal{L}$.

\begin{figure}[H]
	\centering
\includegraphics[trim={0 8cm 0 2.5cm},clip, width=0.8\linewidth]{Applications/nanoribbon.pdf}
	\caption[Boundary conditions on the nanoribbon.]{Boundary conditions on the nanoribbon for $N_x = 5, \, N_y = 4$. The orange circles correspond to sublattice $\mathcal{A}$, and the blue circles correspond to sublattice $\mathcal{B}$.
	The counting starts at 1 (i.e. the numbers correspond to $i+1$, since $i$ starts at 0).}
	\label{fig:nanoribbon}
\end{figure}

The geometry of the system appears through the hopping matrix $\bm K$ in our code.
This numbering system makes it straightforward to find the neighbors of each site.
Let us begin by considering a site that is not on a zigzag edge.
There are two possible cases. For example, for 

\begin{equation*}
z_i = 0, y_i \neq N_y - 1, x_i \neq 0 ,
\end{equation*}
we have that the nearest neighbors of $i$ are $ j (i) = \{ j ( \bm r) \}$, with $\bm r$ in
\begin{equation*}
\bigg\{ \bm r_j \in \mathcal{L} \bigg| z_j = 1 \,\land\, \bigg[ \bigg( y_j = y_i  \,\land\, ( x_j = x_i \,\lor\, x_j = x_i - 1) \bigg) \,\lor\, \bigg( y_j = y_i + 1  \,\land\, x_j = x_i - 1  \bigg)  \bigg] \bigg\}
\end{equation*}

As opposed to the sites of a honeycomb lattice with \acp{PBC}, which have 3 neighbors, the sites of the zigzag edges have only 2 neighbors.
We summarize all possible cases in the following table.

\begin{table}[H]
\centering
	\caption{Nearest neighbors on the graphene nanoribbon.
	The neighbors in red are only for sites that are not on the edges.}
	\begin{tabular}{|c|c|c|c|} \hline
	\multicolumn{4}{|c|}{\textbf{\acp{OBC} \color{silver}{(\acp{PBC})} }}							\\ \hline
		Case 				& $z_j$	& $y_j$	& $x_j$ 	\\ \hline
		\multicolumn{1}{|c|}{\multirow{3}{*}{$z_i = 0$}}	 &	\multicolumn{1}{c|}{\multirow{3}{*}{1}} & \multicolumn{1}{c|}{\multirow{2}{*}{$y_i$}} & $x_i$   \\ \cline{4-4}
	   	\multicolumn{1}{|c|}{}	& \multicolumn{1}{c|}{\multirow{3}{*}{}} & \multicolumn{1}{c|}{\multirow{2}{*}{}}& \multicolumn{1}{c|}{\multirow{2}{*}{$N_x - 1 - (N_x - x_i) \% N_x$}} \\ \cline{3-3}
	   	\multicolumn{1}{|c|}{}	& \multicolumn{1}{c|}{} & \color{silver}{$y_i +1$} & \multicolumn{1}{c|}{\multirow{2}{*}{}} \\ \hline
%		\multicolumn{1}{|c|}{\multirow{3}{*}{$z_i = 0, y_i \neq N_y - 1, x_i = 0$}}	 &	\multicolumn{1}{c|}{\multirow{3}{*}{1}} & \multicolumn{1}{c|}{\multirow{2}{*}{$y_i$}} & 0   \\ \cline{4-4}
%	   	\multicolumn{1}{|c|}{}	& \multicolumn{1}{c|}{\multirow{3}{*}{}} & \multicolumn{1}{c|}{\multirow{2}{*}{}}& \multicolumn{1}{c|}{\multirow{2}{*}{$N_x - 1$}} \\ \cline{3-3}
%	   	\multicolumn{1}{|c|}{}	& \multicolumn{1}{c|}{} & $y_i +1$ & \multicolumn{1}{c|}{\multirow{2}{*}{}} \\ \hline
		\multicolumn{1}{|c|}{\multirow{3}{*}{$z_i = 1$}}	 &	\multicolumn{1}{c|}{\multirow{3}{*}{0}} & \multicolumn{1}{c|}{\multirow{2}{*}{$y_i$}} & $x_i$   \\ \cline{4-4}
	   	\multicolumn{1}{|c|}{}	& \multicolumn{1}{c|}{\multirow{3}{*}{}} & \multicolumn{1}{c|}{\multirow{2}{*}{}}& \multicolumn{1}{c|}{\multirow{2}{*}{$(x_i + 1) \% N_x$}} \\ \cline{3-3}
	   	\multicolumn{1}{|c|}{}	& \multicolumn{1}{c|}{} & \color{silver}{$y_i -1$} & \multicolumn{1}{c|}{\multirow{2}{*}{}} \\ \hline
%		\multicolumn{1}{|c|}{\multirow{3}{*}{$z_i = 1$, $y_i \neq 0$, $x_i = N_x - 1$}}	 &	\multicolumn{1}{c|}{\multirow{3}{*}{0}} & \multicolumn{1}{c|}{\multirow{2}{*}{$y_i$}} & $N_x - 1$   \\ \cline{4-4}
%	   	\multicolumn{1}{|c|}{}	& \multicolumn{1}{c|}{\multirow{3}{*}{}} & \multicolumn{1}{c|}{\multirow{2}{*}{}}& \multicolumn{1}{c|}{\multirow{2}{*}{0}} \\ \cline{3-3}
%	   	\multicolumn{1}{|c|}{}	& \multicolumn{1}{c|}{} & $y_i -1$ & \multicolumn{1}{c|}{\multirow{2}{*}{}} \\ \hline
	\end{tabular}
%	\hspace{-0.1cm}
%	\begin{tabular}{|c|c|c|c|} \hline
%	\multicolumn{4}{|c|}{\textbf{\acp{OBC}}}							\\ \hline
%		Case 				& $z_j$	& $y_j$	& $x_j$ 	\\ \hline
%	\multicolumn{1}{|c|}{\multirow{2}{*}{$z_i = 0, y_i = N_y - 1$}}	 &	\multicolumn{1}{c|}{\multirow{2}{*}{1}} & \multicolumn{1}{c|}{\multirow{2}{*}{$y_i$}} & $x_i$   \\ \cline{4-4}
%	   	\multicolumn{1}{|c|}{}	& \multicolumn{1}{c|}{\multirow{2}{*}{}}  & \multicolumn{1}{c|}{}& $N_x - 1 - (N_x - x_i) \% N_x$ \\ \hline
%	   	\multicolumn{1}{|c|}{\multirow{2}{*}{$z_i = 0, y_i = N_y - 1, x_i = 0$}}	 &	\multicolumn{1}{c|}{\multirow{2}{*}{1}} & \multicolumn{1}{c|}{\multirow{2}{*}{$y_i$}} & 0   \\ \cline{4-4}
%	   	\multicolumn{1}{|c|}{}	& \multicolumn{1}{c|}{\multirow{2}{*}{}}  & \multicolumn{1}{c|}{}& $N_x - 1$ \\ \hline
%	   	\multicolumn{1}{|c|}{\multirow{2}{*}{$z_i = 1, y_i = 0$}}	 &	\multicolumn{1}{c|}{\multirow{2}{*}{0}} & \multicolumn{1}{c|}{\multirow{2}{*}{$y_i$}} & $x_i$   \\ \cline{4-4}
%	   	\multicolumn{1}{|c|}{}	& \multicolumn{1}{c|}{\multirow{2}{*}{}}  & \multicolumn{1}{c|}{}& $(x_i + 1) \% N_x$  \\ \hline
%	   	\multicolumn{1}{|c|}{\multirow{2}{*}{$z_i = 1, y_i = 0, x_i = N_x - 1$}}	 &	\multicolumn{1}{c|}{\multirow{2}{*}{0}} & \multicolumn{1}{c|}{\multirow{2}{*}{$y_i$}} & 0   \\ \cline{4-4}
%	   	\multicolumn{1}{|c|}{}	& \multicolumn{1}{c|}{\multirow{2}{*}{}}  & \multicolumn{1}{c|}{}& $N_x - 1$ \\ \hline
%	\end{tabular}
	\label{tab:dummytable}
\end{table}

\subsection{\acp{TMD}}
\label{subsec:apTMD}

\begin{figure}[H]
\includegraphics[scale=0.8]{Applications/fillingVsE.png}
\includegraphics[scale=0.8]{Applications/fillingVsEnanoribbon.png}
	\caption[Filling factor as a function of the Fermi energy for \acs{TMD} monolayers and nanoribbons.]{Left: Filling factor as a function of the Fermi energy for a system with \acp{PBC}, as computed by diagonalizing the input matrix of our code, and by the hopping matrix in $\bm k$-space.
	Right: Comparison between the filling factors of a nanoribbon and a periodic system as a function of the Fermi energy, as computed by diagonalizing our input matrix. }
	\label{fig:fillingVsE}
\end{figure}
%\cleardoublepage
