\section{One-dimensional Chain}
\label{sec:1d-chain}

The analytical solution of the \acs{1D} Hubbard model at half filling indicates that the ground state is antiferromagnetic \cite{lieb_absence_1968}.
To build a physical picture of the Hubbard chain, start by considering the $\frac{U}{t} \gg 1$ limit. 
Then, the Hubbard model can be replaced by an effective atomic Heisenberg model defined in the Hilbert subspace with one electron per site, and antiferromagnetic (AF) order sets in.
In the Hubbard model, at zero temperature, it is found that upon decreasing $U$, the system is not only \acs{AF} for large $U$, but remains an \acs{AF} \emph{insulator} down to $U \rightarrow 0$, becoming a \emph{conductor} and losing \acs{AF} order only at $U = 0$.
Thus, for high enough $\beta = 1 / T$, we expect to see signs of \acs{AF} order for all $0 < U < \infty$.
Upon decreasing $\beta$, thermal fluctuations tend to completely destroy long range order.
Conversely, as $\beta$ is increased, we expect to see a divergence in $\chi$, corresponding to a phase transition to the antiferromagnetic ground state.
We identify it by measuring both the equal-time and time-displaced spin-spin correlation functions, $\left\langle S^z_i  S^z_j \right\rangle$ and $\left\langle S^z_i (\tau) S^z_j (0) \right\rangle$ with Monte Carlo.
Fourier transforming as per Eqs.(\ref{eq:S(q)},\ref{eq:chi(q)}), we obtain a peak at $q = \pi$ in the magnetic structure factor $S ( q ) $, and in the magnetic susceptibility $\chi (q)$.
Both peaks increase in magnitude as temperature is decreased, and in fact, within statistical uncertainty, the \say{staggered} susceptibility $\chi (\pi)$ appears to diverge very near $T_c = 0$.
Contrastingly, the $q = 0$ components of both the structure factor and the susceptibility go to zero as the temperature is decreased, indicating  no sign of ferromagnetic ordering in the ground state (see Fig.(\ref{fig:corr_FT})).
\begin{figure}[H]
\hspace{0.2cm}
\includegraphics[scale=0.53]{Applications/magCorr.png}
\hspace{0.7cm}
\includegraphics[scale=0.53]{Applications/SandChi.png}
\caption[Spin-spin correlation function, magnetic structure factor, and susceptibility for a 64 site Hubbard chain at $\beta t = 25 $, for $U = 4t$.]{Left: Spin-spin correlation function $\left\langle S^z_i S^z_j \right\rangle$ centered on the middle of the chain.
Right: Magnetic structure factor, and susceptibility for a 64-site Hubbard chain at $\beta t = 25 $, for $U = 4t$, at half filling $\left\langle n \right\rangle = 1$.\label{fig:corr_FT}}
\end{figure}
%According to spin-wave theory, the magnetic susceptiblity can be obtained from the spin-wave dispersion relation $\varepsilon ( k ) \propto k$, and at low temperature, we have
%\begin{equation}
%\chi \propto \int k dk \frac{e^{\beta k}}{(e^{\beta k} -1)^2} \propto T \ln T
%\end{equation}
%
%We find that within statistical uncertainty (which is bigger for $q=0$ for both the structure factor and the susceptibility), our results match this limit at low temperature already at $U = 4t$ (see Fig.(\ref{fig:Schi0})).
%\begin{figure}[H]
%\includegraphics[scale=0.53]{Applications/S0T.png}
%\hspace{-0.5cm}
%\includegraphics[scale=0.53]{Applications/Sus.png}
%\caption[$q=0$ components of the magnetic structure factor and susceptibility as a function of temperature, indicating that the ground state does not have ferromagnetic ordering.]{By performing simulations for varying temperature and looking at the $q=0$ components of the magnetic structure factor and susceptibility as a function of temperature, we see that the ground state does not have ferromagnetic ordering.
%Furthermore, our result agrees with the low temperature prediction of spin-wave theory for the Heisenberg model, the large $U$ limit of the Hubbard model.\label{fig:Schi0}}
%\end{figure}
In the Heisenberg limit, $U \gg t$, the staggered susceptibility $\chi_{\text{st}} \equiv \chi (\pi)$ diverges as $\chi_{\text{st}} ( T ) \propto 1 / (T - T_c) $.
Already at $U = 4 t$, we find exactly this behavior with a critical temperature $T_c$ very close to zero (see Fig.(\ref{fig:SchiPi})).
In fact, the ground state of the Hubbard model for the 1D chain is also antiferromagnetic, hence the divergence at $T_c = 0$ of the staggered susceptibility.
\vspace{-0.4cm}
\begin{figure}[H]
\includegraphics[scale=0.53]{Applications/SqT.png}
\includegraphics[scale=0.53]{Applications/stSus.png}
\caption[The magnetic structure factor and the  susceptibility have a peak at $q = \pi$ that increases as $T\rightarrow 0$, indicating \emph{antiferromagnetic ordering}.
 Divergence of the staggered susceptibility near $T_c = 0$, signaling the transition to the antiferromagnetic ground state.]{The magnetic structure factor and the susceptibility have a peak at $q = \pi$ that increases as $T\rightarrow 0$, indicating antiferromagnetic ordering.
The staggered susceptibility diverges near $T_c = 0$ (not exactly 0 because the system is finite), signaling the transition to the antiferromagnetic ground state.\label{fig:SchiPi}}
\end{figure}

We ran \texttt{QUEST} for the same 64-site chain we simulated with our code, using $\beta t = 25$, taking a half filled chain ($\left\langle n \right\rangle = 1$), and setting $U = 4t$.
We found a remarkable agreement between the measurements obtained using our code and using \texttt{QUEST}, namely in the magnetic structure factor, which we show in Fig.(\ref{fig:quest_time}).
Then, we took a small $2$-site system to compare the run time and verify the scaling with the number of slices ($\mathcal{O}(LN^3)$) of the determinant \ac{QMC} algorithm.
We noticed that \texttt{QUEST}'s algorithm suffers from large overhead time if the required precision via the Trotter error $\Delta \tau$, or the inverse temperature $\beta$ are large. 
This is due to the pre-conditioning needed to stabilize the products of the larger $L N \times L N$ matrices that are used in their algorithm (ours uses $N \times N$).
For large system sizes, \texttt{QUEST} is better than our algorithm because their implementation uses \emph{hybrid} \ac{QMC}, which (potentially) has subcubic $\mathcal{O}(LN)$ scaling (the cost of stabilizing the matrix product required to obtain the Green's matrix with sufficient accuracy at low temperature still scales with $N$ as $\mathcal{O}(N^3)$).

The results for the $2$-site system can be compared with the results obtained using exact diagonalization, a method which can only be used for small lattice sizes, and which we outlined in chapter \ref{cap:hubbard}.
We kept all the parameters, changing only the system size, and verified that our results agree with a similar study carried out in \cite{hirsch_discrete_1983}, confirming the validity of our implementation (see Fig.(\ref{fig:hirsch1982})).

\begin{figure}[H]
\hspace{0.1cm}
\includegraphics[scale=0.53]{Applications/s_compare.png}
\hspace{0.5cm}
\includegraphics[scale=0.53]{Applications/runtime2sites.png}
\caption[Comparison of the magnetic structure factor with that obtained using \texttt{QUEST}. Run time comparison.]{Left: Comparison of the magnetic structure factor with that obtained using \texttt{QUEST}.
The used parameters are mentioned in the body of the text.
Right: The run time using our code increases linearly with $L$, as expected.
The \texttt{QUEST} algorithm initially scales linearly, but then becomes much slower due to the large overhead time associated with pre-conditioning the comparably much larger matrices they use if $L$ is very big.\label{fig:quest_time}}
\end{figure}
\vspace{-0.5cm}
\begin{figure}[H]
\includegraphics[scale=0.53]{Applications/Ehirsch1982.png}
\hspace{0.3cm}
\includegraphics[scale=0.53]{Applications/SiSjhirsch1982.png}
\caption[Convergence of some of the measured observables to the value given by exact diagonalization for $N=2$, $\beta t = 2 $, $U = 4 t$.
Comparison with the results of \texttt{QUEST}.]{Convergence of some of the measured observables (left: total energy; right: spin-spin correlation) to the value given by exact diagonalization for $N=2$, $\beta t = 2 $, $U = 4 t$.
Comparison with the results of \texttt{QUEST}.\label{fig:hirsch1982}}
\end{figure}
