\section{One-dimensional Chain}
\label{sec:1d-chain}

The analytical solution of the $T= 0$ \acs{1D} Hubbard model indicates that there is no Mott transition \cite{lieb_absence_1968}.
Consider the $frac{U}{t} \gg 1$ limit, where the Hubbard model becomes an atomic Heisenberg model defined in the Hilbert subspace with one electron per site, and antiferromagnetic order sets in.
It is found that upon decreasing $U$, the system remains an antiferromagnetic insulator down to $U \rightarrow 0$, and becomes a conductor only at $U = 0$.
Thus, we expect antiferromagnetic order to set in for all $U$ for high enough $\beta$.
We identify it by studying the spin-spin correlator $\left\langle S^z_i S^z_j \right\rangle$.
Fourier transforming, we obtain a peak at $q = \pi$ in the magnetic structure factor $S ( q ) $, and in the magnetic susceptibility $\chi (q)$.