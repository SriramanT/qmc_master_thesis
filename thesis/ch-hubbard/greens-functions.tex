\section{Green's functions and Wick's theorem}\label{sec:green}

Green's functions are the core of the first perturbative, diagrammatic approaches to the Hubbard model.
They are very useful in reducing the enormous amount of variables that come into play in correlated systems to a manageable number.
Here we try do give some intuition about how they work since they are the central quantity in the \acs{QMC} method we will use.
For a very complete and thorough description of the ideas in this section see \cite{fetter_quantum_2003}.

Considering the imaginary-time variable of the previous chapter $\tau = i t$, for $\tau > 0$, the Green's function is defined as

\begin{equation}
G_{i  j} ( \tau, 0 ) = \left\langle c_{i} ( \tau ) c_{j}^\dagger ( 0 ) \right\rangle  \, , \text{with} \,\, c_{i} ( \tau ) = e^{\mathcal{H} \tau } c_{i} ( 0 ) e^{- \mathcal{H} \tau } 
\end{equation}

\subsection{Single site case}

The Hubbard Hamiltonian does not distinguish between spin-up and spin-down sectors.
Thus, without loss of generality, let us consider the spin-up sector, and compute $G_\uparrow (\tau, 0) =  \left\langle c_{\uparrow} ( \tau) c_{\uparrow}^\dagger ( 0 ) \right\rangle$.
Only the states $\left| n_{\uparrow} n_{\downarrow} \right\rangle = $ $\left| 0\, 0 \right\rangle$, $\left| 0\, 1 \right\rangle$ contribute to the expectation due to the creation operator on the right, which gives 0, unless there is no spin up electron already in the state it acts upon.

\begin{equation}
\begin{split}
c_{\uparrow} ( \tau) c_{\uparrow}^\dagger ( 0 )  \left| 0\, 0 \right\rangle &= e^{\mathcal{H} \tau } c_{\uparrow} ( 0 ) e^{-\mathcal{H} \tau } c_{\uparrow}^\dagger ( 0 )  \left| 0\, 0 \right\rangle = e^{\mathcal{H} \tau } c_{\uparrow} ( 0 ) e^{-\mathcal{H} \tau } \left| 0\, 1 \right\rangle \\
 &= e^{\mathcal{H} \tau } c_{\uparrow} ( 0 ) e^{U \tau / 4 + \mu \tau } \left| 1\, 0 \right\rangle = e^{\mathcal{H} \tau } e^{U \tau / 4 + \mu \tau } \left| 0\, 0 \right\rangle = e^{U \tau / 2 + \mu \tau } \left| 0\, 0 \right\rangle \\
 c_{\uparrow} ( \tau) c_{\uparrow}^\dagger ( 0 )  \left| 0\, 1 \right\rangle &= e^{\mathcal{H} \tau } c_{\uparrow} ( 0 ) e^{-\mathcal{H} \tau } c_{\uparrow}^\dagger ( 0 )  \left| 0\, 1 \right\rangle = e^{\mathcal{H} \tau } c_{\uparrow} ( 0 ) e^{-\mathcal{H} \tau } \left| 1\, 1 \right\rangle \\
 &= e^{\mathcal{H} \tau } c_{\uparrow} ( 0 ) e^{- U \tau / 4 + 2 \mu \tau } \left| 1\, 1 \right\rangle = e^{\mathcal{H} \tau } e^{- U \tau / 4 + 2 \mu \tau } \left| 0\, 1 \right\rangle = e^{U \tau / 2 + \mu \tau } \left| 0\, 1 \right\rangle
\end{split}
\end{equation}

Using the expression for the partition function that we obtained in equation (\ref{eq:singleSitePartition}), we arrive at

\begin{equation}
G_\uparrow (\tau, 0) = \frac{ e^{\tau ( U / 2 + \mu )} e^{-\beta U / 4} + e^{-\tau ( U / 2 - \mu )} e^{\beta (U / 4 + \mu)} }{ e^{-\beta U / 4} ( 1 + 2 e^{\beta ( U /2 + \mu)} + e^{2\beta \mu} ) } ,
\end{equation}
which, at half filling becomes

\begin{equation}
G_\uparrow (\tau, 0) = \frac{ e^{\tau U / 2} e^{-\beta U / 4} + e^{-\tau U / 2} e^{\beta U / 4 } }{ 2 e^{-\beta U / 4}  + 2 e^{\beta U / 4} } ,
\end{equation}

There is a well known relation between the Green's function and the spectral density $A ( \omega )$, which may be regarded as a local density of states:

\begin{equation}\label{eq:specDens}
G ( \tau, 0 ) =  \int_{-\infty}^{+\infty} A ( \omega ) \frac{e^{-\omega \tau} }{ e^{-\beta \omega} + 1 } d\omega ,
\end{equation}

If we replace the following expression for the spectral density in equation (\ref{eq:specDens}), we recover the result for the half filled case.

\begin{equation}
A ( \omega ) = \frac{1}{2} \bigg( \delta ( \omega - \frac{U}{2} ) + \delta ( \omega + \frac{U}{2} ) \bigg)
\end{equation}

We could do a similar calculation for $\mu \neq 0$ by changing the spectral density adequately, but the algebra is slightly more cumbersome, and the result does not bring additional insight.

The spectral density consists of two delta functions separated by $U$, which is reminiscent of our result for the Mott insulating gap.
In the same way that the gap softens (eventually disappearing) when we introduce hopping, the spectral function for the full Hubbard Hamiltonian changes accordingly, reflecting the same information about the properties of the system as the Green's function, encoded in a different manner.
In \acs{QMC}, we can access $G(\tau, 0)$ and deduce the properties of the system from it.

\subsection{Non-interacting case}

In this limit, we can compute $G_{i j} ( \tau, 0 )$ analytically by going to momentum space.

\begin{equation}
c_{\bm k} ( \tau ) = e^{ \mathcal{H} \tau } c_{\bm k} ( 0 ) e^{-\mathcal{H} \tau } = e^{-\varepsilon_{\bm k} \tau } c_{\bm k} ( 0 )
\end{equation}

This equation can be verified by acting with the left hand side and with right hand side on the states $\left| 0 \right\rangle$ and $\left| 1 \right\rangle$, and noting that the result is the same.
Alternatively, one can use the equation of motion $\partial_\tau \hat{A}(\tau) = [ \mathcal{H}, \hat{A} (\tau) ]$.
To generalize the result of equation (\ref{eq:eqGreenNonInt}) for the \emph{equal-time} Green's function to  \emph{unequal-time} we transform the fermionic operators in $G$ to momentum space, and use $\left\langle c_{\bm k} c_{\bm k}^\dagger \right\rangle = 1 - f_{\bm k}$ to obtain

\begin{equation}
G_{i j}(\tau, 0) = \frac{1}{N} \sum_{\bm k} e^{i \bm k \cdot (\bm R_i - \bm R_j )} ( 1 - f_{\bm k} ) e^{-\varepsilon_{\bm k} \tau } ,
\end{equation}
which is translationally invariant corresponding to the symmetry of the Hamiltionian.

We can generalize our definition of the Green's function by using the time-ordering operator $\mathcal{T}$:

\begin{equation}
G_{\bm k}(\tau, 0) = - \left\langle \mathcal{T} c_{\bm k} ( \tau) c_{\bm k}^\dagger ( 0 ) \right\rangle ,
\end{equation}
where

\begin{equation}
\mathcal{T} c_{\bm k} ( \tau) c_{\bm k}^\dagger ( 0 ) =
\begin{cases}
c_{\bm k} ( \tau) c_{\bm k}^\dagger ( 0 ), \,\, \tau > 0 \\
- c_{\bm k}^\dagger ( 0 ) c_{\bm k} ( \tau) , \,\, \tau < 0
\end{cases}
\end{equation}

An important property follows immediately from this definition: $G ( \tau + \beta, 0 ) = - G( \tau, 0 )$ for $ -\beta < \tau < 0$.
The imaginary-time anti-periodicity constraint implies that the frequencies that appear when we Fourier transform are the so called (fermionic) Matsubara frequencies $\omega_n = \frac{(2n + 1) \pi}{\beta}$.\footnote{Analogously, for bosons, imaginary-time periodicity implies that $\omega_n = 2n \pi / \beta$}

\begin{equation}
G ( i \omega_n ) = \int_0^\beta \frac{d\tau}{\beta} G( \tau, 0) e^{i\omega_n \tau} \quad\quad G (\tau, 0) = \sum_n G ( i \omega_n ) e^{ - i \omega_n \tau}
\end{equation}

In momentum space, and imaginary time, the Green's function then become

\begin{equation}
G_{\bm k} (\tau, 0 ) =
\begin{cases}
-e^{-\varepsilon_{\bm k}} ( 1 - f_{\bm k} ) , \,\, 0 < \tau < \beta \\
e^{-\varepsilon_{\bm k}} f_{\bm k} , \,\, -\beta < \tau < 0 ,
\end{cases}
\end{equation}
which leads to

\begin{equation}
G_{\bm k} ( i \omega_n ) = \frac{1}{i\omega_n - \varepsilon_{\bm k} }
\end{equation}
in frequency space. This result may also be obtained by taking the partial derivative of the time-ordered Green's function written in the form $
G_{\bm k} ( \tau, 0 ) = \left\langle c_{\bm k} ( \tau) c_{\bm k}^\dagger ( 0 ) \right\rangle \theta ( \tau ) - \left\langle c_{\bm k} ( 0 ) c_{\bm k}^\dagger ( \tau ) \right\rangle \theta ( -\tau )
$
 and Fourier transforming both sides to solve for $G ( i \omega_n )$.
Taking a time derivative of $G$ implies computing commutators of $\mathcal{H}$ with the fermionic operators.
The equation closes for quadratic Hamiltonians, which we, of course, know to be soluble.

\subsection{Finite temperature Wick's theorem for fermions}
\label{subsec:wick}

As was emphasized in section \ref{sec:exactSolutions}, the key concept that will be used in deriving the auxiliary field method is a mapping to a single-particle problem defined in terms of a quadratic Hamiltonian.
Thus, Wick's theorem may be applied to simplify products of fermionic operators.
In \cite{fetter_quantum_2003}, the theorem is rigorously proven for both the ground state and the finite temperature cases.
Here, we summarize the elements that will be useful in the next chapter, and leave a more detailed discussion following the discussion of \cite{molinari_notes_2017} to appendix \ref{ap:hubbardObSol}.

In general, Green's functions are a much more complicated object than in the simple examples treated so far.
Fortunately, our mapping to a single particle problem allows us to express many-particle Green's functions as a function of single particle Green's functions sampled over the auxiliary field's configurations.
The latter will turn out to play a crucial role in the sampling process itself.
Moreover, any observable may be written in terms of these single particle Green's functions (for a given auxiliary field configuration) by using Wick's theorem.
Thus, measuring any observable requires simply gathering the necessary elements of the single particle Green's function, since the latter must be computed already to sample the auxiliary field's configurations.
This observation stems solely from the fact that the electrons are decoupled once the auxiliary field is introduced, reducing the interacting problem to a single particle problem.

The single particle (equal-time) Green's function for a fixed field configuration $\bm h$ is defined as $G_{ij} \equiv \left\langle c_i c_j^\dagger \right\rangle_{\bm h}$.
Any observable of interest is reduced to a sum of terms of type

\begin{equation}
\begin{split}
\left\langle c_{i_1}^\dagger c_{i_2} c_{i_3}^\dagger c_{i_4} \right\rangle_{\bm h} &= \left\langle c_{i_1}^\dagger c_{i_2} \right\rangle_{\bm h} \left\langle c_{i_3}^\dagger c_{i_4} \right\rangle_{\bm h} + \left\langle c_{i_1}^\dagger c_{i_4} \right\rangle_{\bm h} \left\langle  c_{i_2} c_{i_3}^\dagger  \right\rangle_{\bm h} \\
&= \bigg( \delta_{i_1 i_2} - G_{i_2 i_1}  \bigg) \bigg( \delta_{i_3 i_4} - G_{i_4 i_3}  \bigg) + \bigg( \delta_{i_1 i_4} - G_{i_4 i_1}  \bigg) G_{i_2 i_3}
\end{split} ,
\end{equation}
by using Wick's theorem, and the fact that the Hamiltonian conserves the particle number.
As we shall see later, an analogous relation holds for quantities defined of \emph{time-displaced} Green's functions.
The proof consists of writing those quantities in terms of a combination of equal-time Green's functions and matrices that represent the single particle propagators.
After this step, one can write a version of Wick's theorem for time-displaced Green's functions.
