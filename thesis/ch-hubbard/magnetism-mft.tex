\section{Magnetism and mean field theory}\label{sec:magMFT}

In this section we will build a picture of magnetism in the Hubbard model in increasing level of sophistication.
As our degenerate perturbation theory  calculation of section (\ref{sec:effectiveHeisenber}) showed, the on-site interaction favors the situation in which neighboring fermions have opposite spins through an Heisenberg type interaction.
A different approach leads to the Stoner criterion for ferromagnetism.
The argument is based on creating an imbalance between the numbers of spin-up and spin-down fermions, and analyzing the interplay between the resulting increase in kinetic energy, and decrease in potential energy.
Finally, we formulate a (static) mean field theory for the Hubbard model, and discuss how the picture it gives relates to the non-interacting case.

\subsection{Stoner criterion for ferromagnetism}
\label{subsec:stoner}

Pauli's exclusion principle gives a prescription on how to fill fermionic energy levels so as to yield the lowest possible total energy.
Start from the lowest level, and start filling each level of higher consecutively with two electrons, one of each spin.
This procedure requires the number of spin-up and spin-down electrons to be the same.
Otherwise, there is an energy cost, since we are obliged to fill higher levels.
