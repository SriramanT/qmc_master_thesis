\section{Stabilization}
\label{subsec:stabilization}

The Green's function is needed both to perform importance sampling and to make measurements.
Naively, its computation would involve multiplying a long chain of $\bm B$-matrices, adding the identity, and then taking the inverse.
We saw that this costly procedure can be substituted by a more efficient update scheme, which involves wrapping the Green's function as one sweeps between imaginary time slices.

The stability of this procedure depends on the conditioning of the matrices at hand.
As the temperature is lowered, or the system size increased, round-off errors accumulate and precision is gradually lost.
Thus, we must compute the Green's function from scratch \say{naively}.

As $\beta$ increases more and more, the problem becomes so severe that the Green's function cannot be computed at all!
This is an effect due to finite-precision in computers.
$\bm B$-matrices potentially contain largely different energy scales.
As more and more of them are multiplied together, the condition number of the product increases, and the situation worsens, with the energy scales becoming exponentially divergent.
