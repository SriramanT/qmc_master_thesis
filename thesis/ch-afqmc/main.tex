% %%%%%%%%%%%%%%%%%%%%%%%%%%%%%%%%%%%%%%%%%%%%%%%%%%%%%%%%%%%%%%%%%%%%%%
% Dummy Chapter:
% %%%%%%%%%%%%%%%%%%%%%%%%%%%%%%%%%%%%%%%%%%%%%%%%%%%%%%%%%%%%%%%%%%%%%%

% %%%%%%%%%%%%%%%%%%%%%%%%%%%%%%%%%%%%%%%%%%%%%%%%%%%%%%%%%%%%%%%%%%%%%%
% The Introduction:
% %%%%%%%%%%%%%%%%%%%%%%%%%%%%%%%%%%%%%%%%%%%%%%%%%%%%%%%%%%%%%%%%%%%%%%
\fancychapter{Auxiliary Field Quantum Monte Carlo}
\label{cap:afqmc}

\slshape

The interactions between the electrons in a solid give rise to effects that arise specifically due to the many-body nature of the system. The Hubbard model is a minimal model that encapsulates electron correlations. It goes beyond the periodic ionic potential perturbation to the free electron gas or tight binding approaches, which lead to band theory. As we have seen, one obtains the Hubbard Hamiltonian by adding the simplest possible electron-electron interaction term to a tight binding Hamiltonian: an on-site interaction term that penalizes double occupancy of a site. From it, we can make predictions about properties of a strongly correlated system, namely magnetic and superconducting behavior, and metal-insulator transitions.

Auxiliary-field, or Determinant Quantum Monte Carlo \footnote{AFQMC or DQMC, respectively.} is a simulation method that is commonly used to simulate the Hubbard model, allowing one to capture the elusive effects of electron correlations, for example in the two-dimensional graphene-like nanostructures we are concerned with.

Among the various methods belonging to the family of QMC methods, AFQMC has the advantage of  allowing us to circumvent the sign problem for the half filled Hubbard model. The sign problem is an uncontrolled numerical error due to the antisymmetry of the many-electron wave function, leading to oscillations in the sign of the quantities that we are interested in measuring. These oscillations deem the algorithm exponentially complex in the size of the system, in general, but it is possible to overcome this hurdle for a class of models, namely the Hubbard model at half filling. The difficulty lies in computing averages of a quantity $X$ that is very close to zero, on average, but has a large variance, i.e. $\sigma_X / \left\langle X \right\rangle \gg 1$.

\normalfont

\section{Trotter-Suzuki Decomposition}
\label{subsec:trotter}

We seek a computable approximation of the projection operator $\mathcal{P}$ defined in equation (\ref{eq:projection}). As we shall see, it is found by using a discrete Hubbard-Stratonovich transformation. This transformation introduces an auxiliary field (consisting basically of Ising spins), and we use Monte Carlo to sample configurations from the distribution corresponding to this \emph{classical} configuration space.

For now, let us assume half filling $\mu = 0$, so that there is no sign problem. In fact, many interesting phenomena occur at half filling, for example magnetic ordering and the Mott metal-insulator transition.

\subsection{Hubbard-Stratonovich transformation}

In section \ref{exactSolutions}, we found exact solutions for particular instances of the Hubbard model by finding a closed form for the partition function \cite{hou_numerical_2009}. When devising a numerical method, a good sanity check is to verify that it satisfactorily approximates the partition function.

The operators $\mathcal{H}_K$ and $\mathcal{H}_V$ of equation (\ref{eq:hubbard}) do not commute. This impedes us from factorizing the exponential of their sum $e^{-\beta (\mathcal{H}_K + \mathcal{H}_V)}$ exactly. The Trotter-Suzuki decomposition leads to the sought approximate factorization that is used to approximate the partition function. Quantum states evolve according to

\begin{equation}
\left| \psi (\tau) \right\rangle = e^{-\tau \mathcal{H} } \left| \psi (0) \right\rangle ,
\end{equation}
where $\tau = it$ is the imaginary time. Recall that Diffusion Monte Carlo is based on this imaginary time evolution, filtering out the ground state as the state that takes longer to vanish exponentially. We now seek a finite temperature method. To find it, we invoke an analogy with the evolution of a quantum system according to the previous equation.

Taking the scalar product with a position eigenstate $ \left\langle \bm x \right|$, we obtain $\psi (\bm x, \tau) = \left\langle \bm x | \psi (\tau) \right\rangle$. Using the closure relation $\int d\bm y \left| \bm y \right\rangle  \left\langle \bm y \right| = 1$, we get

\begin{equation}
\psi ( \bm x , \tau ) = \int d\bm y \left\langle \bm x | e^{-\tau \mathcal{H}} | \bm y \right\rangle \psi (\bm y, 0)
\end{equation}

The wave function at position $\bm x$ and time $t$ may be obtained by this equation as long as we know the wave function at $\tau = 0$, $\psi (\bm y, 0)$ for all points in space $\bm y$. The evolution operator matrix element, or Green function, 

\begin{equation}
G ( \bm x, \tau | \bm y, 0 ) \equiv \left\langle \bm x | e^{-\tau \mathcal{H}} | \bm y \right\rangle ,
\end{equation}
as the wave function, satisfies the Schr\"odinger equation, with the initial condition $\psi (\bm y, 0) = \delta (\bm x - \bm y )$. It is then the probability of presence at $\bm x, t$ of a wave packet centered at $\bm y$ at $t = 0$. Note that the solution of the Schr\"odinger equation is then analogous to that of a diffusion equation (that in turn one may obtain as the continuum limit of a random walk). We may write $G$ as a linear combination of the eigenstates of the Hamiltonian

\begin{equation}
G (\bm x, \tau | \bm y, 0) = \sum_\alpha \psi_\alpha^\star (\bm y) \psi_\alpha (\bm x) e^{-E_\alpha \tau} ,
\end{equation}
where we immediately note a striking similarity with equation (\ref{eq:z_asEigen}). The correspondence $\psi (\bm x, \tau) \mapsto Z_\beta$, where $\tau \mapsto \beta$, with respect to section \ref{exactSolutions} makes the analogy evident. $\bm x$ has no correspondence because it is not a parameter, it is just an arbitrary position that we fixed for the sake of the argument.

Computing the partition function at finite temperature

\begin{equation}
Z_\beta = \text{Tr} \big( e^{-\beta \mathcal{H} } \big)
\end{equation}
is analogous to computing the Green function of a quantum system evolving in imaginary time. The inverse temperature $\beta$ now represents the imaginary time $\tau = it$, and $Z_\beta$ may be simply thought of as the wave function of the analogous quantum system at imaginary time (temperature) $\beta$.
 
This expression is not very amenable to numerical computation since it contains an exponential of a sum of operators $\mathcal{H}_K + \mathcal{H}_V$, which is not factorizable and involves computing an infinite number of commutators containing these two operators, as per the Zassenhaus formula, valid for any two generic operators $X$ and $Y$:

\begin{equation}\label{eq:zassenhaus}
\begin{split}
&e^{\delta (X+Y)}=e^{\delta X} e^{\delta Y} e^{-{\frac {\delta^{2}}{2}}[X,Y]} e^{{\frac {\delta^{3}}{6}}(2[Y,[X,Y]]+[X,[X,Y]])} \\
&e^{{\frac {-\delta^{4}}{24}}([[[X,Y],X],X]+3[[[X,Y],X],Y]+3[[[X,Y],Y],Y])} \, ... , 
\end{split}
\end{equation}
where $\delta \in \mathbb{C}$ is an expansion parameter.

Dividing the imaginary time interval $[0, \beta ]$ into $L$ equal sub-intervals of width $\Delta \tau = \beta / L$, we obtain

\begin{equation}
Z = \Tr \bigg( \prod_{l=1}^L e^{-\Delta\tau \mathcal{H} } \bigg) ,
\end{equation}
which is now a product of exponentials of operators multiplied by a constant that can be made small by increasing $L$. The Trotter-Suzuki decomposition follows from truncating equation (\ref{eq:zassenhaus}), and keeping only the first order term in $t$, i.e. the one in $\Delta \tau$ in our case.

\begin{equation}\label{eq:Z_propagator}
Z = \Tr \bigg( \prod_{l=1}^L e^{-\Delta\tau \mathcal{H}_K } e^{-\Delta\tau \mathcal{H}_V } \bigg) + \mathcal{O}(\Delta \tau^2) 
\end{equation}

The kinetic energy term is quadratic in the fermion operators, and is spin-independent and thus may be separated into spin up and spin down components

\begin{equation}
e^{-\Delta\tau \mathcal{H}_K} = e^{-\Delta\tau \mathcal{H}_{K_\uparrow}} e^{-\Delta\tau \mathcal{H}_{K_\downarrow}} ,
\end{equation}
where $\mathcal{H}_{K_\sigma} = -t \bm c_\sigma^\dagger \bm K  \bm c_\sigma$.

The potential energy term, however, is quartic. Surprisingly, it is possible to express it in quadratic form by introducing an extra degree of freedom, the so called \emph{Hubbard-Stratonovich (HS) field} $\bm h \equiv (h_i)_{i=1}^N$, in which each element is essentially an Ising spin. First, note that number operators on different sites commute, so that we have

\begin{equation}
\begin{split}
e^{-\Delta\tau \mathcal{H}_V} &= e^{-U \Delta\tau \sum_{i=1}^N (n_{i\uparrow} - 1/2 ) (n_{i\downarrow} - 1/2 )} \\
&= \prod_i e^{-U \Delta\tau (n_{i\uparrow} - 1/2 ) (n_{i\downarrow} - 1/2 )}
\end{split}
\end{equation}

Now we introduce the discrete Hubbard Stratonovich transformation for $U > 0$ that allows us to recast the equation above in terms of a non-interacting quadratic term $(n_{i\uparrow} - n_{i\downarrow} )$.

\begin{equation}\label{eq:discreteHS}
e^{-U \Delta\tau (n_{i\uparrow} - 1/2 ) (n_{i\downarrow} - 1/2 )} = c_U \sum_{h_i = \pm 1} e^{\nu h_i (n_{i\uparrow} - n_{i\downarrow} )},
\end{equation}
where $c_U = \frac{1}{2} e^{-\frac{U\Delta \tau}{4}}$ and $\nu = \text{arcosh} ( e^{\frac{U\Delta\tau}{2}})$.

To prove this identity, let us write down how the operators  $(n_{i\uparrow} - 1/2 ) (n_{i\downarrow} - 1/2 )$ and $(n_{i\uparrow} - n_{i\downarrow} )$ act on a state on a given site.

\begin{equation}
\begin{split}
(n_{i\uparrow} - 1/2 ) (n_{i\downarrow} - 1/2 )&
\begin{cases}
\left| \, \, \right\rangle = \frac{1}{4} \left| \, \, \right\rangle \\
\left| \uparrow \right\rangle = -\frac{1}{4} \left| \uparrow \right\rangle \\
\left| \downarrow \right\rangle = -\frac{1}{4} \left| \downarrow \right\rangle \\
\left| \uparrow \downarrow \right\rangle = \frac{1}{4} \left| \uparrow \downarrow \right\rangle
\end{cases}\\
(n_{i\uparrow} - n_{i\downarrow} )&
\begin{cases}
\left| \, \, \right\rangle = 0\left| \, \, \right\rangle \\
\left| \uparrow \right\rangle = \left| \uparrow \right\rangle \\
\left| \downarrow \right\rangle = \left| \downarrow \right\rangle \\
\left| \uparrow \downarrow \right\rangle = 0 \left| \uparrow \downarrow \right\rangle
\end{cases}
\end{split}
\end{equation}

Now we simply compare the action of the operators on the left hand side and on the right hand side of equation (\ref{eq:discreteHS}) and find the desired relation by defining

\begin{equation}
\cosh \nu =  \frac{e^\nu + e^{-\nu} }{2} \equiv e^{\frac{U\Delta \tau}{2}}
\end{equation}

\begin{equation}
\begin{split}
&e^{-U \Delta\tau (n_{i\uparrow} - 1/2 ) (n_{i\downarrow} - 1/2 )} \left| \psi \right\rangle = e^{-\frac{U\Delta \tau}{4}} \left| \psi \right\rangle \, , \left| \psi \right\rangle = \left| \, \, \right\rangle, \left| \uparrow \downarrow \right\rangle \\
&e^{-U \Delta\tau (n_{i\uparrow} - 1/2 ) (n_{i\downarrow} - 1/2 )} \left| \uparrow (\downarrow) \right\rangle = e^{\frac{U\Delta \tau}{4}} \left| \uparrow (\downarrow) \right\rangle \\
&c_U \sum_{h_i = \pm 1} e^{\nu h_i (n_{i\uparrow} - n_{i\downarrow} )} \left| \psi \right\rangle = e^{-\frac{U\Delta \tau}{4}} \left| \psi \right\rangle \, , \left| \psi \right\rangle = \left| \, \, \right\rangle, \left| \uparrow \downarrow \right\rangle \\
&c_U \sum_{h_i = \pm 1} e^{\nu h_i (n_{i\uparrow} - n_{i\downarrow} )} \left| \uparrow (\downarrow) \right\rangle= \frac{e^\nu + e^{-\nu}}{2} e^{-\frac{U\Delta \tau}{4}}  \left| \uparrow (\downarrow) \right\rangle
\end{split}
\end{equation}

Note that we require $U > 0$ so that there exists $\nu \in \mathbb{R}$ such that $\cosh \nu = e^{U\Delta \tau / 2}$. A similar reasoning could be made for $U < 0$. Additionaly, other transformations that recast other types of quartic terms in terms of quadratic ones exist, but we shall not need them in what follows \cite{hirsch_monte_1983}. The transformation we derived is the one we will use throughout.

We have now made progress. At the expense of introducing an extra $N$-dimensional HS-field $\bm h$, we obtained an \emph{exact} representation of the quartic term in terms of quadratic terms \cite{hou_numerical_2009}.

\begin{equation} 
 e^{-\Delta\tau \mathcal{H}_V} = \prod_{i=1}^N \bigg( c_U \sum_{h_i = \pm 1} e^{\nu h_i ( n_{i\uparrow} - n_{i\downarrow} )} \bigg),
\end{equation} 
which can be manipulated to arrive at a more compact form.

\begin{equation}\label{eq:exp_quartic}
\begin{split}
&e^{-\Delta\tau \mathcal{H}_V} =  (c_U)^N \sum_{h_i = \pm 1} e^{\nu h_i ( n_{1\uparrow} - n_{1\downarrow} )} \sum_{h_i = \pm 1} e^{\nu h_i ( n_{2\uparrow} - n_{2\downarrow} )}  \\
&... \sum_{h_i = \pm 1} e^{\nu h_i ( n_{N\uparrow} - n_{N\downarrow} )} \\
&= (c_U)^N \sum_{h_i = \pm 1} e^{\sum_{i=1}^N [(\nu h_i ( n_{i\uparrow} - n_{i\downarrow} ) ]} \\
&\equiv (c_U)^N \text{Tr}_h e^{\sum_{i=1}^N [(\nu h_i ( n_{i\uparrow} - n_{i\downarrow} ) ]} \\
&= (c_U)^N \text{Tr}_h e^{\sum_{i=1}^N \nu h_i n_{i\uparrow}} e^{-\sum_{i=1}^N \nu h_i n_{i\uparrow}} \\
&= (c_U)^N \text{Tr}_h ( e^{\mathcal{H}_{V_\uparrow}} e^{\mathcal{H}_{V_\downarrow}} ) ,
\end{split}
\end{equation}
where the spin up and spin down operators $\mathcal{H}_{V_\sigma}$ are defined as follows

\begin{equation}
\mathcal{H}_{V\sigma} = \sum_{i=1}^N \nu h_i n_{i\sigma} = \sigma \nu \bm c_\sigma^\dagger \bm V(\bm h) \bm c_\sigma,
\end{equation}
with $\bm V(\bm h)$ being simply the HS-field put into a diagonal $N\times N$ matrix: $\bm V(\bm h) \equiv \text{diag}(h_1, h_2, ..., h_N)$.

For each imaginary time slice $l$ (where $l \in [1, L]$) we may define a HS-field $\bm h_l$, which in turn specifies $\bm V_l$ and $\mathcal{H}_{V_\sigma}^l$. We may now replace the result of equation (\ref{eq:exp_quartic}) in equation (\ref{eq:Z_propagator}), and exchange the traces to obtain

\begin{equation}\label{eq:Z_quadratic}
\begin{split}
&Z_h = (c_U)^{NL} \text{Tr}_{\bm h} \Tr \bigg[ \prod_{l=1}^L \underbrace{\bigg( e^{-\Delta\tau  \mathcal{H}_{K_\uparrow}} e^{\mathcal{H}_{V_\uparrow}^l} \bigg)}_{B_{l, \uparrow}(\bm h_l)} \\
&\underbrace{\bigg( e^{-\Delta\tau  \mathcal{H}_{K_\downarrow}} e^{\mathcal{H}_{V_\downarrow}^l} \bigg)}_{B_{l, \downarrow}(\bm h_l)} \bigg],
\end{split}
\end{equation}
where all operators are now quadratic in the fermion operators:

\begin{equation}
\begin{split}
&\mathcal{H}_{K_\sigma} = - t \bm c_\sigma^\dagger \bm K \bm c_\sigma \\
&\mathcal{H}_{V_\sigma}^l = \sigma \nu \bm c_\sigma^\dagger \bm V_l (\bm h_l) \bm c_\sigma
\end{split}
\end{equation}
for $\sigma = \pm 1$ and $\bm V_l ( \bm h_l ) = \text{diag} ( h_{l, 1} , h_{l, 2}, ... , h_{l, N} )$.

Furthermore, we have defined the $\bm B$-matrices

\begin{equation}
\bm B_{l, \sigma} ( \bm h_l ) = e^{t \Delta \tau \bm K} e^{\sigma \nu \bm V_l (\bm h_l)}
\end{equation}

Note that the argument of the first exponential is positive since $\bm K$ is defined so that its entries are 0's and 1's; otherwise (defining $\bm K$ with 0's and $-1$'s) it would be negative.

The problem of computing the partition has been reduced to computing the trace of a product of exponentials of quadratic forms. Thus, we may still rewrite equation (\ref{eq:Z_quadratic}) by making use of the following identity.

Let $\mathcal{H}_l$ be quadratic forms of the fermion operators:

\begin{equation}
\mathcal{H}_l = c_i^\dagger (H_l)_{ij} c_j,
\end{equation}
where the summation is implied, and where $H_l$ are real matrices. Then, the following identity holds

\begin{equation}\label{eq:quadraticIdentity}
\Tr \big[ e^{-\mathcal{H}_1 } e^{-\mathcal{H}_2 } ... e^{-\mathcal{H}_L } \big] = \text{det} ( \bm I + e^{-H_L} e^{-H_{L-1}} ... e^{-H_1} )
\end{equation}

For simplicity, we present the proof for a simpler case, corresponding to a single $\bm B$-matrix, i.e. a product of exponentials of two quadratic operators \cite{hirsch_two-dimensional_1985}. It could then be easily extended to the more general case. Let the two arbitrary real matrices be $\bm M$ and $\bm N$. Then, a particular case of the previous identity is

\begin{equation}\label{eq:particularIdentity}
\Tr \big[ e^{-c_i^\dagger M_{ij} c_j} e^{-c_i^\dagger N_{ij} c_j} \big] = \text{det} ( \bm I + e^{-{\bm M}} e^{-{\bm N}} ) ,
\end{equation}
where a summation over repeated indices is implied, as it wil be throughout this proof.

To prove this identity, we start by proving that

\begin{equation}\label{eq:identity_prod_exps}
e^{-c_i^\dagger M_{ij} c_j}  e^{-c_i^\dagger N_{ij} c_j} = e^{-\sum_\nu c_\nu^\dagger \rho_\nu c_\nu} ,
\end{equation}
where $\lambda_\nu = e^{-\rho_\nu}$ are the eigenvalues of the matrix $e^{-{\bm M}} e^{-{\bm N}}$.

The proof consists of showing that any many-particle state are propagated in the same way when acted upon by any of these two operators, i.e. the LHS operator leads the system to the same state as the RHS operator.

A generic single-particle state reads

\begin{equation}
\left| \phi \right\rangle = \sum_j a_j c_j^\dagger \left| 0 \right\rangle ,
\end{equation}
where $a_j$ are arbitrary coefficients, and $\left| 0 \right\rangle$ is the vacuum state.

Let $\{\left| \mu \right\rangle \}$ be the basis in which the matrix $\bm N$ is diagonal. Using Dirac notation, we then have

\begin{equation}
\bm N = \sum_{\mu} \left| \mu \right\rangle n_\mu \left\langle \mu \right|
\end{equation}

Defining new fermionic operators

\begin{equation}\label{eq:changeBasis1}
\begin{split}
c_\mu &= \sum_j \left\langle \mu | j \right\rangle c_j \\
c_\mu^\dagger &= \sum_j \left\langle j | \mu \right\rangle c_j^\dagger ,
\end{split}
\end{equation}
which may be inverted to obtain

\begin{equation}\label{eq:changeBasis2}
\begin{split}
c_j &= \sum_\mu \left\langle j | \mu \right\rangle c_\mu \\
c_j^\dagger &= \sum_\mu \left\langle \mu | j \right\rangle c_\mu^\dagger ,
\end{split}
\end{equation}

Now we prove yet another identity that goes into proving equation (\ref{eq:identity_prod_exps}).

\begin{equation}\label{eq:LHS_identity}
e^{-c_i^\dagger N_{ij} c_j} = \prod_\mu \big[ \mathbbm{1} + ( e^{-n_\mu} - 1 ) c_\mu^\dagger c_\mu \big]
\end{equation}

\begin{equation*}
\begin{split}
&\exp({-c_i^\dagger N_{ij} c_j}) = \exp({ - \sum_{\mu\nu} \left\langle \mu | i \right\rangle c_\mu^\dagger N_{ij} \left\langle j | \nu \right\rangle c_\nu }) \\
&= \exp({-\sum_{ij}\sum_{ \mu\nu\sigma} \left\langle \mu | i \right\rangle  \left\langle i | \sigma \right\rangle c_\mu^\dagger n_\sigma \left\langle \sigma | j \right\rangle \left\langle j | \nu \right\rangle c_\nu }) , \\
& \text{(using the closure relation} \sum_i \left| i \right\rangle \left\langle i \right|= \mathbbm{1} \text{)} \\
&= \exp({-\sum_{ \mu\nu\sigma} \overbrace{\left\langle \mu | \sigma \right\rangle}^{\delta_{\mu\sigma}}  c_\mu^\dagger n_\sigma \overbrace{\left\langle \sigma | \nu \right\rangle}^{\delta_{\sigma\nu}} c_\nu }) \\
&= \exp({-\sum_\mu c_\mu^\dagger n_\mu c_\mu}) \\
&= \prod_\mu e^{-n_\mu \hat{n}_\mu} \\
&= \prod_\mu [ \mathbbm{1} + ( - n_\mu \hat{n}_\mu + \frac{n_\mu^2}{2!} \hat{n}_\mu^2 - \frac{n_\mu^3}{3!} \hat{n}_\mu^3 + ... ] \\
&= \prod_\mu [ \mathbbm{1} + ( - n_\mu + \frac{n_\mu^2}{2!} - \frac{n_\mu^3}{3!} + ... ) \hat{n}_\mu ] \\
& \text{(since  } \hat{n} = \hat{n}^k \text{  for all  } k \in \mathbb{N} \text{  for fermions since  } n = 0, 1 ) \\
&= \prod_\mu [ \mathbbm{1} + ( e^{-n_\mu} - 1 ) c_\mu^\dagger c_\mu ] \quad\quad \qedsymbol
\end{split}
\end{equation*}

Let

\begin{equation}
\left| \phi \right\rangle = \sum_j a_j c_j^\dagger \left| 0 \right\rangle
\end{equation}
be an arbitrary many-particle state. 

Now we use the previous identity to prove that applying the operator of equation (\ref{eq:LHS_identity}) to $\left| \phi \right\rangle$ we obtain

\begin{equation}
e^{-c_i^\dagger N_{ij} c_j} \left| \phi \right\rangle = \sum_j a_j' c_j^\dagger \left| 0 \right\rangle , 
\end{equation}
with
\begin{equation}
a_j' = \sum_{i} (e^{-B})_{ji} a_i
\end{equation}

We start by writing $\left| \phi \right\rangle$ in the basis $\{ \left| \mu \right\rangle \}$ (in which $\bm N$ is diagonal).

\begin{equation}
\left| \phi \right\rangle = \sum_{i,\mu} a_i \left\langle \mu | i \right\rangle c_\mu^\dagger \left| 0 \right\rangle
\end{equation}

Then, we apply the RHS of equation (\ref{eq:LHS_identity}) to $\left| \phi \right\rangle$ written in this basis.

\begin{equation}\label{eq:proofAuxIdentity}
\begin{split}
&\sum_\nu \bigg[ \mathbbm{1} + ( e^{-n_\mu} - 1 ) c_\nu^\dagger c_\nu \bigg] c_\mu^\dagger \left| 0 \right\rangle \\
=&\bigg[ \mathbbm{1} + ( e^{-n_\mu} - 1 ) c_\mu^\dagger c_\mu \bigg] c_\mu^\dagger \left| 0 \right\rangle \\
=& c_\mu^\dagger \left| 0 \right\rangle + (e^{-n_\mu - 1} - 1) c_\mu^\dagger \left| 0 \right\rangle \\
=& c_\mu^\dagger e^{-n_\mu} \left| 0 \right\rangle
\end{split}
\end{equation}

\begin{equation}
\begin{split}
&\sum_\nu \bigg[ \mathbbm{1} + ( e^{-n_\mu} - 1 ) c_\nu^\dagger c_\nu \bigg] \left| \phi \right\rangle \\
=& \sum_{i\mu} \left\langle \mu | i \right\rangle a_i e^{-n_\mu} c_\mu^\dagger \\
=& \sum_{j\mu i} \left\langle j | \mu \right\rangle e^{-n_\mu} \left\langle \mu | i \right\rangle a_i \left| j \right\rangle \\
=& \sum_{j i} \underbrace{\sum_{\mu\nu} \left\langle j | \mu \right\rangle e^{-N_{\mu\nu}} \left\langle \nu | i \right\rangle}_{(e^{-\bm N})_{ji}} a_i \left| j \right\rangle \\
=& \sum_j a_j' c_j^\dagger \left| 0 \right\rangle
\end{split}
\end{equation}

Similarly, by repeating the procedure performing a change of basis to the eigenbasis of $\bm M$, we obtain the more general relation

\begin{equation}\label{eq:propagation_exps}
\begin{split}
&e^{-c_i^\dagger M_{ij} c_j} e^{-c_i^\dagger N_{ij} c_j} \left| \phi \right\rangle = \sum_j a_j'' c_j^\dagger \left| 0 \right\rangle \\
& a_j'' = \sum_i ( e^{-\bm M} e^{-\bm N} )_{ji} a_i
\end{split}
\end{equation}

The amplitude of a propagated state is given by multiplying the initial amplitude by the matrix $e^{-\bm M} e^{-\bm N}$, whichever the basis we choose. Then, since equation (\ref{eq:propagation_exps}) holds in particular for the choice of the eigenbasis of $e^{-\bm M}e^{-\bm N}$ as our basis of single-particle states, if we start with an eigenstate

\begin{equation}
\left| \phi \right\rangle = c_\nu^\dagger \left| 0 \right\rangle ,
\end{equation}
then the amplitude of the propagated state will be given by

\begin{equation}
(e^{-\bm M} e^{-\bm N} )_{\nu\nu} = e^{-\rho_\nu} ,
\end{equation}
the same as we would obtain from equation (\ref{eq:identity_prod_exps}). Clearly, if we start with a state that is an arbitrary combination of states of the eigenbasis, we would obtain the identity (\ref{eq:identity_prod_exps}).

The identity was proven for a single-particle state. Does it generalize to more than one particle? As we did before, we start with propagation by a single factor $e^{-\bm N}$. Take a two-particle state

\begin{equation}
\left| \phi \right\rangle = c_{\mu_1}^\dagger c_{\mu_2}^\dagger \left| 0 \right\rangle
\end{equation}

Now propagate it with $\bm N$, i.e. 

\begin{equation}
\begin{split}
&e^{-c_i^\dagger N_{ij} c_j} \left| \phi \right\rangle = \prod_\mu \bigg[ 1 + (e^{-n_\mu} -1 ) c_\mu^\dagger c_\mu \bigg] c_{\mu_1}^\dagger c_{\mu_2}^\dagger \left| 0 \right\rangle \\
&= e^{-n_{\mu_1}} e^{-n_{\mu_2}} c_{\mu_1}^\dagger c_{\mu_2}^\dagger \left| 0 \right\rangle ,
\end{split}
\end{equation}
where we simply note that by similar reasoning to the previous case, we would in equation (\ref{eq:proofAuxIdentity}) keep two terms corresponding to $\mu_1 \neq \mu_2$. If $\mu_1 = \mu_2$, then both sides are equal to zero due to Pauli's exclusion principle and the equality holds trivially. This reasoning clearly generalizes to an arbitrary superposition of many-particle states. Moreover, we proved the result for a product of two factors $e^{-\bm M} e^{-\bm N}$, but it is also easy to see that by successive changes of basis, we could extend our result to an arbitrary number of factors.

To complete our proof of the identity (\ref{eq:particularIdentity}) that is so crucial in formulating AFQMC, we use the auxiliar identity we just proved (\ref{eq:identity_prod_exps}).

\begin{equation}\label{eq:completeProof}
\begin{split}
&\Tr \bigg[ e^{-\sum_\nu c_\nu^\dagger \rho_\nu c_\nu} \bigg] = \Tr \bigg[ \prod_\nu e^{-c_\nu^\dagger \rho_\nu c_\nu } \bigg] \text{  since  } [\hat{n}_\mu, \hat{n}_\nu] = 0 \\
&= \prod_\nu ( 1 + e^{-\rho_\nu} ) = \text{det} [ \bm I + e^{-\bm M} e^{-\bm N} ], \quad\quad \qedsymbol
\end{split}
\end{equation}
where the last equality stems from the fact that the determinant of a diagonal matrix is just the product of the eigenvalues.

When applied to our problem, equation (\ref{eq:quadraticIdentity}) essentially makes the computation of the trace possible! Note that if we were to compute it na\"ively, we would soon run out of computer memory. The dimension of the Hilbert space of the Hubbard model is exponential in N (actually $4^N$), where $N$ is the number of lattice sites. The determinant is calculated for a matrix whose size is polynomial in $N$. 

Equation (\ref{eq:quadraticIdentity}) allows us to write the partition function (\ref{eq:Z_quadratic}) in computable form

\begin{equation}\label{eq:effectiveDensityMatrix}
\begin{split}
Z_{\bm h} &=  \Tr_{\bm h} \bigg[ (c_U)^{NL} \text{det} [ \bm M_\uparrow (\bm h)] \text{det} [  \bm M_\downarrow (\bm h) ] \bigg] \\
&\equiv  \Tr_{\bm h} \bigg[  \tilde{\rho}_{\text{eff}} (\bm h) \bigg] ,
\end{split}
\end{equation}
where the fermion matrices $\bm M_\sigma$ are defined in terms of the $\bm B$-matrices for a given spin $\sigma$ and a given HS-field $\bm h$:

\begin{equation}
\bm M_\sigma (\bm h) = \bm I + \bm B_{L,\sigma} ( h_L) \bm B_{L-1,\sigma} ( h_{L-1}) ... \bm B_{1\sigma} ( h_1)
\end{equation}

Equation (\ref{eq:effectiveDensityMatrix}) defines an effective density matrix (now a function!) $\tilde{\rho}_{\text{eff}} (\bm h)$ in the HS-field space.

The computable approximation of the distribution operator $\mathcal{P}$ corresponding to this partition function is

\begin{equation}
P(\bm h) = \frac{A}{Z_h} \det [ \bm M_{\uparrow}(\bm h) ] \det [ \bm M_{\downarrow}(\bm h) ] ,
\end{equation}
where $A = (c_U)^{NL}$ is a normalization constant. This is now a distribution function over configurations $\bm h$ since the problem is classical!

For the particular case of no interactions $U = 0$, we have that $\nu = 0$, and $\bm M_\sigma (\bm h)$ are constant matrices, independent of the HS-field. The Trotter-Suzuki approximation then becomes exact and the Hubbard Hamiltonian may be simulated exactly after evaluating $\bm M_\sigma (\bm h)$ a single time. No updates are required.

As a final remark, note that we managed to map a quantum problem to a classical problem in higher dimension. The degrees of freedom of the quantum problem correspond to the $i$ indices  of the $c$-operators. In our formulation, an additional imaginary time slice index $l$ was introduced, leading to a mapping that is not specific to the Hubbard model, but that actually applies very generally for any quantum system.

\subsubsection{Monte Carlo sampling of the HS-field}

The computational problem is now that of sampling configurations of the $\bm h$ field drawn from the distribution $P(\bm h)$ using \emph{Classical} Monte Carlo. The size of the state space has been (hopefully) reduced to $2^{NL}$ (assuming that $L < N$).

It remains to choose a dynamics and a sampling scheme. The simplest strategy to change from a configuration $\bm h$ to a new one $\bm h'$ is single spin-flip dynamics. We choose a random point $(l, i)$, and we flip the spin at that \say{site}

\begin{equation}
h_{l, i}' = - h_{l, i},
\end{equation}
keeping all others unchanged.

The most common scheme to ensure that the distribution of the accepted sample is $P(\bm h)$ is the Metropolis-Hastings algorithm.

After the warm-up steps, i.e. after we ensure that we are correctly sampling from the required distribution, we may perform measurements, waiting for some (Monte Carlo) time before each of them to ensure that the correlations within the sample are negligible. Let the total number of Monte Carlo steps (warm-up $W$ + measurement $M$) be $S = W + M$. The idea is that we run the algorithm for $W$ steps, before starting the measurements. Then we measure the state of the system every $2\tau$ steps, where $\tau$ is the correlation time, i.e. the time it takes for some representative correlation function to drop to $e^{-1}$ its original value.


\begin{algorithm}
\caption{Auxiliary Field Quantum Monte Carlo}
\label{afqmcSampling}
\begin{algorithmic}[5]
  \STATE Initialize HS field $\bm h$  \\
  \STATE Initialize hoppings $\bm K$  \\
  \STATE  $(h_{l, i}) = (\pm 1)_{l=1, i = 1}^{L, N}$
  \STATE $(l, i) \leftarrow (1, 1)$
  \FOR{$\text{step} = 1$ to $S$}
  \STATE \footnotesize{Propose new configuration by flipping a spin} \\ \normalsize{$h_{l, i}' = - h_{l, i}$} 
  \STATE \footnotesize{Compute the acceptance ratio $a_{l, i}$} \\
  \normalsize{$\frac{\text{det}[\bm M_\uparrow (\bm h')]\text{det}[\bm M_\downarrow (\bm h')]}{\text{det}[\bm M_\uparrow (\bm h)]\text{det}[\bm M_\downarrow (\bm h)]}$}
  \STATE \normalsize{Metropolis step}
  \STATE \footnotesize{Draw random number $r \in [0,1]$}
  \IF{$r \le \min(1, a_{l, i})$}
  \STATE $\bm h = \bm h'$
  \ELSE
  \STATE $\bm h = \bm h$
  \ENDIF
  \STATE Next site
  \IF{$i < N$}
  \STATE $l = l$ , $i = i +1 $
  \ELSE
  \IF {$l < L$}
  \STATE $l = l+1$ , $i = 1 $
  \ENDIF
  \IF {$l = L$}
  \STATE $l = 1$ , $i=1$
  \ENDIF
  \ENDIF
  \ENDFOR
\end{algorithmic}
\end{algorithm}

The Metropolis acceptance/rejection scheme leads to a rank-one update of the matrices $\bm M_\sigma (\bm h)$, which affords an efficient evaluation of the acceptance ratio $a_{l, i}$ \cite{hou_numerical_2009}.

Consider two matrices $\bm A_1$, $\bm A_2$ written in the form

\begin{equation}
\bm A_{1,2} = \bm I + \bm F \bm V_{1,2} ,
\end{equation}
where $\bm F$ is some matrix. $\bm V_{1,2}$ are diagonal and non-singular and differ only in the $(1,1)$ entry, so that

\begin{equation}
\bm V_1^{-1} \bm V_2 = \bm I + \alpha_1 \bm e_1 \bm e_1^T ,
\end{equation}
where $\bm e_1$ is a vector corresponding to the first column of the identity matrix $\bm I$, and

\begin{equation*}
\alpha_1 = \frac{V_2(1,1)}{V_1(1,1)} - 1
\end{equation*}

Then, $\bm A_2$ is clearly a rank-one update of $\bm A_1$.

\begin{equation*}
\begin{split}
\bm A_2 &= \bm I + \bm F \bm V_1 + \bm F \bm V_1 ( \bm V_1^{-1} \bm V_2 - \bm I ) \\
&= \bm A_1 + \alpha_1 ( \bm A_1 - \bm I ) \bm e_1 \bm e_1^T \\
&= \bm A_1 [ \bm I + \alpha_1 ( \bm I - \bm A_1^{-1} )\bm e_1 \bm e_1^T ]
\end{split}
\end{equation*}

Using the identity $\text{det}[\bm I + \bm x \bm y^T] = 1 + \bm y^T \bm x$ for any two column vectors, we may write the ratio of the determinants of matrices $\bm A_1$ and $\bm A_2$ as

\begin{equation}\label{eq:efRatio}
r_1 = \frac{\text{det}[\bm A_2]}{\text{det}[\bm A_1]} = 1 + \alpha_1 ( 1 - \bm e_1^T \bm A_1^{-1} \bm e_1 ) ,
\end{equation}
which reduces the computation of the ratio $r_1$ to computing the $(1,1)$ entry of $\bm A^{-1}$.

Now we generalize this idea for a sequence of matrices $\bm A_1, \bm A_2, ..., \bm A_i, ..., \bm A_n$ generated by successive rank-one updates: $\bm A_{i+1} = \bm I + \bm F \bm V_{i+1}, \, i = 1, 2, ..., n-1$, with

\begin{equation}
\bm V_i^{-1} \bm V_{i+1} = \bm I + \alpha_i \bm e_i \bm e_i^T \quad \alpha_i = \frac{\bm V_{i+1}(1,1)}{\bm V_i (1,1)} -1
\end{equation}

The Sherman-Morrison-Woodbury formula gives an expression for the inverse of $\bm A_2$ as a rank-one update of $\bm A_1^{-1}$.

\begin{equation}
\begin{split}
\bm A_2^{-1} &= \bigg[ \bm I - \frac{\alpha_1}{r_1} ( \bm I - \bm A_1^{-1} ) \bm e_1 \bm e_1^T  \bigg] \bm A_1^T \\
&= \bm A_1^{-1} - \frac{\alpha_1}{r_1} \bm u_1 \bm w_1^T ,
\end{split}
\end{equation}
where

\begin{equation*}
\bm u_1 = (\bm I - \bm A_1^{-1} ) \bm e_1 \quad \bm w_1 = (\bm A_1^{-1})^T \bm e_1
\end{equation*}

Using equation (\ref{eq:efRatio}), we find the updates

\begin{equation}
\begin{split}
r_i &= \frac{\text{det}[\bm M_{i+1}]}{\text{det}[\bm M_{i}]} = 1 + \alpha_i ( 1 - \bm e_i^T \bm A_i^{-1}  \bm e_i ) , \,\, \text{and} \\
\bm M_{i+1}^{-1} &= \bm M_i^{-1} - \frac{\alpha_i}{r_i} \bm u_i \bm w_i^T ,
\end{split}
\end{equation}
where $\bm u_i = (\bm I - \bm A_i^{-1} ) \bm e_i$ and $\bm w_i = (\bm A_i^{-1})^T \bm e_i$.

It is possible to generalize this procedure to compute the inverse of $\bm M_k$ as a rank$-(k-1)$ update of $\bm A_1^{-1}$:

\begin{equation}
\bm M_k^{-1} = \bm M_1^{-1} - \bm U_{k-1} \bm D_k \bm W_{k-1}^T ,
\end{equation}
where

\begin{equation}
\bm U_k = [ \bm u_1 , \bm u_2, ..., \bm u_{k-1} ] \quad \text{and} \quad \bm W = [ \bm w_1, \bm w_2, ..., \bm w_{k-1} ] ,
\end{equation}
and $\bm D_k = \text{diag}(\alpha_1 / r_1, \alpha_2 / r_2, ..., \alpha_{k-1} / r_{k-1})$.

\subsubsection{Making measurements}

In QMC simulations, physical observables are extracted by measuring them directly over the course of the sampling of the  configuration space. The single-particle (equal time) Green's Function is useful to obtain quantities such as density and kinetic energy. It turns out that it is simply the inverse of the $\bm M$-matrix that we already compute to obtain the acceptance ratio at each step.

\begin{equation}
\begin{split}
G_{ij}^\sigma &= \left\langle c_{i\sigma} c_{j\sigma}^\dagger \right\rangle_{\bm h} \\
&= \bigg( M_\sigma^{-1} (\bm h) \bigg)_{ij} \\
&= \bigg( [\bm I + \bm B_{L,\sigma} ( h_L ) \bm B_{L-1,\sigma} ( h_{L-1} ) ... \bm B_{1,\sigma} ( h_1 ) ]^{-1} \bigg)_{ij}
\end{split}
\end{equation}

The equal time Green's function is a fermion average for a given HS-field configuration \cite{santos_introduction_2003}. The corresponding thermal average is given by

\begin{equation}
\begin{split}
&\left\langle c_i c_j^\dagger \right\rangle = \frac{1}{Z} \Tr \bigg[ e^{-\beta \mathcal{H}} c_i c_j^\dagger \bigg] \\
&= \frac{1}{Z} \Tr_{\bm h} \Tr \bigg[ c_{i\sigma} c_{j\sigma}^\dagger \prod_{l=1}^L  B_{l, \uparrow}(\bm h_l) 
B_{l, \downarrow}(\bm h_l) \bigg],
\end{split}
\end{equation}

The density matrix $e^{-\beta\mathcal{H} }$ may be written as a trace over HS-field configurations of a product of L factors corresponding to each imaginary time slice. Recall equation (\ref{eq:Z_quadratic}): the partition function Z is just the trace over the Hilbert space of the aforementioned density matrix. Equivalently, it may be thought of as a trace over HS-field configurations of the effective density matrix $\hat{\rho}_{\text{eff}}(\bm h)$ defined in equation (\ref{eq:effectiveDensityMatrix}).

The Green's function is defined for fixed $\bm h$. Omitting the spin index $\sigma$, without loss of generality, we obtain 

\begin{equation}
\begin{split}
G_{ij}&\equiv \left\langle c_i c_j^\dagger \right\rangle_{\bm h} = \frac{\Tr[\bm B_L(h_l) \bm B_{L-1}(h_{l-1}) ... \bm B_1 (h_1) c_i c_j^\dagger]}{\tilde{\rho}_{\text{eff}}} \\
&= \frac{\Tr[\bm B_L \bm B_{L-1} ... \bm B_1 c_i c_j^\dagger] }{\Tr[\bm B_L \bm B_{L-1} ... \bm B_1] }
\end{split}
\end{equation}

The trace is evaluated by changing to a basis $\{\left|\alpha\right\rangle\}$, where $c_i$ is diagonal and then repeating the procedure for $c_j^\dagger$, now changing again to a basis $\{\left|\beta\right\rangle\}$, where $c_j^\dagger$ is diagonal. Using equation (\ref{eq:changeBasis2}), we obtain

\begin{equation}
\left\langle c_i c_j^\dagger \right\rangle_{\bm h} = \frac{\sum_{\alpha, \beta} \left\langle i | \alpha \right\rangle \left\langle \beta | j \right\rangle \Tr[c_{\alpha} c_\beta^\dagger \bm B_L \bm B_{L-1} ... \bm B_1] }{\Tr [\bm B_L \bm B_{L-1} ... \bm B_1] } 
\end{equation}

After taking the trace, (on the diagonal basis) the only nonzero contribution will be for $\alpha = \beta$. When $c_\alpha c_\beta^\dagger$ acts on the bra to its left, only that term survives in the sum since $c_\alpha$ is a diagonal operator in the basis $\{\left|\alpha\right\rangle\}$. On the other hand, the second equality in equation (\ref{eq:completeProof}) gives the contribution to the trace of the exponential of $c_\alpha^\dagger c_\alpha$ appearing in the $\bm B$-matrices. 

\begin{equation}\label{eq:traceExp}
\Tr [\bm B_L \bm B_{L-1} ... \bm B_1] = \prod_{\nu} ( 1 + e^{-\rho_\nu} ) ,
\end{equation}f
where $\{\left|\nu\right\rangle\}$ is the basis in which the product of the $\bm B$'s is diagonal.

\begin{equation}
\begin{split}
&\left\langle c_i c_j^\dagger \right\rangle_{\bm h} =\sum_{\alpha} | \alpha \rangle \langle i | \frac{ \Tr[c_{\alpha} c_\alpha^\dagger \bm B_L \bm B_{L-1} ... \bm B_1] }{\Tr [\bm B_L \bm B_{L-1} ... \bm B_1] } | j \rangle \langle \alpha | \\
&=\sum_{\alpha} | \alpha \rangle \langle i | \frac{ \Tr [( \mathbbm{1} - c_{\alpha}^\dagger c_\alpha) \bm B_L \bm B_{L-1} ... \bm B_1] }{\Tr [\bm B_L \bm B_{L-1} ... \bm B_1] } | j \rangle \langle \alpha | \\
&= \sum_{\alpha} | \alpha \rangle \langle i | 1 -  \frac{ \Tr [c_{\alpha}^\dagger c_\alpha e^{-\Delta \tau \hat{h}}]}{\Tr [e^{-\Delta \tau \hat{h}}] } | j \rangle \langle \alpha | \\
&= \sum_{\alpha} | \alpha \rangle \langle i | 1 -  \frac{1}{1 + e^{\Delta\tau \varepsilon_\alpha}} | j \rangle \langle \alpha | \\
&= \sum_{\alpha} | \alpha \rangle \langle i | \frac{e^{\Delta\tau \varepsilon_\alpha}}{1 + e^{\Delta\tau \varepsilon_\alpha}} | j \rangle \langle \alpha | \\
&= \sum_{\alpha} | \alpha \rangle \langle i | \frac{1}{1 + e^{-\Delta\tau \varepsilon_\alpha}} | j \rangle \langle \alpha | \\
&= \bigg[ \frac{1}{\bm B_L \bm B_{L-1} ... \bm B_{1}} \bigg]_{ij} ,
\end{split}
\end{equation}
where in the fourth equality we used an analogy with the Fermi function defined as 

\begin{equation}
f_\alpha = \frac{\Tr[e^{-\beta\mathcal{H}} \hat{n}_\alpha]}{\Tr[e^{-\beta\mathcal{H}} ]} = \big(1 + e^{\beta\varepsilon_\alpha} \big)^{-1}
\end{equation}
for $\mu = 0$ and with $\beta \mapsto \Delta \tau$. The product of $\bm B$-matrices was written as the exponential $e^{-\Delta \tau \hat{h}}$, which can be done because we have shown before that it is possible to diagonalize the product in a basis in which the trace amounts to the simple form of equation (\ref{eq:traceExp}).

An alternative way of arriving to this result is to note that in the expression we obtain in the second equality, only the term $\nu = \alpha$ from equation (\ref{eq:traceExp}) contributes \cite{santos_introduction_2003}, leading to the final result with $\rho_\alpha = \Delta\tau \varepsilon_\alpha$.

The electron density may be obtained from the Green function

\begin{equation}
\rho_{i\sigma} = \left\langle c_{i\sigma}^\dagger c_{i\sigma} \right\rangle = 1 - \left\langle c_{i\sigma} c_{i\sigma}^\dagger \right\rangle = 1 - G_{ii}^\sigma ,
\end{equation}

It is natural to think of averaging this over the lattice, and over the spins. This is justified by the fact that the Hubbard Hamiltonian is translationally invariant. Thus, $\rho_{i\sigma}$ should be independent of the spatial site. This statement is strict when exactly solving the model, but it becomes only approximate, i.e. valid only on average in our simulations. Thus, we take the average

\begin{equation}
\rho = \frac{1}{2N} \sum_\sigma \sum_{i=1}^N \rho_{i\sigma}
\end{equation}
in an attempt to reduce statistical errors.

One must pay attention to the symmetry of the model at hand, since a similar model for a disordered system including randomness would not be translationally invariant anymore. Moreover, it is implicit that $\rho_{i\sigma}$ is already averaged over the HS-field configurations that were sampled through the simulation.

The average kinetic energy is similarly obtained.

\begin{equation}
\begin{split}
\left\langle \mathcal{H}_K \right\rangle &= - t  \sum_{\left\langle i, j \right\rangle , \sigma} \left\langle ( c_{i\sigma}^\dagger c_{j\sigma} + c_{j\sigma}^\dagger c_{i\sigma} ) \right\rangle \\
&= t \sum_{\left\langle i, j \right\rangle , \sigma} ( G_{ij}^\sigma + G_{ji}^\sigma ) ,
\end{split}
\end{equation}
where the minus sign is due to the switching of the order of the operators bringing the $c^\dagger$ to the right.

\subsubsection{Correlation functions}

One of the most important goals of QMC simulations is to inspect the system for order of various types, and to find  associated phase transitions. This is done by computing correlation functions $C (j) $, measuring how correlated two sites separated by a distance $j$ are.

\begin{equation}
C(j) = \big\langle \mathcal{O}_{i+j} \mathcal{O}_{i}^\dagger \big\rangle - \langle \mathcal{O}_{i+j} \big\rangle\big\langle\mathcal{O}_{i}^\dagger \big\rangle ,
\end{equation}
where $\mathcal{O}$ is an operator corresponding to the order parameter of the phase transition. For example, we might be looking for magnetic order, in which case the relevant operators are $\mathcal{O}_i = n_{i\uparrow} - n_{i\downarrow} \, , \, \mathcal{O}_i^\dagger = n_{i\uparrow} - n_{i\downarrow}$, or superconductivity, where we would like to measure correlations in fermion pair formation: $\mathcal{O}_i = c_{i\downarrow} c_{i\uparrow} \, , \, \mathcal{O}_i^\dagger = c_{i\uparrow}^\dagger c_{i\downarrow}^\dagger$.

In general, we expect a high temperature disordered phase, for which correlations decay exponentially $C(j) \propto e^{-j/\xi}$, where $\xi$ is a characteristic length called the correlation length. At some point, there can be a transition to a low temperature phase, where $C(j) \propto m^2$, where $m$ is the order parameter for the transition. Right at the transition, that is at $T = T_c$, there might be singular behavior. In continuous phase transitions, the correlation length diverges $\xi \propto (T-T_c)^{-\nu}$, and the correlations decay slower (in fact algebraically): $C(j) \propto j^{-\eta}$, in an intermediate behavior between exponential decay and a constant. The \emph{critical} exponents $\nu$, and $\eta$ are characteristic of the transition, or more accurately, of the universality class it belongs to.

The behavior of all these quantities on finite lattices does not precisely correspond to the infinite system behavior. The tails of the functions, i.e. the $j\rightarrow \infty$ limit is not well captured. Finite-size scaling is a method to improve on these predictions.

To evaluate correlation functions we use Wick's theorem. Expectations of more than two fermion creation and annihilation operators reduce to products of expectations of pairs of creation and annihilation operators. For example, for pair order:

\begin{equation}
\big\langle C(j) \big\rangle = \big\langle c_{i+j, \downarrow} c_{i+j, \uparrow} c_{i, \uparrow}^\dagger c_{i, \downarrow}^\dagger \big\rangle
\end{equation}

How would one measure a correlation function experimentally? Fortunately, there is a quantity that is easy to measure called structure factor, which is just the Fourier transform of the correlation function

\begin{equation}
S(q) = \sum_j e^{iqj} C(j) 
\end{equation}

The accuracy of QMC simulations can be evaluated by comparing the results for correlation functions with the corresponding structure factors, which can be measured experimentally.
\cleardoublepage
