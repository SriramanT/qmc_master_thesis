
\fancychapter{Auxiliary Field Quantum Monte Carlo}
\label{cap:afqmc}

\slshape

We start with a brief review of the Monte Carlo method in statistical physics, and then explain how the fundamental concepts we present can be generalized to study quantum-many fermion systems.
Focusing on the auxiliary field method, we discuss the mapping to the free-fermion problem that reduces the problem to the evaluation of determinants, and show how an efficient algorithm can be designed by successive global updates of the free-fermion Green's functions.
Then, we explain how to control the instabilities in this update scheme.
In particular, we explain how to circumvent low temperature instabilities due to the ill-conditioning of the matrices representing the free-fermion propagators.
We also show that this problem has to do with the coexistence of significantly different energy scales, and tends to get worse as the size increases.
Finally, we compute the estimators allowing the measurement of some relevant observables for \acp{TMDNR}.

\normalfont

\section{Monte Carlo Method in Statistical Physics}
\label{sec:classical_mc}

Monte Carlo methods form the largest and arguably most useful class of numerical methods used to approach statistical physics problems.
Statistical physics often deals with computing quantities that describe the behavior of condensed matter systems.
The main difficulty one faces when doing so has to do with the collective nature of these systems.
Many identical components comprise them, and while the equations that govern the behavior of the whole may be easy to write down, their solution is in general a remarkably laborious mathematical problem.
It is both the sheer number of equations and the coupling between them that deems the task of finding an exact solution either very tough or even impossible.
Concomitantly, the exponentially large number of possible configurations of the typical condensed matter system can be daunting.
Thus, it is rather striking that we are able to describe a system that is governed by a macroscopically large number of equations in terms of only a few variables.
The loss of information in doing so is only apparent.

The statistical description is so effective because most of the possible states of the system are extremely improbable when compared to the relevant very narrow part of configuration space.
The success of the field is largely attributed to the averaging out that naturally occurs when measuring a property of a macroscopic system.
%Since analytical solutions are more often than not hopeless, these problems are solved numerically.
%Numerical results give valuable information lying between theory and experiment, and connecting them.
%It is not even clear whether an analytical solution would be of any use in many cases, and a statistical treatment often allows us to study more effectively the key properties of a system.
%Numerically, this is done using Monte Carlo.

Suppose you try to sample uniformly from the probability distribution of all possible configurations of one of the aforementioned systems.
Changes are your algorithm will not end before the Universe does.
This is the computational complexity hurdle.
A related issue is that of finite size effects.
We are far from being able to simulate a macroscopically sized system. 
At best we can simulate a system that has only a minuscule fraction of the size of a real system.
Amazingly there are techniques that allow us to efficiently extract information out of relatively small scale simulations.
Nonetheless, increasing the system size systematically improves the reliability of a simulation.
Thus, it is important to design efficient algorithms to probe larger  systems in a fixed computer time frame.

The law of large numbers affords an approximation to integrals which can be written as an expectation of a random variable. Upon drawing enough independent samples from the corresponding distribution, the sample mean gets arbitrarily close to the integral at stake.

\begin{equation}\label{eq:int_mean}
\mathbb{E} [f(X)] = \int dx f(x) p(x),
\end{equation}
where $p(x)$ is the distribution of $X$. 

We could simply draw $M$ independent and identically distributed samples $x_{1,...M}$ from $p(x)$ and approximate the integral as

\begin{equation}
\frac{1}{M} \sum_{k=1}^M f (x_k) , 
\end{equation}
which in most cases converges to the desired expectation, as long as $M$ is large enough. How large?

\begin{equation}\label{eq:variance}
\text{Var}\bigg( \frac{1}{M} \sum_{k=1}^M f(x_k) \bigg) = \frac{1}{M} \text{Var}\bigg( f(x_1) \bigg) \propto \mathcal{O}\bigg(\frac{1}{M}\bigg)
\end{equation}

Thus, the correction to the sample mean is of order $\mathcal{O}(\frac{1}{\sqrt{M}})$ as long as $\text{Var}\big( f(x_1) \big) \sim 1$, which can be achieved by using importance sampling, a variance reduction technique we will shortly discuss.

But how do we sample from an arbitrary distribution $p(X)$? The idea is to first make an educated choice of a Markov Chain with the prescribed stationary distribution from which we ultimately desire to sample from, $p(X)$. After a sufficiently high number of steps, a Markov Chain Monte Carlo (MCMC) algorithm generates samples from the target distribution. Imposing some conditions on this Markov Chain, namely that it should be irreducible, aperiodic and positive recurrent, the ergodic theorem guarantees that the empirical measures of the aforementioned sampler approach the target stationary distribution. Another important condition to impose on this Markov Chain is detailed balance. Let the transition matrix be $\bm P = [P_{\mu \rightarrow \nu}]$, and the state space $\Omega$ be $\{\pi_\mu | \mu=1, ..., |\Omega| \}$, where $|\Omega|$ is the total number of possible states. Then, the condition of detailed balance is defined for all $\mu, \nu$ as

\begin{equation}\label{eq:detBal}
\pi_\mu P_{\mu \rightarrow \nu} = P_{\nu \rightarrow \mu} \pi_\nu
\end{equation}

Consider a system in state $\mu$ that makes transitions to state $\nu$ at a rate $R_{\mu \rightarrow  \nu}$ (that specifies the system's dynamics) and vice-versa.
The probability that a system is in state $\mu$ at time $t$, $p_\mu (t)$, such that $\sum_\mu p_\mu (t) = 1$, is given by the master equation(s):

\begin{equation}\label{eq:master}
\frac{d p_\mu}{dt} = \sum_\nu \big[ p_\nu (t) R_{\nu \rightarrow \mu} - p_\mu (t) R_{\mu \rightarrow \nu} \big] \quad \forall \mu \in \Omega
\end{equation}

The equilibrium occupation probabilities at finite temperature $T$ follow the Boltzmann distribution.

\begin{equation}
\pi_\mu = \lim_{t \rightarrow \infty} p_\mu (t) = \frac{1}{Z} e^{ - E_\mu / k_B T} ,
\end{equation}
where $E_\mu$ is the energy of state $\mu$, $k_B$ is Boltzmann's constant, and $Z$ is the partition function, from which we can extract thermodynamic functions in terms of expectations of physical quantities $\left\langle Q \right\rangle$, and response functions in terms of their variance $\sigma_Q^{\,\, 2}$.

Imposing the condition of stationarity on Eq.(\ref{eq:master}), $d_t p_\mu = 0$ , and noting that $ P_{\mu\rightarrow \nu} = R_{\mu\rightarrow \nu}  dt$, we obtain the equilibrium condition

\begin{equation}\label{eq:equilibrium}
\sum_\nu \pi_\mu P_{\mu \rightarrow \nu} = \sum_\nu P_{\nu \rightarrow \mu} \pi_\nu \iff \pi_\mu \sum_\nu P_{\mu \rightarrow \nu} = \sum_\nu P_{\nu \rightarrow \mu} \pi_\nu \iff \pi_\mu = \sum_\nu P_{\nu \rightarrow \mu} \pi_\nu
\end{equation}

This condition is enough to ensure convergence to an equilibrium of the dynamics of the Markov process.
However, it does not guarantee that the reached distribution is our desired one, $\bm \pi$, after running the process for long enough.
The probability of a state evolves according to

\begin{equation}
\pi_\nu ( t + 1 ) = \sum_\mu P_{\mu\rightarrow\nu}  \pi_\mu ( t ) \iff \bm \pi ( t + 1 ) = \bm P \bm \pi ( t )
\end{equation}

The stationary distribution of a Markov chain obeys

\begin{equation}
\bm \pi ( \infty ) = \bm P \bm \pi ( \infty ) ,
\end{equation}
however, condition (\ref{eq:equilibrium}) also allows  for limit cycles of length $n$, where $\bm \pi$ rotates around a number of configurations:

\begin{equation}
\bm \pi ( \infty ) = \bm P^n \bm \pi ( \infty ) ,
\end{equation}
where $\bm P^n$ is the n-th power of $\bm P$.

Detailed balance is a stronger requirement than the equilibrium condition, which eliminates limit cycles, thus ensuring that our sampler draws configurations from the desired distribution.
Intuitively, detailed balance corresponds to incorporating time-reversal symmetry in a simulation.
The condition imposes a constraint on the Markov transition probabilities:

\begin{equation}\label{eq:markovCondition}
\frac{P_{\mu\rightarrow\nu}}{P_{\nu\rightarrow\mu}} = \frac{\pi_\nu}{\pi_\mu} = e^{-\beta ( E_\nu - E_\mu ) }
\end{equation}

Crucially, Monte Carlo methods employ \emph{importance sampling}.
It turns out that we can improve upon our estimate of $\mathbb{E} [f(X)]$ by reducing the variance of the estimator. If we introduce a separate distribution $q(x)$, and define a weight function as $w(x) = p(x)/ q(x)$, we can rewrite equation (\ref{eq:int_mean}):

\begin{equation}
\mathbb{E} [f(X)] = \int dx f(x) q(x) w(x) = \mathbb{E} [f(Y) w(Y)],
\end{equation}
with $Y \sim q$, i.e. the random variable $Y$ follows the distribution $q(Y)$.

It appears as though we didn't gain anything. However, by choosing $q$ wisely, we can actually reduce the variance we computed in Eq.(\ref{eq:variance}):

\begin{equation}
\text{Var}\bigg( \frac{1}{M} \sum_{k=1}^M f(y_k) w(y_k) \bigg) = \frac{1}{M} \text{Var}\bigg( f(y_1) w(y_1) \bigg)
\end{equation}

Since we didn't make any assumptions about $q(Y)$, it may be chosen so as to minimize the variance, hence the error of the Monte Carlo estimator, improving the approximation of the expectation. However, note that the error remains proportional to $\frac{1}{\sqrt{M}}$.
In practice, we devise a method to select the portion of state space which contains states contributing more significantly to the average.
This procedure ensures that $\text{Var}\big( f(y_1) w(y_1) \big) \sim 1$, improving the efficiency of our sampler.
The choice of the weight function translates to the averaging process by changing the estimator.
Explicitly computing the average

\begin{equation}
\left\langle Q \right\rangle = \frac{ \sum_\mu Q_\mu e^{-\beta E_\mu} }{ \sum_\mu e^{-\beta E_\mu}}
\end{equation}
is only tractable for very small systems.
In practice, we choose a subset of M states $\{\mu_1, \mu_2, ..., \mu_M \} $, and estimate the average as

\begin{equation}
Q_M = \frac{ \sum_{i=1}^M Q_{\mu_i} \pi_{\mu_i}^{-1} e^{ -\beta E_{\mu_i} } }{ \sum_{j=1}^M \pi_{\mu_j}^{-1} e^{ -\beta E_{\mu_j} }  }
\end{equation}

The estimate improves as $N$ increases, and when $N\rightarrow \infty$, $Q_M \rightarrow \left\langle Q \right\rangle$.
The accuracy of the estimator depends on the choice of the probabilities $\bm \pi$, which is related to the aforementioned variance.
For example, if $\bm \pi$ corresponds to the uniform distribution, i.e. $\pi_\mu = \frac{1}{| \Omega |} \forall \mu \in \Omega$, we have

\begin{equation}
Q_M = \frac{ \sum_{i=1}^M Q_{\mu_i} e^{ -\beta E_{\mu_i} } }{ \sum_{j=1}^M e^{ -\beta E_{\mu_j} }  } ,
\end{equation}
which turns out to be a poor choice since most of the visited states contribute negligibly to the average, leading to an inaccurate estimate.
The sum is dominated by a small subset of states, which we would like to access.
The idea of the Quantum (Classical) Monte Carlo method is to simulate the random quantum (thermal) fluctuations of a system, as it oscillates between states in a given time frame \cite{newman_monte_1999}. Instead of visiting these states uniformly, the most relevant part of the phase space is sampled more frequently, overcoming the seemingly exponential complexity of computing a sample mean numerically.
Even though only a small fraction of the system's states are sampled, we then obtain an accurate estimate of physical quantities of interest, namely energy, and correlation functions. This is implemented via a proposal-acceptance scheme.

To exploit the freedom given by condition (\ref{eq:markovCondition}), we note that we can always introduce a non-zero \say{stay-at-home} probability $P_{\mu \rightarrow \mu} \in [0, 1] $.
Regardless of its value, detailed balance is satisfied.
Similarly, any adjustment in $P_{\mu\rightarrow \nu}$ must be compensated by changing $P_{\nu\rightarrow \mu}$ to preserve their ratio.
Break the transition probability into a selection probability and an acceptance ratio, respectively:

\begin{equation}
\frac{P_{\mu\rightarrow\nu}}{P_{\nu\rightarrow\mu}}= \frac{S_{ \mu\rightarrow\nu} A_{\mu\rightarrow\nu}}{S_{ \nu\rightarrow\mu} A_{\nu\rightarrow\mu}}
\end{equation}

The Markov process now consists of generating a chain of states according to $S_{ \mu\rightarrow\nu}$, which are then accepted or rejected depending on $A_{\mu\rightarrow\nu}$.
Since we want to make the algorithm as efficient as possible, we want to make the acceptance ratio as close to one as possible to avoid useless steps.	
The most common way to do this is to fix the largest of them to one, and adjust the other accordingly.
The acceptance ratio will be close to one more often if $S_{ \mu\rightarrow\nu}$ includes most of the dependence of $P_{\mu\rightarrow\nu}$ on the characteristics of the states $\mu, \nu$.
Ideally, states would always be selected with the correct transition probability, and the acceptance ratio would be fixed to unity.
Good algorithms approach this situation, and much effort has been directed at optimizing them to do so.
By far, the most common sampling scheme choice is the Metropolis-Hastings algorithm, which we now  describe.

We select the transition probability to be uniform, and impose detailed balance through the choice of the acceptance ratios:

\begin{equation}
\frac{ P_{\mu\rightarrow\nu }}{ P_{\nu\rightarrow\mu }} = \frac{ A_{\mu\rightarrow\nu }}{ A_{\nu\rightarrow\mu } } = e^{-\beta ( E_\nu - E_\mu )}
\end{equation}

Suppose that $E_\mu < E_\nu $.
Then, $A ( \nu \rightarrow \mu ) > A ( \mu \rightarrow \nu ) $, and since only the acceptance ratio is fixed, we may freely set $A ( \nu \rightarrow \mu ) = 1$, which fixes $A ( \mu \rightarrow \nu ) = e^{-\beta ( E_\nu - E_\mu ) }$.
This choice maximizes the efficiency of the algorithm.
In short, we propose a random new state uniformly, and then we accept it with probability $A_{\mu\rightarrow \nu} = \min (1,  e^{-\beta ( E_\nu - E_\mu )})$.

Before we can use the states generated by our sampler to measure averages of physical quantities, we must reach the stationary distribution of the Markov process.
We consider this condition to be satisfied after a time $\tau_{\text{eq}}$, measured in steps of the algorithm.
When we consider a lattice model with a discrete set of states at each site $i = 1, 2, ..., N$, we say that a \emph{sweep} is completed whenever $N$ Monte Carlo steps are performed.
Thus, the number of \say{warm-up} sweeps is of order $W \sim \tau_{\text{eq}} / N$.

Before running a simulation, we need to decide how many sweeps we need to get an accurate estimate of the average.
The problem is that we need uncorrelated samples to average over.
To clarify, let us choose a specific model.
The paradigmatic model of statistical physics is the Ising model, a classical model of a magnet, which consists of considering spins-$1/2$ on a lattice, interacting only with their nearest neighbors.
Since each spin can only take on two values, say $\pm 1$, there are $2^N$ possible states.
The Hamiltonian reads

\begin{equation}
H = - J \sum_{\left\langle i, j \right\rangle } s_i s_j - B \sum_i s_i ,
\end{equation}
where $\left\langle i, j \right\rangle$ means that $i, j $ are nearest neighbors on the lattice.

A simple strategy to sample configurations of the Ising model is single-spin-flip dynamics.
We start with a random configuration of the spins, and then propose new configurations at each step by flipping a single spin at a given site.
A sweep is completed after we propose a spin flip at every site on the lattice.

Consecutive configurations generated by this chain differ only slightly.
Thus, it takes some time for the system to reach a configuration which is significantly different from the initial one.
This characteristic time is called the correlation time $\tau_c$.
A rigorous manner to estimate $\tau_c$ is through the time-displaced auto-correlation function associated to some quantity being measured.
An example of a relevant quantity for the case of the Ising model is the magnetization per site:

\begin{equation}
m = \frac{1}{N} \sum_i s_i
\end{equation}

Its associated time-displaced auto-correlation is

\begin{equation}
\chi_m ( t ) = \int dt' \bigg( m ( t' ) - \left\langle m \right\rangle \bigg) \bigg( m ( t' + t ) - \left\langle m \right\rangle \bigg) = \int dt' \bigg( m ( t' ) m ( t' + t ) - \left\langle m \right\rangle^2 \bigg)
\end{equation}
giving a measure of how correlated two measurements of the magnetization separated by a simulation time $t$ are.

The typical time-scale on which $\chi_m (t)$ falls off is a measure of the correlation time of the simulation.
In particular, at long times it falls off exponentially.
The definition of $\tau_c$ stems from this characteristic long-time behavior: $\chi_m (t) \sim e^{-t / \tau_c}$.
In practice, after waiting for $2\tau_c$, the measurements are virtually uncorrelated.
Let $A$ be the number of sweeps roughly  corresponding to $2\tau_c$ steps.
Then, if we make $S$ sweeps of the lattice during the simulation, the number of independent measurements (i.e. with $A$ sweeps between them) is

\begin{equation}\label{eq:nMeas}
M = \frac{S - W}{A}
\end{equation}

There are many ways to estimate $\tau_c$ from $\chi_m (t)$.
The simplest consists of making an exponential fit in a given range of times.
However, this might be unreliable since the estimate depends strongly on the chosen range.
An alternative is to compute the \say{integrated} correlation time:

\begin{equation}
\int_0^\infty dt \frac{\chi_m ( t ) }{\chi_m ( 0 ) } = \int_0^\infty dt e^{-t / \tau_c} = \tau_c  ,
\end{equation}
which is less sensitive, but not perfect either since the error that is introduced when the assumption  that \say{long-time} behavior has been reached is arbitrary and introduces an uncontrolled error.
Moreover, the very long-time behavior of the auto-correlation is rather noisy and must be excluded.

Using measured data for the magnetization at evenly-spaced times, we may construct the time-displaced auto-correlation function up to an unimportant constant, which does not affect the estimate of the correlation time:

\begin{equation}
\chi_m (t) = \frac{1}{ t_{\text{max}} - t } \sum_{t' = 0}^{t_{\text{max}} - t } m (t') m(t' + t) - \frac{1}{ t_{\text{max}} - t } \sum_{t' = 0}^{t_{\text{max}} - t } m (t') \frac{1}{ t_{\text{max}} - t } \sum_{t' = 0}^{t_{\text{max}} - t } m(t' + t) ,
\end{equation}
where $t_{\text{max}}$ is the total simulation time in MC steps.

One should be careful when using this expression at very long times.
As $t$ approaches $t_{\text{max}}$, the upper limit of the sums decreases, and the integration interval becomes narrower.
Since $m (t)$ fluctuates randomly at very long times, the statistical error associated to $\chi_m (t)$ becomes more prominent as $t$ approaches $t_{\text{max}}$.
This turns out not to be problematic since typical simulations run for many correlation times.
Thus, the tails of the auto-correlation may safely be neglected because the correlations will have already vanished, by definition.

To finish our discussion on the issue of computing the time-displaced correlator, we note that if we have a total of $N_s$ samples of, for instance, magnetization data, the complexity of computing $\chi_m$ is $\mathcal{O}(N_s^2)$.
It is possible to speed up this process by computing its Fourier transform $\tilde{\chi}_m(\omega)$, and inverting to recover $\chi_m (t)$.
This can be done via a standard \ac{FFT} algorithm in $\mathcal{O}(2 N_s \log N_s )$ flops.
To do this, we apply the following trick

\begin{equation}
\begin{split}
\tilde{\chi}_m ( \omega ) &= \int dt e^{i\omega t} \int dt' \bigg( m ( t' ) - \left\langle m \right\rangle \bigg) \bigg( m ( t' + t ) - \left\langle m \right\rangle \bigg) \\
&= \int dt \int dt' e^{-i\omega t'} \bigg( m ( t' ) - \left\langle m \right\rangle \bigg) e^{i\omega ( t' + t )} \bigg( m ( t' + t ) - \left\langle m \right\rangle \bigg) = \tilde{m}' (\omega) \tilde{m}' (- \omega) = | \tilde{m}' (\omega) |^2 ,
\end{split} 
\end{equation}
where $\tilde{m}' (\omega)$ is the Fourier transform of $m' (t) = m(t) - \left\langle m \right\rangle$\footnote{The only difference between $\tilde{m}' (\omega)$ and $\tilde{m} (\omega)$, is that $\tilde{m}' (0) = 0$, while $\tilde{m} (0) \neq 0$. Thus, one can also compute $\tilde{m} (\omega)$ and then set its $\omega = 0$ component to zero.}.

In practice, when implementing a MC algorithm, we take a measurement say every sweep (which can be less than a correlation time), and then compute the time-displaced correlator at the end of the simulation to estimate the correlation time.
How can we estimate the error in the mean of the $N_s$ correlated samples without knowing $\tau_c$ in advance?
Take again magnetization measurements.
Suppose your $N_s$ samples were  independent.
The standard deviation of their mean would be well known:

\begin{equation}\label{eq:errorUncorr}
\sigma = \sqrt{ \frac{ \frac{1}{N_s} \sum_{i=0}^{N_s} ( m_i - \overline{m} )^2 }{N_s - 1} } = \sqrt{ \frac{1}{N_s - 1} ( \overline{m^2} - \overline{m}^2 } )
\end{equation}

Intuitively, to get the correct result, we could simply replace $N_s$ by $M$, as computed in Eq.(\ref{eq:nMeas}), in the last step.
This is because the mean shouldn't change very much when including the correlated configurations with nearly the same magnetization.
However, the number of uncorrelated samples is certainly smaller than $N_s$, and can be estimated to be $M$ by auto-correlation studies.

As shown in \cite{muller-krumbhaar_dynamic_1973}, if the samples are separated by a time interval $\Delta t$,  the correct expression is

\begin{equation}\label{eq:errorCorr}
\sigma = \sqrt{ \frac{1 + 2\tau_c / \Delta t}{N_s - 1} ( \overline{m^2} - \overline{m}^2 )  } ,
\end{equation}
which reduces to Eq.(\ref{eq:errorUncorr})  when $\Delta t \gg \tau_c$, since in that case the samples are virtually  uncorrelated.
Also, we now have a rigorous justification for estimating the amount of time between uncorrelated samples as $2\tau_c$, since for much longer times, the samples become uncorrelated.

Often, we work with heavily correlated samples, so that instead we have $\Delta t \ll \tau_c$.
In this limit, the 1 in the numerator of Eq.(\ref{eq:errorCorr}) can be neglected, and noting that the number of sweeps between measurements corresponding to $\Delta t$ is $\Delta S = (S - W ) / N_s$, we have

\begin{equation}\label{eq:errorMC}
\sigma \approx \sqrt{ \frac{A}{S - W} ( \overline{m^2} - \overline{m}^2 )  }  ,
\end{equation}
the same result we would obtain by simply replacing $N_s$ by $M$.
What we found, in rigorous terms, was that the presence of many correlated samples does not significantly change the sample mean if $\Delta t \ll \tau_c$.
If we take enough correlated samples, their influence on the sample mean averages out.

Eq.(\ref{eq:errorMC}) has the advantage of being independent of $\Delta t$, which allows us to choose $\Delta t$ freely, without affecting the final error.
This is handy since it allows us to choose $\Delta t$ small so as not to lose data.

\section{Theoretical framework}
\label{subsec:theoreticalFramework}

Auxiliary-field, or Determinant \acs{QMC} is a simulation method that is commonly used to simulate the Hubbard model, allowing one to capture the elusive effects of electron correlations, for example in the two-dimensional graphene-like nanostructures we are concerned with.
In general, the sign problem deems the algorithm exponentially complex in the size of the system and in inverse temperature, but it is possible to overcome this hurdle for a class of models, namely the Hubbard model on a bipartite lattice at half filling ($\mu = 0$ in our conventions).
In fact, many interesting phenomena occur at half filling, for example magnetic ordering and the Mott metal-insulator transition.
The difficulty lies in computing the nearly vanishing average of a random variable $X$ with comparatively large variance, i.e. $\sigma_X / \left\langle X \right\rangle \gg 1$.

Ultimately, we seek a computable approximation of the projection operator $\mathcal{P}$ defined in equation (\ref{eq:projection}).
As we shall see, it is found by using a discrete Hubbard-Stratonovich transformation.
This transformation introduces an auxiliary field (consisting basically of Ising spins), and we use Monte Carlo to sample configurations from the distribution corresponding to this \emph{classical} configuration space.

\subsection{Trotter-Suzuki Decomposition}
\label{subsec:trotter}

In section \ref{sec:exactSolutions}, we found exact solutions for particular instances of the Hubbard model by finding a closed form for the partition function \cite{hou_numerical_2009}. When devising a numerical method, a good sanity check is to verify that it satisfactorily approximates the partition function since it is the quantity that can be used to obtain  any observable.
Computing the partition function of a quantum system in equilibrium

\begin{equation}\label{eq:partitionBeta}
Z_\beta = \text{Tr} \big( e^{-\beta \mathcal{H} } \big) = \sum_{\alpha} \left\langle \psi_\alpha | e^{-\beta \mathcal{H} } | \psi_\alpha \right\rangle
\end{equation}
is equivalent to studying its \emph{imaginary} time evolution.
The inverse temperature $\beta$ represents the imaginary time $\tau = it$, and $Z_\beta$ may be simply thought of as the wave function of the analogous quantum system at imaginary time $\beta$.
%To see this, recall that quantum states evolve according to
%
%\begin{equation}
%\left| \psi (t) \right\rangle = e^{-i \mathcal{H} t } \left| \psi (0) \right\rangle \iff \left| \psi (\tau) \right\rangle = e^{-\tau \mathcal{H} } \left| \psi (0) \right\rangle,
%\end{equation}
%
%Taking the scalar product with a position eigenstate $ \left\langle \bm x \right|$, we obtain $\psi (\bm x, \beta) = \left\langle \bm x | \psi (\beta) \right\rangle$.
%Using the closure relation $\int d\bm y \left| \bm y \right\rangle  \left\langle \bm y \right| = 1$, we get
%
%\begin{equation}
%\psi ( \bm x , \tau ) = \int d\bm y \left\langle \bm x | e^{-\tau \mathcal{H}} | \bm y \right\rangle \psi (\bm y, 0)
%\end{equation}
%
%The wave function at position $\bm x$ and time $\tau$ may be obtained by this equation as long as we know the wave function at $\tau = 0$, $\psi (\bm y, 0)$ for all points in space $\bm y$. The Green's function
%
%\begin{equation}
%G ( \bm x, \tau | \bm y, 0 ) \equiv \left\langle \bm x | e^{-\tau \mathcal{H}} | \bm y \right\rangle ,
%\end{equation}
%satisfies the Schr\"odinger equation, subject to the initial condition $\psi (\bm y, 0) = \delta (\bm y - \bm x )$. Thus, it corresponds to the probability of presence at $\bm x, \tau$ of a wave packet centered at $\bm y$ at $\tau = 0$. Thus, solving the Schr\"odinger equation is  analogous to solving the diffusion equation (that in turn one may obtain as the continuum limit of a random walk). We may write $G$ as a linear combination of the eigenstates of the Hamiltonian
%
%\begin{equation}
%G (\bm x, \tau | \bm y, 0) = \sum_\alpha \psi_\alpha^\star (\bm y) \psi_\alpha (\bm x) e^{-E_\alpha \tau}  ,
%\end{equation}
%so that
%
%\begin{equation}\label{eq:psiZ}
%\psi ( \bm x, \tau ) = \sum_\alpha \int d\bm y \, \psi_\alpha^\star (\bm y) \psi_\alpha (\bm x) e^{-E_\alpha \tau} \delta ( \bm y - \bm x ) = \sum_\alpha \psi_\alpha^\star (\bm x) \psi_\alpha (\bm x) e^{-E_\alpha \tau} = \sum_{\alpha} \left\langle \psi_\alpha | e^{-\tau \mathcal{H} } | \psi_\alpha \right\rangle
%\end{equation}
%where we immediately notice a striking similarity with equation (\ref{eq:z_asEigen}), by making $\psi (\bm x, \tau) \mapsto Z_\beta$.
In fact, for a zero temperature system, projective methods use this same principle to find the ground state.
In that case, the partition function strictly corresponds to the ground state wave function when $\tau \rightarrow \infty$ (in practice, one takes $\tau = \Theta$ large enough).
%As we can see from Eq.(\ref{eq:psiZ}), the higher energy states are exponentially suppressed in this limit, so that $\psi (\bm x, \tau) \rightarrow |\psi_0 (x)|^2 e^{-E_0 \tau}$.
%The initial condition does not even have to be $\psi ( \bm y, 0) = \delta ( \bm y - \bm x )$.
%More generally, as we saw for diffusion \acs{QMC}, we only require $\int d\bm y \, \psi_0^\star ( \bm y ) \psi ( \bm y, 0 ) = C \neq 0$, and we obtain $\psi (\bm x, \tau\rightarrow \infty) \rightarrow C \psi_0 (x) e^{-E_0 \tau}$.
 
Eq.(\ref{eq:partitionBeta}) is not very amenable to numerical computation since it contains an exponential of a sum of non-commuting operators $e^{-\beta (\mathcal{H}_K + \mathcal{H}_V)}$ as per Eq.(\ref{eq:def_energies}). 
The exponential is not factorizable and involves computing an infinite number of commutators containing these two operators, as per the Zassenhaus formula, valid for any two generic operators $X$ and $Y$\footnote{This is just the inverse of the well known Baker–Campbell–Hausdorff formula commonly used in quantum mechanics.}:

\begin{equation}\label{eq:zassenhaus}
e^{\delta (X+Y)}=e^{\delta X} e^{\delta Y} e^{-{\frac {\delta^{2}}{2}}[X,Y]} e^{{\frac {\delta^{3}}{6}}(2[Y,[X,Y]]+[X,[X,Y]])}  e^{{\frac {-\delta^{4}}{24}}([[[X,Y],X],X]+3[[[X,Y],X],Y]+3[[[X,Y],Y],Y])} \, ... , 
\end{equation}
where $\delta \in \mathbb{C}$ is an expansion parameter.
The Trotter-Suzuki decomposition leads to the sought approximate factorization that is used to approximate the partition function.
Dividing the imaginary time interval $[0, \beta ]$ into $L$ equal sub-intervals of small width $\Delta \tau = \beta / L$:

\begin{equation}\label{eq:partBreak}
Z =  \Tr \bigg( \prod_{l=0}^{L-1} e^{-\Delta\tau \mathcal{H} } \bigg) ,
\end{equation}
we obtain a form that is more amenable to computation.
Actually, it also arises naturally by writing the matrix elements of the projection operator $\mathcal{P}$ as path integrals:

\begin{equation}
\left\langle \psi | e^{-\beta \mathcal{H} } | \psi' \right\rangle = \sum_{\left| \psi_1 \right\rangle, \left| \psi_2 \right\rangle,..., \left| \psi_{L-1} \right\rangle }  \left\langle \psi | e^{-\Delta \tau \mathcal{H} } | \psi_1 \right\rangle \left\langle \psi_1 | e^{-\Delta \tau \mathcal{H} } | \psi_2 \right\rangle ... \left\langle \psi_{L - 1} | e^{-\Delta \tau \mathcal{H} } | \psi' \right\rangle 
\end{equation}

Then, the partition function only selects paths that are periodic in imaginary time:

\begin{equation}
Z = \text{Tr} \big( e^{-\beta \mathcal{H} } \big) = \sum_{\left| \psi_0 \right\rangle} \left\langle \psi_0 | e^{-\beta \mathcal{H}} | \psi_0 \right\rangle = \sum_{\{ \left| \psi_l \right\rangle \}} \prod_{l = 0}^{L - 1} \left\langle \psi_{l} | e^{-\Delta \tau \mathcal{H} } | \psi_{l+1} \right\rangle \delta_{0, L} = \Tr \bigg( \prod_{l=0}^{L-1} e^{-\Delta\tau \mathcal{H} } \bigg) ,
\end{equation}
where we have $\left| \psi_L \right\rangle = \left| \psi_0 \right\rangle$, and we  recover the result of Eq.(\ref{eq:partBreak}) by simply reorganizing the summations over $\{ \left| \psi_l \right\rangle \}$, so as to make appear closure relations, resulting in unit operators.

The \say{Trotter breakup} follows from truncating equation (\ref{eq:zassenhaus}), and keeping only the first order term in $\Delta \tau$.

\begin{equation}\label{eq:Z_propagator}
Z = \Tr \bigg( \prod_{l=0}^{L-1} e^{-\Delta\tau \mathcal{H}_{K} } e^{-\Delta\tau \mathcal{H}_{V} } \bigg) + \mathcal{O}(\Delta \tau^2) 
\end{equation}

The kinetic energy term is quadratic in the fermion operators, and is spin-independent and thus may be separated into spin up and spin down components

\begin{equation}
e^{-\Delta\tau \mathcal{H}_K} = e^{-\Delta\tau \mathcal{H}_{K_\uparrow}} e^{-\Delta\tau \mathcal{H}_{K_\downarrow}} ,
\end{equation}
where $\mathcal{H}_{K_\sigma} = \bm c_\sigma^\dagger (-t_\sigma \bm K_\sigma - \mu_\sigma \bm I )  \bm c_\sigma$, and we allow spin-dependent hoppings and chemical potential.

The potential energy term, however, is quartic.
Fortunately, it is possible to express it in quadratic form by introducing an extra degree of freedom, the so called \emph{Hubbard-Stratonovich (HS) field} $\bm h \equiv (h_i)_{i=1}^N$, in which each element is essentially an Ising spin.
First, note that number operators on different sites commute, so that we have

\begin{equation}
e^{-\Delta\tau \mathcal{H}_V} = e^{-U \Delta\tau \sum_{i=1}^N (n_{i\uparrow} - 1/2 ) (n_{i\downarrow} - 1/2 )} = \prod_i e^{-U \Delta\tau (n_{i\uparrow} - 1/2 ) (n_{i\downarrow} - 1/2 )}
\end{equation}

Now we introduce the discrete Hubbard Stratonovich transformation for $U > 0$ that allows us to recast the equation above in terms of a non-interacting quadratic term $n_{i\uparrow} - n_{i\downarrow} $.

\begin{equation}\label{eq:discreteHS}
e^{-U \Delta\tau (n_{i\uparrow} - 1/2 ) (n_{i\downarrow} - 1/2 )} = c_U \sum_{h_i = \pm 1} e^{\nu h_i (n_{i\uparrow} - n_{i\downarrow} )},
\end{equation}
where $c_U = \frac{1}{2} e^{-\frac{U\Delta \tau}{4}}$ and $\nu = \text{arcosh} ( e^{\frac{U\Delta\tau}{2}})$.

To prove this identity, let us write down how the operators  $(n_{i\uparrow} - 1/2 ) (n_{i\downarrow} - 1/2 )$ and $(n_{i\uparrow} - n_{i\downarrow} )$ act on a state on a given site.

\begin{equation}
(n_{i\uparrow} - 1/2 ) (n_{i\downarrow} - 1/2 )
\begin{cases}
\left| \, \, \right\rangle = \frac{1}{4} \left| \, \, \right\rangle \\
\left| \uparrow \right\rangle = -\frac{1}{4} \left| \uparrow \right\rangle \\
\left| \downarrow \right\rangle = -\frac{1}{4} \left| \downarrow \right\rangle \\
\left| \uparrow \downarrow \right\rangle = \frac{1}{4} \left| \uparrow \downarrow \right\rangle
\end{cases} \quad
(n_{i\uparrow} - n_{i\downarrow} )
\begin{cases}
\left| \, \, \right\rangle = 0\left| \, \, \right\rangle \\
\left| \uparrow \right\rangle = \left| \uparrow \right\rangle \\
\left| \downarrow \right\rangle = \left| \downarrow \right\rangle \\
\left| \uparrow \downarrow \right\rangle = 0 \left| \uparrow \downarrow \right\rangle
\end{cases}
\end{equation}

Now we simply compare the action of the operators on the left hand side and on the right hand side of equation (\ref{eq:discreteHS}) and find the desired relation by defining

\begin{equation}
\cosh \nu =  \frac{e^\nu + e^{-\nu} }{2} \equiv e^{\frac{U\Delta \tau}{2}}
\end{equation}

\begin{equation}
\begin{split}
&e^{-U \Delta\tau (n_{i\uparrow} - 1/2 ) (n_{i\downarrow} - 1/2 )} \left| \psi \right\rangle = e^{-\frac{U\Delta \tau}{4}} \left| \psi \right\rangle \, , \left| \psi \right\rangle = \left| \, \, \right\rangle, \left| \uparrow \downarrow \right\rangle \\
&e^{-U \Delta\tau (n_{i\uparrow} - 1/2 ) (n_{i\downarrow} - 1/2 )} \left| \uparrow (\downarrow) \right\rangle = e^{\frac{U\Delta \tau}{4}} \left| \uparrow (\downarrow) \right\rangle \\
&c_U \sum_{h_i = \pm 1} e^{\nu h_i (n_{i\uparrow} - n_{i\downarrow} )} \left| \psi \right\rangle = e^{-\frac{U\Delta \tau}{4}} \left| \psi \right\rangle \, , \left| \psi \right\rangle = \left| \, \, \right\rangle, \left| \uparrow \downarrow \right\rangle \\
&c_U \sum_{h_i = \pm 1} e^{\nu h_i (n_{i\uparrow} - n_{i\downarrow} )} \left| \uparrow (\downarrow) \right\rangle= \frac{e^\nu + e^{-\nu}}{2} e^{-\frac{U\Delta \tau}{4}}  \left| \uparrow (\downarrow) \right\rangle
\end{split}
\end{equation}

Note that we require $U > 0$ so that there exists $\nu \in \mathbb{R}$ such that $\cosh \nu = e^{U\Delta \tau / 2}$. A similar reasoning could be made for $U < 0$. Additionaly, other transformations that recast other types of quartic terms in terms of quadratic ones exist, but we shall not need them in what follows \cite{hirsch_monte_1983}. The transformation we derived is the one we will use throughout.

We have now made progress. At the expense of introducing an extra $N$-dimensional HS-field $\bm h$, we obtained an \emph{exact} representation of the quartic term in terms of quadratic terms \cite{hou_numerical_2009}.

\begin{equation} 
 e^{-\Delta\tau \mathcal{H}_V} = \prod_{i=0}^{N-1} \bigg( c_U \sum_{h_i = \pm 1} e^{\nu h_i ( n_{i\uparrow} - n_{i\downarrow} )} \bigg),
\end{equation} 
which can be manipulated to arrive at a more compact form.

\begin{equation}\label{eq:exp_quartic}
\begin{split}
e^{-\Delta\tau \mathcal{H}_V} &=  (c_U)^N \sum_{h_i = \pm 1} e^{\nu h_i ( n_{1\uparrow} - n_{1\downarrow} )} \sum_{h_i = \pm 1} e^{\nu h_i ( n_{2\uparrow} - n_{2\downarrow} )} ... \sum_{h_i = \pm 1} e^{\nu h_i ( n_{N\uparrow} - n_{N\downarrow} )} \\
&= (c_U)^N \sum_{h_i = \pm 1} e^{\sum_{i=0}^{N-1} [(\nu h_i ( n_{i\uparrow} - n_{i\downarrow} ) ]} \equiv (c_U)^N \text{Tr}_{\bm h} e^{\sum_{i=0}^{N-1} [(\nu h_i ( n_{i\uparrow} - n_{i\downarrow} ) ]} \\
&= (c_U)^N \text{Tr}_{\bm h} e^{\sum_{i=0}^{N-1} \nu h_i n_{i\uparrow}} e^{-\sum_{i=0}^{N-1} \nu h_i n_{i\uparrow}} = (c_U)^N \text{Tr}_{\bm h} ( e^{\mathcal{H}_{V_\uparrow}} e^{\mathcal{H}_{V_\downarrow}} ) ,
\end{split}
\end{equation}
where the spin up and spin down operators $\mathcal{H}_{V_\sigma}$ are defined as follows

\begin{equation}
\mathcal{H}_{V\sigma} = \sum_{i=0}^{N-1} \nu h_i n_{i\sigma} = \sigma \nu \bm c_\sigma^\dagger \bm V(\bm h) \bm c_\sigma,
\end{equation}
with $\bm V(\bm h)$ being simply the HS-field put into a diagonal $N\times N$ matrix: $\bm V(\bm h) \equiv \text{diag}(h_0, h_1, ..., h_{N-1})$.

For each imaginary time slice $l$, we may define a HS-field $\bm h_l$, which in turn specifies $\bm V_l$ and $\mathcal{H}_{V_\sigma}^l$.
We may now replace the result of equation (\ref{eq:exp_quartic}) in equation (\ref{eq:Z_propagator}), and exchange the traces to obtain

\begin{equation}\label{eq:Z_quadratic}
Z  = \Tr_{\bm h} \Tr [ e^{-\int_0^\beta S_{\bm h}(\tau) d\tau } ] \approx (c_U)^{NL} \text{Tr}_{\bm h} \Tr \bigg[ \prod_{l=0}^{L-1} \underbrace{\bigg( e^{-\Delta\tau  \mathcal{H}_{K_\uparrow}} e^{\mathcal{H}_{V_\uparrow}^l} \bigg)}_{B_{l, \uparrow}(\bm h_l)} \underbrace{\bigg( e^{-\Delta\tau  \mathcal{H}_{K_\downarrow}} e^{\mathcal{H}_{V_\downarrow}^l} \bigg)}_{B_{l, \downarrow}(\bm h_l)} \bigg] ,
\end{equation}
where all operators are now quadratic in the fermion operators:

\begin{equation}
\mathcal{H}_{K_\sigma} = \bm c_\sigma^\dagger ( - t_\sigma \bm K_\sigma -\mu_\sigma \bm I ) \bm c_\sigma \quad \mathcal{H}_{V_\sigma}^l = \sigma \nu \bm c_\sigma^\dagger \bm V_l (\bm h_l) \bm c_\sigma
\end{equation}
for $\sigma = \pm 1$ and $\bm V_l ( \bm h_l ) = \text{diag} ( h_{l, 0} , h_{l, 1}, ... , h_{l, N-1} )$.

Furthermore, we have defined the $\bm B$-matrices

\begin{equation}
\bm B_{l, \sigma} ( \bm h_l ) = e^{\Delta \tau ( t_\sigma \bm K_\sigma + \mu_\sigma \bm I)} e^{\sigma \nu \bm V_l (\bm h_l)}
\end{equation}

The problem of computing the partition function has been reduced to computing the trace of a product of exponentials of quadratic forms. Thus, we may still rewrite equation (\ref{eq:Z_quadratic}) by making use of the following identity.

Let $\mathcal{H}_l$ be quadratic forms of the fermion operators:

\begin{equation}
\mathcal{H}_l = c_i^\dagger (H_l)_{ij} c_j,
\end{equation}
where the summation is implied, and where $H_l$ are real matrices.
Then, the following identity holds

\begin{equation}\label{eq:quadraticIdentity}
\Tr \big[ e^{-\mathcal{H}_1 } e^{-\mathcal{H}_2 } ... e^{-\mathcal{H}_L } \big] = \text{det} ( \bm I + e^{-H_L} e^{-H_{L-1}} ... e^{-H_1} )
\end{equation}

For simplicity, in appendix \ref{ap:theoAFQMC}, we present the proof for a simpler case, corresponding to a single $\bm B$-matrix, i.e. a product of exponentials of two quadratic operators \cite{hirsch_two-dimensional_1985}.
It could then be easily extended to the more general case \cite{hanke_electronic_nodate}(ch.4, ap. III).
The products of up and down spin $\bm B$-matrices give rise to a product of two such determinants, which may easily be deduced from the proof of appendix \ref{ap:theoAFQMC}.

When applied to our problem, Eq.(\ref{eq:quadraticIdentity}) essentially makes the computation of the trace possible! Note that if we were to compute it naively, we would soon run out of computer memory.
The dimension of the Hilbert space of the Hubbard model is exponential in the number of sites N (actually $4^N$). At worst, the determinant can be calculated in $\mathcal{O}(N^3)$ flops for a matrix whose size is polynomial in $N$. 
Thus, the computable form of the partition function (\ref{eq:Z_quadratic}) is

\begin{equation}\label{eq:effectiveDensityMatrix}
Z =  \Tr_{\bm h} \bigg[ (c_U)^{NL} \text{det} [ \bm M_\uparrow (\bm h)] \text{det} [  \bm M_\downarrow (\bm h) ] \bigg] = \sum_{\{ \bm h \} } P ( \bm h ) \equiv \sum_c p_c
\end{equation}
where the fermion matrices $\bm M_\sigma$ are defined in terms of the $\bm B$-matrices that depend on the HS-field $\bm h$:

\begin{equation}
\bm M_\sigma (\bm h) = \bm I + \bm B_{L,\sigma} ( \bm h_L) \bm B_{L-1,\sigma} ( \bm h_{L-1}) ... \bm B_{1\sigma} ( \bm h_1) = \bm I + \prod_{l= L -1}^0 \bm B_{l,\sigma} ( \bm h_l )
\end{equation}

By casting the fermionic trace as a product of determinants, we obtained the computable approximation of the distribution operator $\mathcal{P}$ corresponding to $Z_{\beta}$ as advertised in Eq.(\ref{eq:Zsign}).

\begin{equation}
P(\bm h) = \frac{A}{Z_{\bm h}} \det [ \bm M_{\uparrow}(\bm h) ] \det [ \bm M_{\downarrow}(\bm h) ] ,
\end{equation}
where $A = (c_U)^{NL}$ is a normalization constant.
This is now a distribution function over configurations $c$ of the field $\bm h$ since the problem is \say{classical} (the quotes serve to emphasize that $p_c$ can be negative)!

For the particular case of no interactions $U = 0$, we have that $\nu = 0$, and $\bm M_\sigma (\bm h)$ are independent of the HS-field. 
The Trotter-Suzuki approximation then becomes exact and the Hubbard Hamiltonian may be simulated exactly after evaluating $\bm M_\sigma (\bm h)$ a single time.
No updates are required.

We mapped a quantum problem to a \say{classical} problem with an extra imaginary-time dimension.
Note that the size of the state space might have been increased to $2^{NL}$ (assuming that $L > 2$), and even if it didn't, it still remains exponential.
However, it can be now be probed more easily: while we might have increased the number of possible configurations by introducing the mapping, we arrived at a form which is tractable by a standard Monte Carlo method, as described in the previous section.
This is because we may now efficiently navigate through the exponentially large state space of the system using importance sampling.
Schematically, the degrees of freedom of the quantum problem correspond to the $i$-indices of the $c$-operators.
In our formulation, an additional imaginary time slice index $l$ was introduced, leading to a mapping that is not specific to the Hubbard model, but applies generally for any quantum system.

\subsection{Monte Carlo sampling of the HS-field}
\label{subsec:mc_hs}

The computational problem is now that of sampling configurations of the $\bm h$ field drawn from the distribution $P(\bm h)$ using \emph{Classical} Monte Carlo.
It remains to choose a dynamics and a sampling scheme. The simplest strategy to change from a configuration $\bm h$ to a new one $\bm h'$ is single spin-flip dynamics. We choose a random point $(l, i)$, and we flip the spin at that space-time  \say{site}:
$
h_{l, i}' = - h_{l, i},
$
keeping all others unchanged.
The most common scheme to ensure that the distribution of the accepted sample is $P(\bm h)$ is the Metropolis-Hastings algorithm, but other choices exist, such as the heat bath algorithm.
After the warm-up steps, we are correctly sampling from the required distribution, and we may perform measurements.

\begin{algorithm}
\caption{Auxiliary Field Quantum Monte Carlo Sampling Scheme}
\label{afqmcSampling}
\begin{algorithmic}[5]
  \STATE Initialize HS field $\bm h$  \\
  \STATE Initialize hoppings $\bm K$  \\
  \STATE  $(h_{l, i}) = (\pm 1)_{l=0, i = 0}^{L-1, N-1}$
  \STATE $(l, i) \leftarrow (0, 0)$
  \FOR{$\text{step} = 1$ to $S$}
  \STATE \footnotesize{Propose new configuration by flipping a spin} \\ \normalsize{$h_{l, i}' = - h_{l, i}$} 
  \STATE \footnotesize{Compute the acceptance ratio $a_{l, i}$} \\
  \normalsize{$\frac{\text{det}[\bm M_\uparrow (\bm h')]\text{det}[\bm M_\downarrow (\bm h')]}{\text{det}[\bm M_\uparrow (\bm h)]\text{det}[\bm M_\downarrow (\bm h)]}$}
  \STATE \textbf{\normalsize{Metropolis step}}
  \STATE \footnotesize{Draw random number $r \in [0,1]$}
  \IF{$r \le \min(1, a_{l, i})$}
  \STATE $\bm h = \bm h'$
  \ELSE
  \STATE $\bm h = \bm h$
  \ENDIF
  \STATE Next space-time \say{site}
  \IF{$i < N - 1$}
  \STATE $l = l$ , $i = i +1 $
  \ELSE
  \IF {$l < L - 1$}
  \STATE $l = l+1$ , $i = 0 $
  \ENDIF
  \IF {$l = L - 1$}
  \STATE $l = 0$ , $i=0$
  \ENDIF
  \ENDIF
  \ENDFOR
\end{algorithmic}
\end{algorithm}

The acceptance/rejection scheme leads to a rank-one update of the matrices $\bm M_\sigma (\bm h)$\footnote{We will see that it is actually more convenient to work with their inverses, the Green's matrices.}, which affords an efficient evaluation of the acceptance ratio $a_{l, i}$ \cite{hou_numerical_2009} (see appendix \ref{ap:theoAFQMC}).
The acceptance ratio is given in terms of determinants of the Green's matrices, but these need not be explicitly computed at each step.
Instead, a \emph{global} update of the Green's matrices at each step suffices to obtain the ratio between the determinants of the Green's matrices of the current and previous configurations.
This brings the computational complexity from $\mathcal{O}(N^3)$ to $\mathcal{O}(N^2)$ at each step.
This results in an overall cubic scaling of the algorithm, more precisely $\mathcal{O}(\beta N^3)$ (the process is repeated $L N$ times, and $\beta \sim L$).

Suppose we start at the first imaginary-time slice, $l = 0$.
Using the result of appendix \ref{ap:theoAFQMC}, for $i = 0$, the proposal $h_{0 0}' = - h_{0 0}$ leads to

\begin{equation}
r_{0 0} = \bigg[ 1 + \alpha_{0, \uparrow} ( 1 - \bm e_0^T \bm M_\uparrow^{-1} ( \bm h ) \bm e_0 ) \bigg] \bigg[ 1 + \alpha_{0, \downarrow} ( 1 - \bm e_0^T \bm M_\downarrow^{-1} ( \bm h ) \bm e_0 ) \bigg] \equiv d_{0, \uparrow} d_{0, \downarrow} ,
\end{equation}
where we defined the unit vectors $\bm e_0, \bm e_1, ..., \bm e_{N-1}$, and the ratios of determinants

\begin{equation*}
d_{i, \sigma} = 1 + \alpha_{i, \sigma} ( 1 - G^\sigma_{i i} ) \quad \text{with} \quad \alpha_{i, \sigma} = e^{-2 \sigma \nu h_{i, i}} - 1
\end{equation*}

The most expensive operation is the computation of the $(0, 0)$ entry of $G^\sigma (\bm h)$.
However, this object is always computed in advance (it is the main object of the simulation!), so the computation of the acceptance ratio is essentially free.
Whenever a step is accepted, the Green's matrices are updated in $\mathcal{O}(N^2)$ flops, and the acceptance ratio is recomputed as our notation suggests

\begin{equation}
\bm G^\sigma ( \bm h ) \leftarrow \bm G^\sigma ( \bm h ) - \frac{\alpha_{i, \sigma}}{r_{i, i}} \bm u_{i, \sigma} \bm w_{i, \sigma}^T \quad \quad r_{l i} = d_{i, \uparrow} d_{i, \downarrow} ,
\end{equation}
where $\bm u_{i, \sigma} = ( \bm I - \bm G^\sigma ( \bm h ) ) \bm e_i$, and $\bm w_{i, \sigma} = [ \bm G^\sigma (\bm h) ]^T \bm e_i$.

Notice that only one entry of each of the Green's matrices is used at each step for sampling.
To improve the efficiency of our implementation, we can compute only the relevant entry that is required for sampling until a sweep of the space lattice is completed, at which point we need to update the entire Green's function (since this final update has complexity $\mathcal{O}(N^3)$, the complexity of the algorithm does not change, although some speed-up is expected).
This block high rank update is a  \say{delayed update} in the sense that we avoid unnecessary computations until they are absolutely needed.

This procedure generalizes for all other time slices.
First, note that the order of the operators in Eq.(\ref{eq:quadraticIdentity}) may be changed by using the cyclic property of the trace.
Concomitantly, for example, when we advance to $l = 1$, we may write the $\bm M$-matrices by wrapping the equivalent $\widetilde{\bm M}$ matrices.

\begin{equation}
\bm M_\sigma ( \bm h ) = \bm B_{0, \sigma}^{-1} ( \bm h_0 ) \widetilde{\bm M}_\sigma (\bm h) \bm B_{0, \sigma} ( \bm h_0 )  \, \quad \widetilde{\bm M}_\sigma (\bm h) = \bm I + \bm B_{0, \sigma} ( \bm h_0 ) \bm B_{L-1, \sigma} ( \bm h_{L-1} ) \bm B_{L-2, \sigma} ( \bm h_{L-2} ) ... \bm B_{1, \sigma} ( \bm h_1 )
\end{equation}

The Metropolis ratio can be computed with $\widetilde{\bm M}$, and the Green's functions are also wrapped :

\begin{equation}
r = \frac{\det[\bm M_\uparrow (\bm h'] \det[\bm M_\downarrow ( \bm h')]}{\det[\bm M_\uparrow (\bm h)] \det[\bm M_\downarrow ( \bm h)} = \frac{\det[\widetilde{\bm M}_\uparrow (\bm h'] \det[\widetilde{\bm M}_\downarrow ( \bm h')]}{\det[\widetilde{\bm M}_\uparrow (\bm h)] \det[\widetilde{\bm M}_\downarrow ( \bm h)} 
\quad\quad
\widetilde{\bm G^\sigma} ( \bm h_0) \leftarrow \bm B_{0, \sigma}( \bm h_0 ) \widetilde{\bm G}^\sigma (\bm h) \bm B_{0, \sigma}^{-1}  ( \bm h_0 )
\end{equation}

The wrapping trick makes $B_{1, \sigma} ( \bm h_1)$ appear at the position of the $\widetilde{\bm M}$-matrix where $\bm B_{0, \sigma} ( \bm h_0)$ appeared for $l = 0$.
Thus, we can use everything that was derived for $l = 0$ with the wrapped Green's functions $\widetilde{\bm G}$.
This is repeated consecutively as we advance in imaginary-time.

Since the cost of wrapping is $\mathcal{O}(N^3)$, the cost of computing $r$ is essentially that of updating the Green's matrices.
Each update requires $2N^2$ elementary operations, so that a sweep through the HS-matrix $\bm h$ costs $2 N^3 L$ flops.
One must pay attention to the efficiency (by delayed updates), and stability of the updating and wrapping of the Green's matrices.
When numerically divergent scales are present, the instability of this scheme must be controlled by computing the Green's functions from scratch periodically.
When doing so, the stability of the product of the (potentially large) chain of $\bm B$-matrices is ensured by using QR decomposition with partial pivoting, following \texttt{QUEST}'s implementation \cite{hou_numerical_2009}.

\subsection{Making measurements}

In QMC simulations, physical observables are extracted by measuring them directly over the course of the sampling of the  configuration space. The single-particle (equal time) Green's Function is useful to obtain quantities such as density and kinetic energy. It turns out that it is simply the inverse of the $\bm M$-matrix that we already compute to obtain the acceptance ratio at each step.

\begin{equation}
G_{ij}^\sigma = \left\langle c_{i,\sigma} c_{j,\sigma}^\dagger \right\rangle_{\bm h} = \bigg( \bm M_\sigma^{-1} (\bm h) \bigg)_{ij} = \bigg( [\bm I + \prod_{l= L -1}^0 \bm B_{l,\sigma} ( \bm h_l ) ]^{-1} \bigg)_{ij}
\end{equation}

The equal time Green's function is a fermion average for a given HS-field configuration \cite{santos_introduction_2003}.

The electron density may be obtained from the Green function

\begin{equation}
\rho_{i, \sigma} = \left\langle c_{i,\sigma}^\dagger c_{i,\sigma} \right\rangle = 1 - \left\langle c_{i,\sigma} c_{i,\sigma}^\dagger \right\rangle = 1 - G_{ii}^\sigma ,
\end{equation}

It is natural to think of averaging this over the lattice.
This is justified by the fact that the Hubbard Hamiltonian is translationally invariant.
Thus, $\rho_{i\sigma}$ should be independent of the spatial site.
This statement is strict when exactly solving the model, but it becomes only approximate, i.e. valid only on average in our simulations.
Thus, we take the average

\begin{equation}
\rho = \frac{1}{N} \sum_\sigma \sum_{i=0}^{N-1} \rho_{i, \sigma} = 2 - \frac{1}{N} \sum_\sigma \sum_{i=0}^{N-1} G_{ii}^\sigma
\end{equation}
in an attempt to reduce statistical errors.

One must pay attention to the symmetry of the model at hand, since a similar model for a disordered system including randomness would not be translationally invariant anymore.
Moreover, it is implicit that $\rho_{i\sigma}$ is already averaged over the HS-field configurations that were sampled through the simulation.

The average kinetic energy is similarly obtained.

\begin{equation}
\left\langle \mathcal{H}_K \right\rangle = - t  \sum_{\left\langle i, j \right\rangle , \sigma} \left\langle ( c_{i\sigma}^\dagger c_{j\sigma} + c_{j\sigma}^\dagger c_{i\sigma} ) \right\rangle = t \sum_{\left\langle i, j \right\rangle , \sigma} ( G_{ij}^\sigma + G_{ji}^\sigma ) = t \sum_{ i, j , \sigma} K_{ij} ( G_{ij}^\sigma + G_{ji}^\sigma )  ,
\end{equation}
where the minus sign is due to the switching of the order of the operators bringing the $c^\dagger$ to the right.

\subsection{Correlation functions}

One of the most important goals of QMC simulations is to inspect the system for order of various types, and to find  associated phase transitions. This is done by computing correlation functions $C (j) $, measuring how correlated two sites separated by a distance $j$ are.

\begin{equation}
C(j) = \big\langle \mathcal{O}_{i+j} \mathcal{O}_{i}^\dagger \big\rangle - \langle \mathcal{O}_{i+j} \big\rangle\big\langle\mathcal{O}_{i}^\dagger \big\rangle ,
\end{equation}
where $\mathcal{O}$ is an operator corresponding to the order parameter of the phase transition. For example, we might be looking for magnetic order, in which case the relevant operators are $S^z_i$, i.e. $\mathcal{O}_i = n_{i\uparrow} - n_{i\downarrow} \, , \, \mathcal{O}_i^\dagger = n_{i\uparrow} - n_{i\downarrow}$, or superconductivity, where we would like to measure correlations in fermion pair formation: $\mathcal{O}_i = c_{i\downarrow} c_{i\uparrow} \, , \, \mathcal{O}_i^\dagger = c_{i\uparrow}^\dagger c_{i\downarrow}^\dagger$.

In general, we expect a high temperature disordered phase, for which correlations decay exponentially $C(j) \propto e^{-j/\xi}$, where $\xi$ is a characteristic length called the correlation length. At some point, there can be a transition to a low temperature phase, where $C(j) \propto m^2$, where $m$ is the order parameter for the transition. Right at the transition, that is at $T = T_c$, there might be singular behavior. In continuous phase transitions, the correlation length diverges $\xi \propto (T-T_c)^{-\nu}$, and the correlations decay slower (in fact algebraically): $C(j) \propto j^{-\eta}$, in an intermediate behavior between exponential decay and a constant. The \emph{critical} exponents $\nu$, and $\eta$ are characteristic of the transition, or more accurately, of the universality class it belongs to.

The behavior of all these quantities on finite lattices does not precisely correspond to the infinite system behavior. The tails of the functions, i.e. the $j\rightarrow \infty$ limit is not well captured. Finite-size scaling is a method to improve on these predictions.

To evaluate correlation functions we use Wick's theorem. Expectations of more than two fermion creation and annihilation operators reduce to products of expectations of pairs of creation and annihilation operators. For example, for spin order in $x/y$ direction:

\begin{equation}
\big\langle C(j) \big\rangle = \big\langle c_{i+j, \downarrow}^\dagger c_{i+j, \uparrow} c_{i, \uparrow}^\dagger c_{i, \downarrow} \big\rangle = G_{i+j, i}^\uparrow G_{i, i + j}^\downarrow
\end{equation}

How would one measure a correlation function experimentally? Fortunately, there is a quantity that is easy to measure called structure factor, which is just the Fourier transform of the correlation function

\begin{equation}
S(\bm q) = \sum_j e^{i\bm q \cdot \bm R_j} C(j) 
\end{equation}

The accuracy of QMC simulations can be evaluated by comparing the results for the Fourier transformed correlation functions with the corresponding experimentally measured structure factors.
\cleardoublepage
